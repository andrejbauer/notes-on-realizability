\chapter{The Internal Language}
\label{chap:internal-language}

In Chapter~\ref{chap:realizability} we defined assemblies as our basic
structures of interest. To learn more about them we study which
operations on assemblies are supported by the category $\AsmA$. It
turns out that there is not only a rich collection of constructions
available, but also an interpretation of first-order logic that
assigns to a logical formula its computational meaning.

\section{Constructions of assemblies}
\label{sec:constructions}

In their everyday lives mathematicians use a limited set of
constructions on sets: products, disjoint sums, subsets, quotient
sets, images, function spaces, inductive and coinductive definitions,
powersets, unions, intersections, and complements. For all of these,
except powersets, there are analogous constructions of assemblies.
Therefore, if we need a computable version of a given mathematical
structure, may just mimic its set-theoretic definition in the
category of assemblies.

\subsection{The cartesian structure}
\label{sec:cartesian-structure}

A construction which makes a new set, space, or an algebraic structure
from old ones is usually characterized by a \emph{universal property}
which determines it up to isomorphism. The universal property is
shared among versions of the same construction in different
categories. We start slowly with an easy one, the binary product.

Recall the definition of a \emph{(binary) product} in a category: the
product of objects $S$ and $T$ is an object $P$ with morphisms $p_1 :
P \to S$ and $p_2 : P \to T$, satisfying the following universal
property: for all morphisms $f : U \to S$ and $g : U \to T$ there is a
unique morphism $h : U \to P$ making the following diagram commute:
%
\begin{equation*}
  \xymatrix@+1em{
    &
    {U}
    \ar[ld]_{f}
    \ar[d]^h
    \ar[rd]^{g}
    &
    \\
    {S}
    &
    {P}
    \ar[l]^{p_1}
    \ar[r]_{p_2}
    &
    {T}
  }
\end{equation*}
%
The products $(P, p_1, p_2)$ is determined uniquely up to a unique
isomorphism. For suppose we had another product $(Q, q_1, q_2)$ of~$S$
and~$T$. By the universal property of $P$ there is a map $h : P \to Q$
such that $p_1 \circ h = q_1$ and $p_2 \circ h = q_2$. Similarly, by
the universal property of~$Q$ there is $k : Q \to P$ such that $q_1
\circ k = p_1$ and $q_2 \circ k = p_2$. Now $h \circ k$ satisfies
%
\begin{equation*}
  p_1 \circ h \circ k = q_1 \circ k = p_1
  \qquad\text{and}\qquad
  p_2 \circ h \circ k = q_2 \circ k = p_2.
\end{equation*}
%
Since $\id_P$ also satisfies $p_1 \circ \id_P = p_1$ and $p_2 \circ
\id_P = p_2$, it follows by the uniqueness condition of the universal
property that $h \circ k = \id_P$. A similar argument shows that $k
\circ h = \id_Q$, hence $P$ and $Q$ are isomorphic.

A category \emph{has binary products} if every pair of objects has a
binary product. In most cases we can actually provide an operation
$\times$ which maps a pair of objects $S$, $T$ to a specifically given
product $S \times T$ with corresponding projections. The unique map
$h$ determined by $f$ and $g$ is denoted by $\pair{f,g}$.

In the category~$\Set$ the product is just the usual cartesian product
of sets. In assemblies we need to worry about the underlying types and
realizability relations. Let us verify that the product of assemblies
$\asm{S}$ and $\asm{T}$ is the assembly
%
\begin{equation*}
  \asm{S} \times \asm{T} =
  (S \times T, |S| \times |T|, {\rz_{S \times T}})
\end{equation*}
%
with the realizability relation
%
\begin{equation*}
  \combPair\;q\;r \rz_{S \times T} (x,y)
  \iff
  q \rz_S x
  \land
  r \rz_T y.
\end{equation*}
%
and the projection maps $\pi_1 : S \times T \to S$, $\pi_1 : (x,y)
\mapsto x$, and $\pi_2 : S \times T \to T$, $\pi_2 : (x,y) \mapsto y$,
which are realized by $\combFst$ and $\combSnd$, respectively.
%
To see that $(\asm{S} \times \asm{T}, \pi_1, \pi_2)$ has the universal
property, suppose $f : \asm{U} \to \asm{S}$ and $g : \asm{U} \to
\asm{T}$ are realized by $\R{f} \in \comp{A}_{|U| \to |S|}$ and $\R{g}
\in \comp{A}_{|U| \to |T|}$, respectively. There is a unique map $h :
U \to S \times T$ for which $f = \pi_1 \circ h$ and $g = \pi_2 \circ g$,
namely $h(u) = \pair{f,g}(u) = (f(u), g(u))$. We only need a realizer
for~$h$, and $\R{h} =
\pcalam{\annot{u}{|U|}}{\combPair\;(\R{f}\;u)\;(\R{g}\;u)}$ does the
job:
%
\begin{multline*}
  \R{u} \rz_U u
  \implies
  \R{f}\;\R{u} \rz_S f(u)
  \land
  \R{g}\;\R{u} \rz_T g(u)
  \implies {} \\
  \combPair\; (\R{f}\;\R{u})\; (\R{g}\;\R{u}) \rz_{S \times T} (f(u), g(u))
  \iff
  \R{h}\;\R{u} \rz_{S \times T} h(u).
\end{multline*}

Once we have binary products we may form $n$-ary products $S_1 \times
\cdots \times S_n$ for $n \geq 1$ by repeatedly forming binary
products. The case $n = 0$ corresponds to the \emph{terminal object},
which is an object $\one$ such that for every object~$S$ there is
exactly one morphism $S \to \one$. In the category of sets the
terminal object is (any) singleton set, say $\one = \set{\star}$. Then
$\nabla \one$ is the terminal assembly, since for any assembly
$\asm{S}$ the only map $S \to \one$ is realized. We denote the
terminal assembly as $\one$.

We may also ask whether $\AsmA$ has infinite products. The answer
depends on the underlying TPCAs $(\AA, \compAA)$. We state without
proof that $\Asm{\Scott, \comp{\Scott}}$ and $\Asm{\Baire,
  \comp{\Baire}}$ have countable products, whereas $\Asm{\NN}$ does
not.

Products are a special case of categorical limits. Two other common
kinds of limits are equalizers and pullbacks. An \emph{equalizer} of a
pair of morphisms $f, g : S \to T$ is an object $E$ with a morphism $e
: E \to S$ such that $e$ equalizes $f$ and $g$, which means that $f
\circ e = g \circ e$, and the following universal property is
satisfied: if $k : K \to S$ also equalizes $f$ and $g$ then there
exists a unique morphism $i : K \to E$ such that $k = e \circ i$:
%
\begin{equation*}
  \xymatrix@+1em{
    {E}
    \ar[r]^{e}
    &
    {S}
    \ar@<+0.25em>[r]^{f}
    \ar@<-0.25em>[r]_{g}
    &
    {T}
    \\
    {K}
    \ar[u]^{i}
    \ar[ru]_{k}
    & & 
  }
\end{equation*}
%
Think of $E$ as the solution-set of equation $f(x) =
g(x)$.\footnote{If we have a system of equations $f_i(x) = g_i(x)$, $i
  = 1, \ldots, n$, then we may express them as a single vector
  equation $f(x) = g(x)$ where $f(x) = (f_1(x), \ldots, f_n(x))$ and
  $g(x) = (g_1(x), \ldots, g_n(x))$. Equalizers are thus an abstract
  formulation of the notion ``solution of a system of equations''.}
Indeed, in the category of sets the equalizer of functions $f, g : S
\to T$ is the subset $E = \set{x \in S \such f(x) = g(x)}$ and $e : E
\to S$ is the subset inclusion. In the category of assemblies we need
to augment this with realizers. The equalizer of $f, g : \asm{S} \to
\asm{T}$ is
%
\begin{equation}
  \label{eq:asm-equalizer}%
  \asm{E} =
  (\set{x \in S \such f(x) = g(x)},
   |S|,
   {\rz_E})
\end{equation}
%
where $\R{x} \rz_E x$ if, and only if, $\R{x} \rz_S x$. The map $e : E
\to S$ is the subset inclusion, $e(x) = x$. It is realized by
$\xpcalam{\annot{x}{|S|}}{x}$. Clearly, $e$ equalizes~$f$ and~$g$. We
leave the verification of the universal property as exercise.

A \emph{pullback}, sometimes called \emph{fibered product}, is a
combination of product and equalizer. Given morphisms $f : S \to U$
and $g : T \to U$, the pullback of $f$ and $g$ is an object $P$ with
morphisms $p_1 : P \to S$ and $p_2 : P \to T$ such that $f \circ p_1 =
g \circ p_2$. Furthermore, if $q_1 : Q \to S$ and $q_2 : Q \to T$ are
such that $f \circ q_1 = g \circ q_2$ then there is a unique $i : Q
\to P$ which makes the following diagram commute:
%
\begin{equation*}
  \xymatrix@+0.5em{
    {Q}
    \ar[dr]^{i}
    \ar@/_1em/[ddr]_{q_2}
    \ar@/^1em/[rrd]^{q_1}
    &
    &
    \\
    &
    {P} \pbcorner
    \ar[r]^{p_1}
    \ar[d]_{p_2}
    &
    {S}
    \ar[d]^{f}
    \\
    &
    {T}
    \ar[r]_{g}
    &
    {U}
  }
\end{equation*}
%
The fact that $P$ is a pullback is traditionally marked in a diagram
with the ``corner'' symbol. In the category of assemblies the pullback
of $f : \asm{S} \to \asm{U}$ and $g : \asm{T} \to \asm{U}$ is the
assembly
%
\begin{equation*}
  \asm{P} = (\set{(x,y) \in S \times T \such f(x) = g(y)}, |S| \times
  |T|, {\rz_P})
\end{equation*}
%
where $\combPair\;\R{x}\;\R{y} \rz_P (x,y)$ if, and only if, $\R{x}
\rz_S x$ and $\R{y} \rz_T y$.

Finite products, the terminal object, equalizers, and pullbacks are
special cases of \emph{finite limits}. A category which has all finite
limits is called \emph{cartesian} or \emph{finitely
  complete}.\footnote{We do not like much the still older terminology
  \emph{left exact} or just \emph{lex}.}

\begin{proposition}
  The categories $\AsmA$ and $\Mod{\AA, \compAA}$ are cartesian.
\end{proposition}

\begin{proof}
  It is well known that every finite limit may be constructed as a
  combination of a finite product and an equalizer, hence $\AsmA$ is
  cartesian. It is easy to verify that finite products and equalizers
  of modest assemblies are again modest, therefore $\Mod{\AA,
    \compAA}$ is cartesian.
\end{proof}


%%%%%%%%%%%%%%%%%%%%%%%%%%%%%%%%%%%%%%%%%%%%%%%%%%
\subsection{Cocartesian structure}
\label{sec:cocartesian-structure}


Colimits are the dual of limits. In particular, the dual of products,
terminal object, equalizers, and pullbacks are respectively
coproducts, initial object, coequalizers, and pushouts. We study which
of these exist in~$\AsmA$.

First we discuss (binary) coproducts of sets, also known as disjoint
sums. For some reason there does not seem to be a well-established and
practical notation for these, possibly because the related union
operation is taken as primitive in set theory. The disjoint sum of
sets~$S$ and~$T$ is usually defined as
%
\begin{equation*}
  S + T = (\set{0} \times S) \cup (\set{1} \times T).
\end{equation*}
%
The canonical injections $\iota_1 : S \to S + T$ and $\iota_2 : T \to
S + T$ are the maps $x \mapsto (0,x)$ and $y \mapsto (1,x)$,
respectively. A slight notational inconvenience arises when want to
define a map $f : S + T \to U$ by cases $f_1 : S \to U$ and $f_2 : T
\to U$. One possibility is to write
%
\begin{equation*}
  f(u) =
  \begin{cases}
    f_1(x) & \text{if $u = (0,x)$,}\\
    f_2(y) & \text{if $u = (1,y)$,}
  \end{cases}
\end{equation*}
%
but this is seen rarely. In practice mathematicians prefer to assume,
or shall we say \emph{pretend}, that the sets~$S$ and~$T$ are disjoint
and just write $S + T = S \cup T$. This allows us to get rid of the
encoding by pairs,
%
\begin{equation*}
  f(u) =
  \begin{cases}
    f_1(u) & \text{if $u \in S$,}\\
   f_2(u) & \text{if $u \in T$.}
  \end{cases}
\end{equation*}
%
Unlike people, computers do not pretend, and so as computer scientists
we need notation that is actually correct. However, it is unnecessary
to write the elements of a disjoint sum as uninformative pairs $(0,x)$
and $(1,y)$. Instead, we simply take the injections $\iota_1$ and
$\iota_2$ as \emph{labels} that indicate which part of a disjoint sum
we are referring to. Thus, every element of $S + T$ is either of the
form $\iota_1(x)$ for a unique $x \in S$, or $\iota_2(y)$ for a unique
$y \in T$.\footnote{If you feel the urge to really encode everything
  with sets, you can still define $\iota_1(x) = (0, x)$ and
  $\iota_2(y) = (1, y)$ but then forget the definition.} In a specific
case we may choose different, descriptive names for the injections.

Definition by cases is a primitive concept involving disjoint sums
which deserves its own notation, preferably one that fits on a single
line. We may mimic Haskell and write
%
\begin{equation*}
  \case{e}{\iota_1(x)}{e_1}{\iota_2(y)}{e_2}.
\end{equation*}
%
Read this as ``if $e$ is of the form $\iota_1(x)$ then $e_1$, else if
$e$ is of the form $\iota_2(y)$ then $e_2$''. The variables $x$ and
$y$ are bound in $e_1$ and $e_2$, respectively. The definition of $f$
above would be written as
%
\begin{equation*}
  f(u) = \case{u}{\iota_1(x)}{f_1(x)}{\iota_2(y)}{f_2(y)},
\end{equation*}
%
or spanning several lines
%
\begin{equation*}
  f(u) = \xcase{u}{\iota_1(x)}{f_1(x)}{\iota_2(y)}{f_2(y).}
\end{equation*}
%
We shall use this notation. Let us mention that in Haskell $\iota_1$
and $\iota_2$ are called $\mathtt{Left}$ and $\mathtt{Right}$,
respectively.

In a general category a \emph{(binary) coproduct} of objects~$S$
and~$T$ is an object $C$ with morphisms $\iota_1 : S \to C$ and
$\iota_2 : T \to C$ such that, for all morphisms $f : S \to U$ and $g
: T \to U$ there exists a unique $h : C \to U$ such that the following
diagram commutes:
%
\begin{equation*}
  \xymatrix@+1em{
    &
    {U}
    &
    \\
    {S}
    \ar[r]_{\iota_1}
    \ar[ru]^{f}
    &
    {C}
    \ar[u]^{h}
    &
    {T}
    \ar[l]^{\iota_2}
    \ar[lu]_{g}
  }
\end{equation*}
%
Notice that we have exactly reversed all the morphisms with respect to
the definition of products. We write the coproduct of $S$ and $T$ as
$S + T$ when it is given as an operation, and the unique morphism~$h$
as~$[f,g]$.

Whether assemblies $\AsmA$ have binary coproducts is an interesting
question. The answer seems to depend on the structure of the
underlying TPCAs.

\begin{definition}
  A TPCA~$\AA$ with \emph{sums} is a TPCA with a binary operation $+$
  on the types such that, for all types $s$, $t$, and $u$ there exist
  constants
  %
  \begin{align*}
    \combLeft_{s,t} &\in \Atyp{s \to (s + t)} \\
    \combRight_{s,t} &\in \Atyp{t \to (s + t)} \\
    \combCase_{s,t,u} &\in \Atyp{(s + t) \to (s \to u) \to (t \to u) \to u}
  \end{align*}
  %
  satisfying, for all $x$, $y$, $f$, $g$ of appropriate types,
  %
  \begin{align*}
    \defined{\combLeft_{s,t}\;x} &, \\
    \defined{\combRight_{s,t}\;y} &,\\
    \combCase_{s,t,u}\;(\combLeft_{s,t}\;x)\;f\;g &\klgeq f\;x, \\
    \combCase_{s,t,u}\;(\combRight_{s,t}\;y)\;f\;g &\klgeq g\;y.
  \end{align*}
  %
  We say that the elements $\combLeft$, $\combRight$, and $\combCase$
  are \emph{suitable} for sums when they satisfy these properties.

  A \emph{sub-TPCA with sums} is a sub-TPCA $\compAA$ of~$\AA$ such
  that there exists $\combLeft$, $\combRight$, $\combCase$
  in~$\compAA$ suitable for sums in $\AA$.
\end{definition}

\begin{proposition}
  \label{prop:asm-coproducts-iff-tpca-sums}
  Suppose $\AA$ is a TPCA and $\compAA$ its sub-TPCA. The category
  $\AsmA$ has binary coproducts if, and only if, $\AA$ is a TPCA with
  sums and $\compAA$ is its sub-TPCA with sums.
\end{proposition}

\begin{proof}
  Suppose first that $\AA$ has sums and that $\compAA$ is a sub-TPCA
  with sums. The coproduct of $\asm{S}$ and $\asm{T}$ is the assembly
  %
  \begin{equation*}
    \asm{S} + \asm{T} = (S + T, |S| + |T|, \rz_{S + T})
  \end{equation*}
  %
  where $\rz_{S+T}$ is most easily defined in terms of the existence
  predicate $\Ex_{S+T}$:
  %
  \begin{equation*}
    \Ex_{S+T}(u) =
    \xcase{u}
    {\iota_1(x)}{\set{\combLeft\;\R{x} \such \R{x} \rz_S x}}
    {\iota_2(y)}{\set{\combRight\;\R{y} \such \R{y} \rz_T y}.}
  \end{equation*}
  %
  That is, the realizers for $\iota_1(x)$ are of the form
  $\combLeft\;\R{x}$ where $\R{x} \rz_S x$, and the realizers for
  $\iota_2(y)$ are of the form $\combRight\;\R{y}$ where $\R{y} \rz_T
  y$. The canonical inclusions $\iota_1 : S \to S + T$ and $\iota_2 :
  T \to S + T$ are realized by $\combLeft_{|S|,|T|}$ and
  $\combRight_{|S|,|T|}$, respectively.
  %
  To see that $\asm{S} + \asm{T}$ has the required universal property,
  consider $f : \asm{S} \to \asm{U}$ and $g : \asm{T} \to \asm{U}$,
  realized by $\R{f}$ and $\R{g}$, respectively. The map $h = [f,g] :
  S + T \to U$, defined by
  %
  \begin{equation*}
    h(u) = \case{u}{\iota_1(x)}{f(x)}{\iota_2(y)}{g(y)},
  \end{equation*}
  %
  is realized by
  $\pcalam{\annot{u}{|S|+|T|}}{\combCase\;u\;\R{f}\;\R{g}}$. It is the
  unique morphism satisfying $h \circ \iota_1 = f$ and $h \circ
  \iota_2 = g$.

  Conversely, suppose $\AsmA$ has binary coproducts. For every type
  $t$, define the assembly
  %
  \begin{equation*}
    \asm{A}_t = (\Atyp{t}, t, {\rz_t})
  \end{equation*}
  %
  with $r \rz_t q \iff r = q$. For types $s$ and $t$ let $s+t$ be the
  underlying type of the coproduct $\asm{A}_s + \asm{A}_t$,
  %
  \begin{equation*}
    s + t = |\asm{A}_s + \asm{A}_t|.
  \end{equation*}
  %
  Let $\combLeft_{s,t}$ and $\combRight_{s,t}$ be a realizers for the
  canonical inclusions $\iota_1 : \asm{A}_s \to \asm{A}_s + \asm{A}_t$
  and $\iota_2 : \asm{A}_t \to \asm{A}_s + \asm{A}_t$, respectively.

  Suppose $s$, $t$, and $u$ are types. Define $a \in \compAA_{s \to (s
    \to u) \to (t \to u) \to u}$ and $b \in \compAA_{t \to (s \to u)
    \to (t \to u) \to u}$ by
  %
  \begin{equation*}
    a = \pcalam{\annot{x}{s}
                 \annot{f}{s \to u}
                 \annot{g}{t \to u}}
                 {f\;x}
   \qquad\text{and}\qquad
    b = \pcalam{\annot{x}{s}
                 \annot{f}{s \to u}
                 \annot{g}{t \to u}}
                 {g\;x}.
  \end{equation*}
  %
  The map $x \mapsto a\;x$ is a morphism from $\asm{A}_s$ to
  $\asm{A}_{(s \to u) \to (t \to u) \to u}$ because it is realized
  by~$a$. Similarly, the map $y \mapsto b\;y$ is a morphism from
  $\asm{A}_t$ to $\asm{A}_{(s \to u) \to (t \to u) \to u}$, realized
  by~$b$. There is a unique morphism $h : \asm{A}_s + \asm{A}_t \to
  \asm{A}_{(s \to u) \to (t \to u) \to u}$ such that $h(\iota(x)) =
  a\;x$ and $h(\iota(y)) = b\;y$ for all $x \in \asm{A}_s$ and $y \in
  \asm{A}_t$. There exists
  %
  \begin{equation*}
    \combCase_{s,t,u} \in \compAtyp{(s+t) \to (s \to u) \to (t \to u) \to u}
  \end{equation*}
  %
  which realizes~$h$. We claim that $\combLeft_{s,t}$,
  $\combRight_{s,t}$, and $\combCase_{s,t,u}$ have the desired
  properties. It is obvious that $\defined{\combLeft_{s,t}\;x}$ and
  $\defined{\combRight_{s,t}\;y}$ for all $x \in \Atyp{s}$, $y \in
  \Atyp{t}$. Next, because $\combCase_{s,t,u}$ realizes $h$,
  $\combLeft_{s,t}$ realizes $\iota_1$, and $h(\iota_1(x)) = a\;x$, we
  have $\combCase_{s,t,u}\;(\combLeft_{s,t}\;x) = a\;x$, therefore
  %
  \begin{equation*}
    \combCase_{s,t,u}\;(\combLeft_{s,t}\;x)\;f\;g \kleq
    a\;x\;f\;g \kleq f\;x
  \end{equation*}
  %
  for all $x$, $f$, and $g$ of relevant types. Similarly,
  $\combCase_{s,t,u}\;(\combRight_{s,t}\;y)\;f\;g \kleq g\;y$ holds
  as well.
\end{proof}

The obvious question to ask is when a TPCA has sums. We do not know
whether there is a TPCA without sums, and we do not explore the
question further. We satisfying ourselves with a sufficient condition
that covers the instances we care about.

\begin{definition}
  \label{def:tpca-booleans}%
  A TPCA $\AA$ has \emph{booleans} when there is a type $\ttbool$, and
  for each type~$t$ elements
  %
  \begin{equation*}
    \combFalse, \combTrue \in \Atyp{\ttbool}
    \qquad\text{and}\qquad
    \combIf_t \in \Atyp{\ttbool \to t \to t \to t}
  \end{equation*}
  %
  satisfying, for all $x, y \in \Atyp{t}$,
  %
  \begin{equation*}
    \combIf_t\;\combTrue\;x\;y = x
    \qquad\text{and}\qquad
    \combIf_t\;\combFalse\;x\;y = y.
  \end{equation*}
  %
  We say that $\combFalse$, $\combTrue$, $\combIf_t$ are
  \emph{suitable} for booleans in~$\AA$.

  A sub-TPCA with booleans is a sub-TPCA $\compAA$ of~$\AA$ such that
  there exists $\combFalse$, $\combTrue$, $\combIf_t$ in $\compAA$
  which are suitable for booleans in~$\AA$.
\end{definition}


\begin{proposition}
  \label{prop:tpca-sums-iff-booleans}
  A TPCA $\AA$ has sums if, and only if, it has booleans. Furthermore,
  a sub-TPCA~$\compAA$ is a sub-TPCA with sums if, and only if, it is
  a sub-TPCA with booleans.
\end{proposition}

\begin{proof}
  Suppose $\AA$ has sums. Pick any type $o$, an element $\omega_o \in
  \Atyp{o}$, and define
  %
  \begin{align*}
    \ttbool &= o + o, \\
    \combTrue &= \combLeft\;\omega_o, \\
    \combFalse &= \combRight\;\omega_o \\
    \combIf_t &= \pcalam{\annot{b}{\ttbool}
                         \annot{x}{t}
                         \annot{y}{t}}
                        {\combCase_{o,t,t}\;b\;(\combK_{o,t}\;x)\;(\combK_{o,t}\;y)}
  \end{align*}
  %
  It is easy to check that these satisfy the conditions from
  Definition~\ref{def:tpca-booleans}.

  Conversely, suppose $\AA$ has booleans, and let $s$, $t$, and $u$ be
  types. There exist $\omega_s \in \Atyp{s}$ and $\omega_t \in
  \Atyp{t}$. Define
  %
  \begin{align*}
    s + t &= \ttbool \times (s \times t),\\
    \combLeft_{s,t} &=
    \pcalam{\annot{x}{s}}{\combPair\;\combTrue\;(\combPair\;x \; \omega_t)} \\
    \combRight_{s,t} &=
    \pcalam{\annot{y}{t}}{\combPair\;\combFalse\;(\combPair \; \omega_s \; y)} \\
    \combCase_{s,t,u} &=
    \pcalam{\annot{z}{s+t}
            \annot{f}{s \to u}
            \annot{g}{s \to u}}
    {\combIf\;(\combFst\;z)\;(f\;(\combFst\;(\combFst\;z)))\;(g\;(\combSnd\;(\combFst\;z)))}
  \end{align*}
  %
  These have the required properties, as is easily checked.
\end{proof}

\begin{proposition}
  \label{prop:n-tpca-booleans}
  Every N-TPCA has booleans and every sub-N-TPCA is a sub-N-TPCA with
  booleans.
\end{proposition}

\begin{proof}
  Let $\AA$ be a N-TPCA and $\compAA$ its sub-N-TPCA. Define
  %
  \begin{align*}
    \ttbool &= \ttnat, \\
    \combFalse &= \numeral{0}, \\
    \combTrue &= \numeral{1}, \\
    \combIf_t &= \pcalam{\annot{b}{\ttbool}
                         \annot{x}{t}
                         \annot{y}{t}}{
                         \combRec_t \; y \;
                         (\xpcalam{\annot{n}{\ttnat} \annot{z}{t}}{x}) \; b 
                       }
  \end{align*}
  %
  Again, it is easy to check that these have the desired properties.
\end{proof}

\noindent
Finally, let us put all these together.

\begin{proposition}
  If $\AA$ is a N-TPCA and $\compAA$ its sub-N-TPCA then $\AsmA$ has
  binary coproducts.
\end{proposition}

\begin{proof}
  Combine Propositions~\ref{prop:asm-coproducts-iff-tpca-sums},
  \ref{prop:tpca-sums-iff-booleans}, and \ref{prop:n-tpca-booleans}.
\end{proof}

The other finite colimits are more easily dealt with. The initial
object is the \emph{empty assembly}
%
\begin{equation*}
  \zero = (\emptyset, o, \rz_\zero)
\end{equation*}
%
where $o$ is any type.\footnote{If you are wondering what $\rz_\zero$
  is, ask yourself how many relations there are between $\Atyp{o}$ and
  $\emptyset$.} Its universal property is that there is exactly one
morphism $\zero \to \asm{S}$ for every assembly~$\asm{S}$. The
property holds because there is a unique map $\emptyset \to S$, which
is realized by $\combK_{|S|,o}\;a$, where $a \in \compAtyp{o}$.

A \emph{coequalizer} of morphisms $f, g : S \to T$ is an object $Q$
with a morphism $q : T \to Q$ that equalizes~$f$ and $g$, which means
$q \circ f = q \circ g$, and has the following universal property: if
$k : T \to K$ equalizes~$f$ and~$g$ then there is a unique morphism $i
: Q \to K$ such that $k = i \circ q$:
%
\begin{equation*}
  \xymatrix@+1em{
    {S}
    \ar@<+0.25em>[r]^{f}
    \ar@<-0.25em>[r]_{g}
    &
    {T}
    \ar[rd]_{k}
    \ar[r]^{q}
    &
    {Q}
    \ar[d]^{i}
    \\
    & & 
    {K}
  }
\end{equation*}
%
In the category of sets the coequalizer is the quotient $Q =
T/{\equiv}$ of~$T$ by the least equivalence relation~$\equiv$
satisfying $f(x) \equiv g(x)$ for all $x \in S$. The map $q : T \to
T/{\equiv}$ is the canonical quotient map which takes $y \in T$ to its
equivalence class $[y]_{\equiv}$.

In $\AsmA$ the coequalizer of $f, g : \asm{S} \to \asm{T}$ is the
assembly $\asm{T}/{\equiv}$ where $T/{\equiv}$ is the coequalizer of
$f$ and $g$ computed in sets, as described above, $|\asm{T}/{\equiv}|
= |T|$, and $\rz_{T/{\equiv}}$ is defined by
%
\begin{equation*}
  \R{y} \rz_{T/{\equiv}} [z]_{\equiv} \iff
  \xsome{y}{T}{\R{y} \rz_T y \land y \equiv z}.
\end{equation*}
%
The canonical quotient map $q : T \to T/{\equiv}$ is realized by
$\xpcalam{\annot{y}{|T|}}{y}$.

Pushouts, which are dual to pullbacks, exist in~$\AsmA$ if coproducts
do, because every finite colimit is a coequalizer of a finite
coproduct, as recorded in the following proposition. We leave an
explicit description of pushouts as exercise.

\begin{proposition}
  \label{prop:asm-cocartesian}%
  $\AsmA$ is cocartesian\footnote{A category is \emph{cocartesian} or
    \emph{finitely cocomplete} if it has finite colimits.} if, and
  only if, $\AA$ is a TPCA with sums and $\compAA$ a sub-TPCA with
  sums.
\end{proposition}

\begin{proof}
  Coequalizers and the initial object always exist. Therefore, all
  finite colimits exist, provided binary coproducts do. By
  Proposition~\ref{prop:asm-coproducts-iff-tpca-sums} this is
  equivalent to the condition that $\AA$ have sums and $\compAA$ be a
  sub-TPCA with sums.
\end{proof}


\subsection{Monos and epis}
\label{sec:monos-epis}

Recall that $f$ is a \emph{monomorphism (mono)} when it can be
canceled on the left: if $f \circ g = f \circ h$ then $g = h$. The
dual notion is \emph{epimorphism (epi)}, which is a morphism that can
be canceled on the right. In the category of sets the monos and epis
are precisely the injective and surjective maps, respectively.

\begin{proposition}
  \label{prop:asm-mono-epi-when}%
  A morphism in $\AsmA$ is mono if, and only if, it is mono as a map
  in $\Set$, and likewise for epis.
\end{proposition}

\begin{proof}
  It is obvious that a morphism $f : \asm{S} \to \asm{T}$ is a mono in
  $\AsmA$ if it is mono in $\Set$. Conversely, suppose $f$ is mono in
  $\AsmA$, and consider maps $g, h : U \to T$ in $\Set$ such that $f
  \circ g = f \circ h$. Define the assembly $\asm{U} = (U, |S| \times
  |S|, {\rz_U})$ with the realizability relation
  % 
  \begin{equation*}
    \combPair\;\R{x}\;\R{y} \rz_U u
    \iff \R{x} \rz_S f(u) \land \R(y) \rz_S g(u).
  \end{equation*}
  % 
  The maps $g$ and $h$ are morphisms from $\asm{U}$ to $\asm{S}$
  because they are realized by $\combFst$ and $\combSnd$,
  respectively. Since $f$ is mono as a morphism of assemblies, it
  follows that $g = h$.

  Next we consider epis. Again, it is easy to see that a morphism $f :
  \asm{S} \to \asm{T}$ is epi if it is epi in $\Set$. Conversely,
  suppose~$f$ is epi in $\AsmA$ and consider maps $g, h : T \to U$
  in~$\Set$ such that $g \circ f = h \circ f$. The maps $g$ and $h$
  are morphisms $T \to \nabla U$ because they are realized by
  $\combK_{|S|,|\nabla U|} a$ respectively, where $a \in
  \compAtyp{|\nabla U|}$. Since $f$ is epi in $\AsmA$, we may cancel
  it and obtain $g = h$. This shows that $f$ is epi in~$\Set$.
\end{proof}

A \emph{mono-epi} is a morphism $f : S \to T$ which is both mono and
epi. In general such a morphism need not be an isomorphism. For
example, a continuous bijection between topological spaces need not be
a homeomorphism.

\begin{corollary}
  A realized map $f : \asm{S} \to \asm{T}$ is mono-epi if, and only
  if, it is is bijective.
\end{corollary}

\begin{proof}
  This follows directly from Proposition~\ref{prop:asm-mono-epi-when}
  and the fact that in $\Set$ an epi-mono is the same thing as a
  bijection.
\end{proof}

It is easy to provide an epi-mono which is not an isomorphism. For
example if $\asm{S}$ is a modest set with at least two different
elements, then $\id_S : S \to S$ is realized as a morphism $\id_S :
\asm{S} \to \nabla S$, hence it is an epi-mono in $\AsmA$. However,
every morphism $\nabla S \to \asm{S}$ is a constant map, therefore
$\asm{S}$ is not isomorphic to~$\nabla{S}$.

Recall that $f : \asm{S} \to \asm{T}$ is a \emph{regular mono} if
there are $g, h : \asm{T} \to \asm{U}$ such that~$f$ is their
equalizer. Regular monos are well behaved, and we can think of them as
subspace embeddings. Given an assembly $\asm{T}$ and a subset $T'
\subseteq T$, define the assembly $\asm{T}' = (T', |T|, {\rz_{T'}})$
as the restriction of $\asm{T}$, i.e., $\R{x} \rz_{T'} x$ if, and only
if $x \in T'$ and $\R{x} \rz_T x$. The subset inclusion $i : T' \to T$
is a morphism of assemblies because it is realized by
$\xpcalam{\annot{x}{|T|}}{x}$, and it is a regular mono because it is
the equalizer of the maps $g, h : T \to \nabla \two$, defined by
%
\begin{equation*}
  g(x) = 1
  \qquad\text{and}\qquad
  h(x) =
  \begin{cases}
    1 & \text{if $x \in T'$,}\\
    0 & \text{otherwise}.
  \end{cases}
\end{equation*}
%
Even more, every regular mono is of this form, up to isomorphism. To
see this, suppose $f : \asm{S} \to \asm{T}$ is a regular mono, thus an
equalizer of morphisms $g, h : \asm{T} \to \asm{U}$. If we simply
compute the equalizer of $g$ and $h$ again according to the
recipe~\eqref{eq:asm-equalizer} from
Section~\ref{sec:cartesian-structure}, we see that it is precisely the
restriction of $\asm{T}$ to the subset
% 
\begin{equation*}
  T' = \set{x \in T \such g(x) = h(x)}.
\end{equation*}
% 
Since both $i : \asm{T}' \to \asm{T}$ and $f : \asm{S} \to \asm{T}$
are equalizers of $g$ and $h$, they are isomorphic.

We repeat the story for \emph{regular epis}, which are those morphisms
that are coequalizers. They are the well behaved epis which can be
thought of as quotient maps. In fact, if $f : \asm{S} \to \asm{T}$ is
a regular epi, we say that $\asm{T}$ is a \emph{quotient} of~$\asm{S}$.

The match between regular epis and quotients is precise in $\AsmA$.
Note that that in the construction of coequalizers, as described
above, we may start with an arbitrary equivalence relation: given an
assembly $\asm{T}$ and an equivalence relation $\equiv$ on~$T$, define
the \emph{quotient assembly} $\asm{T}/{\equiv} = (T/{\equiv}, |T|,
\rz_{T/{\equiv}})$ whose realizability relation satisfies $\R{x}
\rz_{T/{\equiv}} [y]$ if, and only if, $\R{x} \rz_T x$ and $x \equiv
y$ for some $x \in T$. The quotient map $q : T \to T/{\equiv}$ is
realized by $\xpcalam{\annot{x}{|T|}}{x}$, and is a coequalizer of $g,
h : \asm{V} \to \asm{T}$ where\footnote{You should convince yourself
  that $\asm{V}$ is the \emph{kernel pair} of~$q$, i.e., the pullback
  of $q$ with itself.}
%
\begin{align*}
  \asm{V} &= (\set{(x,y) \in T \times T \such x \equiv y},
             |T| \times |T|, {\rz_V}),\\
  \combPair\;\R{x}\;\R{y} \rz_V (x,y) &\iff
  \R{x} \rz_T x \land \R{y} \rz_T y, \\
  g (x, y) &= x,\\
  h (x, y) &= y.
\end{align*}
%
Every regular epi is isomorphic to one of this form. To see this,
suppose $f : \asm{T} \to \asm{U}$ is a coequalizer of $g, h : \asm{U}
\to \asm{T}$. Let $\equiv$ be the least equivalence relation on~$T$
such that $g(x) = h(x)$ for all $x \in U$. Then $f$ is isomorphic to
$q : \asm{T} \to \asm{T}/{\equiv}$ because $g : \asm{T} \to
\asm{T}/{\equiv}$ is the coequalizer of $g$ and $h$ according to
Section~\ref{sec:cocartesian-structure}.

There is another characterization of regular epis which is used often.

\begin{proposition}
  in $\AsmA$ a morphism $f : \asm{T} \to \asm{U}$ is a regular epi
  if, and only if, there exists $\R{i} \in \compAtyp{|U| \to |T|}$
  such that, whenever $\R{y} \rz_U y$ then $\defined{\R{i}\;\R{y}}$
  and there is $x \in f^{-1}(y)$ such that $\R{i}\;\R{y} \rz_T x$.
\end{proposition}

\begin{proof}
  In Section~\ref{sec:multi-valued-functions} we will see that $\R{i}$
  is a realizer for a multi-valued right inverse of~$f$, while
  in Section~\ref{sec:logical-reg-epi} we will see that $\R{i}$ is a
  realizer for the internal statement
  $\xall{y}{\asm{S}}{\xsome{x}{\asm{T}}{f(x) = y}}$. For the moment we
  just concentrate on the proof.

  Suppose first that we have a regular epi $f : \asm{T} \to \asm{U}$
  which is a coequalizer of~$g, h : \asm{S} \to \asm{T}$. Let $q :
  \asm{T} \to \asm{T}/{\equiv}$ be the coequalizer of~$g$ and~$h$, as
  computed in Section~\ref{sec:cocartesian-structure}. Because~$f$ and
  $q$ both equalize $g$ and $h$, there exists a unique isomorphism $i
  : \asm{U} \to \asm{T}/{\equiv}$ such that $q = i \circ f$. Let
  $\R{i} \in \compAtyp{|U| \to |T|}$ be a realizer for~$i$. We claim
  that it has the required properties. If $\R{y} \rz_U y$ then
  $\defined{\R{i}\;\R{y}}$ and $\R{i}\;\R{y} \rz_{T/{\equiv}} i(y)$.
  Because $f$ is surjective there exists $x' \in T$ such that $f(x') =
  y$, from which we get $i(y) = i(f(x')) = q(x') = [x']$. Since
  $\R{i}\;\R{y} \rz_{T/{\equiv}} [x']$ there is $x \in T$ such that
  $\R{i}\;\R{y} \rz_T x$ and $x \equiv x'$. The element~$x$ is the one
  we are looking for because $x \equiv x'$ implies $f(x) = f(x') = y$.

  Conversely, suppose $\R{i}$ is as in the statement of the
  proposition, and let $\R{f}$ be a realizer for~$f$. To show that $f
  : \asm{T} \to \asm{U}$ is a coequalizer, define the equivalence
  relation $\equiv$ on~$T$ by
  %
  \begin{equation*}
    x \equiv y \iff f(x) = f(y).
  \end{equation*}
  %
  It suffices to show that~$f$ is isomorphic to $q : \asm{T} \to
  \asm{T}/{\equiv}$. In one direction we have the map $j : T/{\equiv}
  \to U$ defined by $j([x]) = f(x)$, realized by $\R{f}$. In the other
  direction we have $i : U \to T/{\equiv}$ defined by $i(f(x)) = [x]$,
  which is well defined because~$f$ is surjective. The map $i$ is
  realized by~$\R{i}$. It is obvious that $i$ and $j$ are inverses of
  each other and that $q = i \circ f$ holds.
\end{proof}

% When is a numbered set total

\subsection{Factorization of morphisms and regular structure}
\label{sec:factorization}

% Explicitly state regularity

\subsection{Cartesian closed structure}
\label{sec:ccc}

\subsection{Families of assemblies}
\label{sec:dependent-types}

% Uniform families

% Dependent products and sums

% State lccc

\subsection{Projective assemblies}
\label{sec:projective-assemblies}

% Presentation axiom

\begin{definition}
  An assembly is \emph{canonically projective} if each element has
  precisely one realizer.
\end{definition}

\noindent
In symbols, an assembly $S$ is canonically projective when, for all
$x, y \in S$, $r \in A_{|S|}$,
%
\begin{equation*}
  r \rz_S x \land r \rz_S y \implies x = y.
\end{equation*}
%
The canonically projective assemblies are, up to isomorphism,
precisely the projective objects of the category of assemblies.

A canonically projective modest set $(S, |S|, {\rz_S})$ is determined,
up to isomorphism, by the set of the total realizers~$\|S\|$. This
follows from Lemma~\ref{lemma:iso-assembly} and the fact that
projectivity ensures that $S$ and $\|S\|$ are in bijective
correspondence. This combined with the fact that every modest set is a
quotient of a canonically projective one leads to the following
definition.

Let $\subcat{A, \comp{A}}$ be the category whose objects are pairs
$(S, |S|)$ where $|S|$ is a type and $S \subseteq A_{|S|}$. A morphism
$f : (S, |S|) \to (T, |T|)$ is a map $f : S \to T$ which is tracked by
some $p \in \comp{A}_{|S| \to |T|}$, i.e., for all $p \in S$,
$\defined{p\;r}$ and $p\;r \in T$. This category is equivalent to the
full subcategory of $\Mod{A, \comp{A}}$ on the canonically projective
modest sets.


\section{The realizability interpretation of logic}
\label{sec:realizability-interpretation}

\section{Realizability toposes}
\label{sec:realizability-toposes}

\section{From constructive to computable mathematics}
\label{sec:constructive-math}



%%% Local Variables: 
%%% mode: latex
%%% TeX-master: "notes"
%%% End: 
