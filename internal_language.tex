\chapter{The Internal Language}
\label{chap:internal-language}

\section{Constructions of assemblies}
\label{sec:constructions}


\subsection{Projective assemblies}
\label{sec:projective-assemblies}

\begin{definition}
  An assembly is \emph{canonically projective} if each element has
  precisely one realizer.
\end{definition}

\noindent
In symbols, an assembly $S$ is canonically projective when, for all
$x, y \in S$, $r \in A_{|S|}$,
%
\begin{equation*}
  r \rz_S x \land r \rz_S y \implies x = y.
\end{equation*}
%
The canonically projective assemblies are, up to isomorphism,
precisely the projective objects of the category of assemblies.

A canonically projective modest set $(S, |S|, {\rz_S})$ is determined,
up to isomorphism, by the set of the total realizers~$\|S\|$. This
follows from Lemma~\ref{lemma:iso-assembly} and the fact that
projectivity ensures that $S$ and $\|S\|$ are in bijective
correspondence. This combined with the fact that every modest set is a
quotient of a canonically projective one leads to the following
definition.

Let $\subcat{A, \comp{A}}$ be the category whose objects are pairs
$(S, |S|)$ where $|S|$ is a type and $S \subseteq A_{|S|}$. A morphism
$f : (S, |S|) \to (T, |T|)$ is a map $f : S \to T$ which is tracked by
some $p \in \comp{A}_{|S| \to |T|}$, i.e., for all $p \in S$,
$\defined{p\,r}$ and $p\,r \in T$. This category is equivalent to the
full subcategory of $\Mod{A, \comp{A}}$ on the canonically projective
modest sets.


\section{The realizability interpretation of logic}
\label{sec:realizability-interpretation}

\section{Realizability toposes}
\label{sec:realizability-toposes}

\section{From constructive to computable mathematics}
\label{sec:constructive-math}



%%% Local Variables: 
%%% mode: latex
%%% TeX-master: "notes"
%%% End: 
