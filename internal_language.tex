\chapter{The Internal Language}
\label{chap:internal-language}

In Chapter~\ref{chap:realizability} we defined assemblies as our basic
structures of interest. To learn more about them we study which
oeprations on assemblies are supported by the category $\Asm{A,
  \comp{A}}$. It turns out that there is not only a rich collection of
constructions available, but also an interpretation of first-order
logic that assigns to a logical formula its computational meaning.

\section{Constructions of assemblies}
\label{sec:constructions}

In their everyday lives mathematicians use a limited set of
constructions on sets, such as cartesian products, disjoint sums,
subsets, quotient sets, powersets, and others. For all of these,
except powersets, there are analogous constructions of assemblies.
Therefore, if we need a computable version of a given mathematical
structure, we may just mimick its set-theoretic definition in the
category of assemblies. This way we do not invent or guess the
datatypes and the realizability relations, but rather simply
\emph{compute} them.

\subsection{Products and equalizers}
\label{sec:products-equalizers}



% terminal object

\subsection{Coproducts and quotients}
\label{sec:products-equalizers}

% initial object

\subsection{Epis and monos}
\label{sec:epis-monos}

% When is a numbered set total

\subsection{Factorization of morphisms}
\label{sec:factorization}

% Explicitly state regularity

\subsection{Cartesian closed structure}
\label{sec:ccc}

\subsection{Families of assemblies}
\label{sec:dependent-types}

% Uniform families

% Dependent products and sums

% State lccc

\subsection{Projective assemblies}
\label{sec:projective-assemblies}

% Presentation axiom

\begin{definition}
  An assembly is \emph{canonically projective} if each element has
  precisely one realizer.
\end{definition}

\noindent
In symbols, an assembly $S$ is canonically projective when, for all
$x, y \in S$, $r \in A_{|S|}$,
%
\begin{equation*}
  r \rz_S x \land r \rz_S y \implies x = y.
\end{equation*}
%
The canonically projective assemblies are, up to isomorphism,
precisely the projective objects of the category of assemblies.

A canonically projective modest set $(S, |S|, {\rz_S})$ is determined,
up to isomorphism, by the set of the total realizers~$\|S\|$. This
follows from Lemma~\ref{lemma:iso-assembly} and the fact that
projectivity ensures that $S$ and $\|S\|$ are in bijective
correspondence. This combined with the fact that every modest set is a
quotient of a canonically projective one leads to the following
definition.

Let $\subcat{A, \comp{A}}$ be the category whose objects are pairs
$(S, |S|)$ where $|S|$ is a type and $S \subseteq A_{|S|}$. A morphism
$f : (S, |S|) \to (T, |T|)$ is a map $f : S \to T$ which is tracked by
some $p \in \comp{A}_{|S| \to |T|}$, i.e., for all $p \in S$,
$\defined{p\,r}$ and $p\,r \in T$. This category is equivalent to the
full subcategory of $\Mod{A, \comp{A}}$ on the canonically projective
modest sets.


\section{The realizability interpretation of logic}
\label{sec:realizability-interpretation}

\section{Realizability toposes}
\label{sec:realizability-toposes}

\section{From constructive to computable mathematics}
\label{sec:constructive-math}



%%% Local Variables: 
%%% mode: latex
%%% TeX-master: "notes"
%%% End: 
