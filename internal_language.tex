\chapter{The Internal Language}
\label{chap:internal-language}

In Chapter~\ref{chap:realizability} we defined assemblies as our basic
structures of interest. To learn more about them we study which
oeprations on assemblies are supported by the category $\Asm{A,
  \comp{A}}$. It turns out that there is not only a rich collection of
constructions available, but also an interpretation of first-order
logic that assigns to a logical formula its computational meaning.

\section{Constructions of assemblies}
\label{sec:constructions}

In their everyday lives mathematicians use a limited set of
constructions on sets: products, disjoint sums, subsets, quotient
sets, function spaces, inductive and coinductive definitions, and
powersets. For all of these, except powersets, there are analogous
constructions of assemblies. Therefore, if we need a computable
version of a given mathematical structure, we may just mimick its
set-theoretic definition in the category of assemblies. This way we do
not invent or guess the datatypes and the realizability relations, but
simply \emph{compute} them.

\subsection{The cartesian structure}
\label{sec:cartesian-structure}

A construction which makes a new set, space, or an algebraic structure
out of old ones is usually characterized by a \emph{universal
  property} which determines it up to isomorphism. We start slowly
with an easy one, the binary product.

Recall the definition of a \emph{(binary) product} in a category: the
product of objects $S$ and $T$ is an object $P$ with morphisms $p_1 :
P \to S$ and $p_2 : P \to T$, satisfying the following universal
property: for all morphisms $f : U \to S$ and $g : U \to T$ there is a
unique morphism $h : U \to P$ making the following diagram commute:
%
\begin{equation*}
  \xymatrix@+1em{
    &
    {U}
    \ar[ld]_{f}
    \ar[d]^h
    \ar[rd]^{g}
    &
    \\
    {S}
    &
    {P}
    \ar[l]^{p_1}
    \ar[r]_{p_2}
    &
    {T}
  }
\end{equation*}
%
This determines $(P, p_1, p_2)$ uniquely up to a unique isomorphism.
For suppose we had another product $(Q, q_1, q_2)$. By the universal
property of $P$ there is a map $h : P \to Q$ such that $p_1 \circ h =
q_1$ and $p_2 \circ h = q_2$. Similarly, there is $k : Q \to P$ such
that $q_1 \circ k = p_1$ and $q_2 \circ k = p_2$. Now $h \circ k$
satisfies
%
\begin{equation*}
  p_1 \circ h \circ k = q_1 \circ k = p_1
  \qquad\text{and}\qquad
  p_2 \circ h \circ k = q_2 \circ k = p_2
\end{equation*}
%
Since $\id_P$ also satisfies $p_1 \circ \id_P = p_1$ and $p_2 \circ
\id_P = p_2$, it follows by the uniqueness condition of the universal
property that $h \circ k = \id_P$. A similar argument shows that $k
\circ h = \id_Q$, hence $P$ and $Q$ are isomorphic.

A category has binary products if every pair of objects has a binary
product. In most cases we can actually provide an operation $\times$
which maps a pair of objects $S$, $T$ to a specifically given product
$S \times T$ with corresponding projections. The unique map $h$
determined by $f$ and $g$ is denoted by $\pair{f,g}$.

In the category~$\Set$ the product is just the usual cartesian product
of sets. In assemblies we need to worry about the underlying types and
realizability relations. Let us verify that the product of assemblies
$\asm{S}$ and $\asm{T}$ is the assembly
%
\begin{equation*}
  \asm{S} \times \asm{T} =
  (S \times T, |S| \times |T|, {\rz_{S \times T}})
\end{equation*}
%
with the realizability relation
%
\begin{equation*}
  \combPair\,q\,r \rz_{S \times T} (x,y)
  \iff
  q \rz_S x
  \land
  r \rz_T y.
\end{equation*}
%
The projection maps $p_1 : S \times T \to S$ and $p_2 : S \times T \to
T$ are realized by $\combFst$ and $\combSnd$, respectively.

To see that the product of assemblies has the universal property,
suppose $f : \asm{U} \to \asm{S}$ and $g : \asm{U} \to \asm{T}$ are
realized by $\R{f} \in \comp{A}_{|U| \to |S|}$ and $\R{g} \in
\comp{A}_{|U| \to |T|}$, respectively. There is a unique map $h : U
\to S \times T$ for which $f = p_1 \circ h$ and $g = p_2 \circ g$,
namely $h(u) = \pair{f,g}(u) = (f(u), g(u))$. We only need a realizer
for~$h$, and $\R{h} =
\pcalam{\annot{u}{|U|}}{\combPair\,(\R{f}\,u)\,(\R{g}\,u)}$ does the
job:
%
\begin{multline*}
  \R{u} \rz_U u
  \implies
  \R{f}\,\R{u} \rz_S f(u)
  \land
  \R{g}\,\R{u} \rz_T g(u)
  \implies {} \\
  \combPair\, (\R{f}\,\R{u})\, (\R{g}\,\R{u}) \rz_{S \times T} (f(u), g(u))
  \iff
  \R{h}\,\R{u} \rz_{S \times T} h(u).
\end{multline*}

Once we have binary products we may form $n$-ary products $S_1 \times
\cdots \times S_n$ for $n \geq 1$ by repeatedly forming binary
products. The case $n = 0$ corresponds to the \emph{terminal object},
which is an object $\one$ such that for every object~$S$ there is
exactly one morphism $S \to \one$. In the category of sets the
terminal object is (any) singleton set, say $\one = \set{\star}$. Then
$\nabla \one$ is the terminal assembly, since for any assembly
$\asm{S}$ the only map $S \to \one$ is realized. We denote the
terminal assembly as $\one$.

We may also ask whether $\AsmA$ has infinite products. The answer
depends on the underlying TPCAs $(\AA, \compAA)$. We state without
proof that $\Asm{\Scott, \comp{\Scott}}$ and $\Asm{\Baire,
  \comp{\Baire}}$ have countable products, whereas $\Asm{\NN}$ does
not.

Products are a special case of categorical limits. Two other common
kinds of limits are equalizers and pullbacks. An \emph{equalizer} of a
pair of morphisms $f, g : S \to T$ is an object $E$ with a morphism $e
: E \to S$ such that $e$ equalizes $f$ and $g$, which means that $f
\circ e = g \circ e$, and the following universal property is
satisfied: if $k : K \to S$ also equalizes $f$ and $g$ then there
exists a unique morphism $i : K \to E$ such that $k = e \circ i$:
%
\begin{equation*}
  \xymatrix@+1em{
    {E}
    \ar[r]^{e}
    &
    {S}
    \ar@<+0.25em>[r]^{f}
    \ar@<-0.25em>[r]_{g}
    &
    {T}
    \\
    {K}
    \ar[u]^{i}
    \ar[ru]_{k}
    & & 
  }
\end{equation*}
%
Think of $E$ as the solution-set of equation $f(x) =
g(x)$.\footnote{If we have a system of equations $f_i(x) = g_i(x)$, $i
  = 1, \ldots, n$, then we may express them as a single vector
  equation $f(x) = g(x)$ where $f(x) = (f_1(x), \ldots, f_n(x))$ and
  $g(x) = (g_1(x), \ldots, g_n(x))$. Equalizers are thus an abstract
  formulation of the notion ``solution of a system of equations''.}
Indeed, in the category of sets the equalizer of functions $f, g : S
\to T$ is the subset $E = \set{x \in S \such f(x) = g(x)}$ and $e : E
\to S$ is the subset inclusion. In the category of assemblies we
need to augment this with realizers. If $f, g : \asm{S} \to \asm{T}$
are realized by $f, g \in \compAtyp{|S| \to |T|}$, respectively, then
their equalizer is
%
\begin{equation*}
  \asm{E} =
  (\set{x \in S \such f(x) = g(x)},
   |S|,
   {\rz_E})
\end{equation*}
%
where $\R{x} \rz_E x$ if, and only if, $\R{x} \rz_S x$. The map $e : E
\to S$ is the subset inclusion, $e(x) = x$. It is realized by
$\xpcalam{\annot{x}{|S|}}{x}$. Clearly, $e$ equalizes~$f$ and~$g$. We
leave the verification of the universal property as exercise.

A \emph{pullback}, sometimes called \emph{fibered products}, is a
combination of product and equalizer. Given morphisms $f : S \to U$
and $g : T \to U$, the pullback of $f$ and $g$ is an object $P$ with
morphisms $p_1 : P \to S$ and $p_2 : P \to T$ such that $f \circ p_1 =
g \circ p_2$. Furthermore, if $q_1 : Q \to S$ and $q_2 : Q \to T$ are
such that $f \circ q_1 = g \circ q_2$ then there is a unique $i : Q
\to P$ which makes the following diagram commute:
%
\begin{equation*}
  \xymatrix@+0.5em{
    {Q}
    \ar[dr]^{i}
    \ar@/_1em/[ddr]_{q_2}
    \ar@/^1em/[rrd]^{q_1}
    &
    &
    \\
    &
    {P} \pbcorner
    \ar[r]^{p_1}
    \ar[d]_{p_2}
    &
    {S}
    \ar[d]^{f}
    \\
    &
    {T}
    \ar[r]_{g}
    &
    {U}
  }
\end{equation*}
%
The fact that $P$ is a pullback is traditionally marked in a diagram
with the ``corner'' symbol. In the category of assemblies the pullback
of $f : \asm{S} \to \asm{U}$ and $g : \asm{T} \to \asm{U}$ is the
assembly
%
\begin{equation*}
  \asm{P} = (\set{(x,y) \in S \times T \such f(x) = g(y)}, |S| \times
  |T|, {\rz_P})
\end{equation*}
%
where $\combPair\,\R{x}\,\R{y} \rz_P (x,y)$ if, and only if, $\R{x}
\rz_S x$ and $\R{y} \rz_T y$.

Finite products, the terminal object, equalizers, and pullbacks are
special cases of \emph{finite limits}. A category which has all finite
limits is called \emph{cartesian} or \emph{finitely
  complete}.\footnote{We do not like much the still older terminology
  \emph{left exact} or just \emph{lex}.}

\begin{proposition}
  The categories $\AsmA$ and $\Mod{\AA, \compAA}$ are cartesian.
\end{proposition}

\begin{proof}
  It is well known that every finite limit may be constructed as a
  combination of a finite product and an equalizer, hence $\AsmA$ is
  cartesian. It is easy to verify that finite products and equalizers
  of modest assemblies are again modest, therefore $\Mod{\AA,
    \compAA}$ is cartesian.
\end{proof}


%%%%%%%%%%%%%%%%%%%%%%%%%%%%%%%%%%%%%%%%%%%%%%%%%%
\subsection{Cocartesian structure}
\label{sec:cocartesian-structure}


The dual notion to finite limit is finite \emph{colimit}. In
particular, products, terminal object, equalizers, and pullbacks
correspond respectively to coproducts, initial object, coequalizers,
and pushouts. Let us show that $\AsmA$ has these. We refer to a
standard text on category theory for background and definitions of
these concepts~\cite{MacLane,Awodey}.

A \emph{(binary) coproduct} or \emph{disjoint sum} of assemblies
$\asm{S}$ and $\asm{T}$ is the assembly
%
\begin{equation*}
  \asm{S} + \asm{T} = (S + T, |S| + |T|, \rz_{S+T})
\end{equation*}



\subsection{Epis and monos}
\label{sec:epis-monos}

% When is a numbered set total

\subsection{Factorization of morphisms}
\label{sec:factorization}

% Explicitly state regularity

\subsection{Cartesian closed structure}
\label{sec:ccc}

\subsection{Families of assemblies}
\label{sec:dependent-types}

% Uniform families

% Dependent products and sums

% State lccc

\subsection{Projective assemblies}
\label{sec:projective-assemblies}

% Presentation axiom

\begin{definition}
  An assembly is \emph{canonically projective} if each element has
  precisely one realizer.
\end{definition}

\noindent
In symbols, an assembly $S$ is canonically projective when, for all
$x, y \in S$, $r \in A_{|S|}$,
%
\begin{equation*}
  r \rz_S x \land r \rz_S y \implies x = y.
\end{equation*}
%
The canonically projective assemblies are, up to isomorphism,
precisely the projective objects of the category of assemblies.

A canonically projective modest set $(S, |S|, {\rz_S})$ is determined,
up to isomorphism, by the set of the total realizers~$\|S\|$. This
follows from Lemma~\ref{lemma:iso-assembly} and the fact that
projectivity ensures that $S$ and $\|S\|$ are in bijective
correspondence. This combined with the fact that every modest set is a
quotient of a canonically projective one leads to the following
definition.

Let $\subcat{A, \comp{A}}$ be the category whose objects are pairs
$(S, |S|)$ where $|S|$ is a type and $S \subseteq A_{|S|}$. A morphism
$f : (S, |S|) \to (T, |T|)$ is a map $f : S \to T$ which is tracked by
some $p \in \comp{A}_{|S| \to |T|}$, i.e., for all $p \in S$,
$\defined{p\,r}$ and $p\,r \in T$. This category is equivalent to the
full subcategory of $\Mod{A, \comp{A}}$ on the canonically projective
modest sets.


\section{The realizability interpretation of logic}
\label{sec:realizability-interpretation}

\section{Realizability toposes}
\label{sec:realizability-toposes}

\section{From constructive to computable mathematics}
\label{sec:constructive-math}



%%% Local Variables: 
%%% mode: latex
%%% TeX-master: "notes"
%%% End: 
