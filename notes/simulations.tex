\section{Simulations}
\label{sec:simulations}%

If you open a book on computability theory, chances are that you will find a statement saying that ``models of computation are equivalent''. The claim refers to a collection of specific models of computation, such as variations of Turing machines, $\lambda$-calculi, and recursive functions. The book supports the claim by describing ad hoc simulations between such models, with varying degrees of detail, after which it hurries on to core topics of computability theory. An opportunity is missed to ask about a general notion of simulation, and a study of its structural properties.


We seize the opportunity and studu such a notion. An excellent one was provided by John Longley~\sidecite{Longley:94}, namely \emph{applicative morphism} of pcas. We extend his definition to account for pcas with sub-pcas, and dare rename the morphisms to \emph{simulations}. For simplicity we do not dwelve into the typed version of simulations, which can also be set up~\cite{longley99:_match}.

\begin{definition}
  \label{def:simulation}%
  %
  \indexsee{morphism!applicative}{applicative, morphism}%
  \indexdef{applicative!morphism}%
  \indexdef{simulation}%
  %
  A \defemph{(pca) simulation}, originally called an \defemph{applicative morphism}~\cite{Longley:94},
  $\rho: \EE \pcato \FF$ between pcas~$\EE$ and~$\FF$ is a total relation $\rho
  \subseteq \EE \times \FF$ for which there exists a \defemph{realizer} $r \in \FF$
  %
  \indexdef{realizer!for applicative morphism}%
  \indexdef{applicative!morphism!realizer for}%
  \indexdef{realizer!for simulation}%
  \indexdef{simulation!realizer for}%
  %
  such that, for all $u, v \in \EE$ and $x, y \in \FF$,
  %
  \begin{itemize}
  \item if $\rho(u, x)$ then $\defined{r \, x}$ and
  \item if $\rho(u, x)$, $\rho(u, y)$ and $\defined{u \, v}$ then
    $\defined{r \, x \, y}$ and $\rho(u \, v, r \, x \, y)$.
  \end{itemize}
  %
  We write $\rho[u] = \set{x \in \FF \such \rho(u, x)}$.

  A \defemph{(sub-pca) simulation} $\rho: (\EE, \subEE) \pcato (\FF, \subFF)$ between pcas with sub-pcas is a simulation
  $\rho : \EE \pcato \FF$ which has a realizer $r \in \subFF$, and which restricts to a simulation
  $\rho : \subEE \pcato \subFF$, with the same realizer~$r$.
\end{definition}

We defined a simulation $\rho : \EE \pcato \FF$ to be a total relation rather than a function because in~$\FF$ there might be many simulations of an element of $\EE$, without any one being preferred. The notation $\rho[u]$ suggests that~$\rho$ is construed as a multi-valued map rather than a relation.

The realizer $r \in \FF$ of a simulation~$\rho$ is an $\FF$-implementation of the application in~$\EE$.

One might expect that a simulation ought to be a map $f : \EE \to \FF$ such that $f(\combK_\EE) = \combK_\FF$, $f(\combS_\FF) = \combS_\FF$, and $f (x \cdot_\EE y) \kleq f x \cdot_\FF f y$. This is how an algebraist would define a morphism, but we are interested in the computational aspects of pcas, not the algebraic ones.

\index{simulation!composition of}%
\index{composition!simulations}%
%
Simulations can be \defemph{composed} as relations.
If $\rho: (\EE, \subEE) \pcato (\FF, \subFF)$ and $\sigma: (\FF,
\subFF) \pcato (\GG, \subGG)$ then $\sigma \circ \rho: (\EE, \subEE)
\pcato (\GG, \subGG)$ is defined, for $x \in \EE$ and $z \in \GG$, by
%
\begin{equation*}
  z \in (\sigma \circ \rho)[x]
  \iff
  \some{y \in \FF} y \in \rho[x] \land z \in \sigma[y].
\end{equation*}

\begin{exercise}
  Show that $\sigma \circ \rho$ is realized if $\rho$ and $\sigma$ are.
\end{exercise}

The identity simulation $\id[(\EE, \subEE)]: (\EE,
\subEE) \pcato (\EE, \subEE)$ is the identity relation on~$\EE$. It is realized by $\pcalam{x\,y} x \, y$.

Pcas with sub-pcas and simulations between them therefore form a category. We equip it with a \defemph{preorder enrichment}\sidenote{A category $\mathcal{C}$ is preorder enriched when hom-sets $\mathcal{C}(X,Y)$ are equipped with preorders (reflexive and transitive relations) under which composition is monotone.} $\preceq$ as follows.
%
\indexdef{category!of simulations}%
\index{translation, of simulations}%
\index{equivalence!of pcas}%
%
Given $\rho, \sigma : (\EE, \subEE) \pcato (\FF, \subFF)$, define $\rho \preceq \sigma$ to hold when
there exists a \defemph{translation} $t \in \subFF$ such that, for all $x \in \EE$ and $y \in \rho[x]$, $\defined{t \, x}$ and $t \, y \in \sigma[x]$.

We write $\rho \sim \sigma$ when $\rho \preceq \sigma$ and $\sigma \preceq \rho$.

\begin{exercise}
  Given $\rho, \rho' : (\EE, \subEE) \pcato (\FF, \subFF)$ and $\sigma, \sigma' : (\FF, \subFF) \pcato (\GG, \subGG)$, show that if $\rho \preceq \rho'$ and $\sigma \preceq \sigma'$ then $\sigma \circ \rho \preceq \sigma' \circ \rho'$.
\end{exercise}

The preorder enrichment induces the notions of equivalence and adjunction of simulations.

\goodbreak

\begin{definition}%
  %
  \index{simulation!adjoint pair of}%
  \index{adjunction!simulation}%
  %
  Consider simulations
  %
  \begin{align*}
    \delta &: (\EE, \subEE) \pcato (\FF, \subFF),
    &
    \gamma &: (\FF, \subFF) \pcato (\EE, \subEE).
  \end{align*}
  %
  They form an \defemph{equivalence} when $\gamma \circ \delta \sim \one_\EE$ and $\delta \circ \gamma \sim \one_\FF$.

  They form an \defemph{adjunction}, written $\gamma \dashv \delta$, when
  %
  $\one_{\FF} \preceq \delta \circ \gamma$ and $\gamma \circ
  \delta \preceq \one_{\EE}$. We say that~$\gamma$ is \defemph{left
    adjoint} to~$\delta$, or that~$\delta$ is \defemph{right adjoint}
  to~$\gamma$.

  \indexdef{simulation!inclusion}%
  \indexdef{simulation!retraction}%
  \indexsee{inclusion!simulation}{simulation, inclusion}%
  \indexsee{retraction!simulation}{simulation, retraction}%
  %
  Such an adjoint pair is an \defemph{inclusion} when $\gamma \circ \delta \sim \id[\EE]$, and a
  \defemph{retraction} when $\delta \circ \gamma \sim \one_\FF$.
\end{definition}

\subsection{Properties of simulations}
\label{sec:prop-simul}

Nothing prevents a simulation from being trivial. In fact, there always is the constant simulation $\tau : \EE \pcato \FF$, defined by $\tau[x] = \set{\combK_{\FF}}$ and realized by $\pcalam{x\,y} \combK$. We should identify further useful properties of simulations.

Discreteness prevents a simulation from conflating simulated elements.

\begin{definition}
  A simulation $\rho: (\EE, \subEE) \pcato (\FF, \subFF)$ is
  %
  \indexdef{simulation!discrete}%
  \indexdef{discrete!simulation}%
  %
  \defemph{discrete} when, for all $x, y \in \EE$ if $\rho[x] \cap \rho[y]$ is in inhabited then $x = y$.
\end{definition}

The next property is single-valuedness, up to equivalence.

\begin{definition}
  A simulation $\rho: (\EE, \subEE) \pcato (\FF, \subFF)$ is
  %
  \indexdef{simulation!projective}%
  \indexdef{projective!simulation}%
  %
  \defemph{projective} when there is a single-valued simulation (a function) $\rho'$ such that $\rho' \sim \rho$.
\end{definition}

\begin{exercise}
  Prove that a simulation $\rho : \EE \pcato \FF$ is projective if, and only if,
  there is $t \in \subFF$ such that, for all $x \in \EE$ and $y, z \in \FF$:
  %
  \begin{itemize}
  \item if $y \in \rho[x]$ then $\defined{t\,y}$ and $t\,y \in \rho[x]$,
  \item if $y \in \rho[x]$ and $z \in \rho[x]$ then $t \, y = t \, z$.
  \end{itemize}
  %
  Thus a simulation is projective if each element of $\EE$ has a canonically chosen simulation in~$\FF$.
\end{exercise}


For every simulation $\rho : \EE \pcato \FF$ it is the case that the Boolean values of~$\FF$ can be converted to the simulated Boolean values. Indeed, take any $a \in \rho[\combTrue_\EE]$ and $b \in \rho[\combFalse_\EE]$ and define $e \in \FF'$ to be
$e = \pcalam{x} \combIf_\FF \, x \, a \, b$, so that $e \, \combTrue_\FF \in \rho[\combTrue_\EE]$ and $e \, \combFalse_\FF \in \rho[\combFalse_\EE]$. The converse translation does not come for free.

\begin{definition}
  A simulation $\rho: (\EE, \subEE) \pcato (\FF, \subFF)$ is
  %
  \indexdef{simulation!decidable}%
  \indexdef{decidable!simulation}%
  %
  \defemph{decidable} when there is $d \in \subFF$,
  called the
  %
  \indexdef{decider}%
  %
  \defemph{decider} for~$\rho$, such that, for all $x \in \FF$,
  %
  \begin{align*}
    x \in \rho[\combTrue_\EE] &\lthen d \, x = \combTrue_\FF,
    \\
    x \in \rho[\combFalse_\EE] &\lthen d \, x = \combFalse_\FF.
  \end{align*}
\end{definition}

\begin{exercise}
  \label{exc:simulation-numerals}%
  Say that a simulation $\rho: (\EE, \subEE) \pcato (\FF, \subFF)$ \defemph{preserves numerals}
  when there is $c \in \subFF$ such that, for all $n \in \NN$ and $x \in \FF$,
  %
  \begin{equation*}
    x \in \rho[\overline{n}_\EE] \implies c \, x = \overline{n}_\FF.
  \end{equation*}
  %
  Prove that a simulation is decidable if, and only if, it preserves numerals.
\end{exercise}


To show how the above definitions can be put to work, we recall several basic results of John Longley's.

\begin{theorem}
  \label{th:simulation-properties2}%
  For $\delta: (\EE, \subEE) \pcato (\FF, \subFF)$ and
  $\gamma: (\FF, \subFF) \pcato (\EE, \subEE)$:
  %
  \begin{enumerate}
  \item
    If $\gamma \circ \delta \preceq \id[\EE]$ then $\delta$ is discrete 
    and $\gamma$ is decidable.
  \item
    If $\gamma \dashv \delta$ then $\gamma$ is projective.
  \end{enumerate}
\end{theorem}

\begin{proof}
  See \cite[Theorem 2.5.3]{Longley:94}.
\end{proof}

\begin{corollary}
  \label{th:simulation-properties}%
  If $\gamma \dashv \delta$ is a retraction then both~$\delta$
  and~$\gamma$ are discrete and decidable, and~$\gamma$ is projective.
\end{corollary}

\begin{proof}
  Immediate. This is~\cite[Corollary 2.5.4]{Longley:94}.
\end{proof}

\begin{corollary}
  \label{th:simulation-properties-more}%
  If $(\EE, \subEE)$ and $(\FF, \subFF)$ are equivalent pcas, then the 
  there exist an equivalence
                                %
  \begin{align*}
    \delta &: (\EE, \subEE) \pcato (\FF, \subFF),
    &
    \gamma &: (\FF, \subFF) \pcato (\EE, \subEE),
  \end{align*}
                                %
  such that $\gamma$ and $\delta$ are single-valued.
\end{corollary}

\begin{proof}
  Both $\delta$ and $\gamma$ are projective by
  \cref{th:simulation-properties2}.
\end{proof}

\subsection{Examples of simulations}
\label{sec:examples-simulations}

Armed with a rigorous and general notion of simulation, we can revisit the classic examples of simulations.

\subsubsection{Initiality of $\klone$}
\label{ex:intiality-K1}%

Turing machines are distinguished by a universal property.

\begin{theorem}
  Up to equivalence, the first Kleene algebra~$\klone$ is initial in the category of pcas and decidable simulations.
\end{theorem}

\begin{proof}
  We sketch the proof from \cite[Theorem 2.4.18]{Longley:94}. Given any pca~$\AA$, define $\kappa : \klone \pcato \AA$ by $\kappa[n] = \set{\numeral{n}_{\AA}}$. Because every partial computable function $\NN \times \NN \parto \NN$ can be represented in~$\AA$, there is $r \in \AA$ such that, for all $k, m, n \in \NN$,
  %
  \begin{equation*}
    r \, \numeral{k} \, \numeral{m} = \numeral{n} \iff \pr{k}{m} = n.
  \end{equation*}
  %
  Such an element~$r$ realizes~$\kappa$. Furthermore, $\kappa$ is decidable because it maps numbers to numerals.

  Suppose $\mu : \klone \pcato \AA$ is another decidable simulation. Because $\mu$ preserve numerals there exists $f \in \AA$ such that if $a \in \mu[n]$ then $f \, a = \numeral{n} \in \kappa[n]$, therefore $\mu \preceq \kappa$. The relation $\kappa \preceq \mu$ holds by the exercise below, therefore $\kappa \sim \mu$.
\end{proof}

\begin{exercise}
  Verify that for any simulation $\rho : (\EE, \subEE) \pcato (\FF, \subFF)$ there is $q \in \subFF$ such that $q \, \overline{n}_{\FF} \in \rho[\overline{n}_{\EE}]$ for all $n \in \NN$.
\end{exercise}

\subsubsection{Simulations and Turing degrees}
\label{ex:pcamorphism-turing-degrees}%

Recall that a \defemph{Turing reduction} of $A \subseteq \NN$ to $B \subseteq \NN$, written $A \leq_T B$ is a $B$-oracle Turing machine~$M$ which computes the characteristic function of~$A$.

\begin{theorem}
  Suppose $A, B \subseteq \NN$. Then $A \leq_T B$ if, and only if, there is a decidable simulation $\klone^A \pcato \klone^B$.
\end{theorem}

\begin{proof}
  This is \cite[Proposition 3.1.6]{Longley:94}.
\end{proof}

\subsubsection{A retraction between $\Lambda$ and $\klone$}
\label{ex:pcamorphism-K1-lambda}%

We construct a retraction from the pca $\Lambda$ of the closed terms of the untyped $\lambda$-calculus, and first Kleene algebra~$\klone$.

Define $\delta : \klone \to \Lambda$ to be the map $\delta(n) = \numeral{n}$ which encodes numbers as Curry numerals.
%
It is a simulation because every partial computable map is $\lambda$-definable, and therefore so is Kleene application.

In the opposite direction, let $\gamma \subseteq \Lambda \times \klone$ be the total relation (remember that $\Lambda$ is the set of closed terms quotiented by $\beta$-reduction),
%
\begin{equation*}
  \gamma[t] = \set{ \code{t'} \such t' \in \Lambda \land t =_\beta t' }.
\end{equation*}
%
That is, the equivalence class of a closed term $t \in \Lambda$ is the set of all codes of terms equal to~$t$.
This is a simulation because there is a computable map~$f : \NN \times \NN \parto \NN$ satisfying $f(\code{t}, \code{u}) = \code{t \, u}$ for all $t, u \in \Lambda$.

It is not hard to see that $\gamma \circ \delta \sim \id[\klone]$. The relation $\id[\klone] \preceq \gamma \circ \delta$ holds because from $n \in \NN$ we can compute the code $\code{\numeral{n}}$. For the opposite relation $\gamma \circ \delta \preceq \id[\klone]$, we note that given any closed $\lambda$-term $t$ such that $t =_\beta \overline{n}$ for some $n \in \NN$, we can compute $n$ by computing the normal form of~$t$.

Verifying $\delta \circ \gamma \preceq \id[\Lambda]$ is a simple matter of programming in the untyped $\lambda$-calculus a term $e$ such that $e \, \overline{\code{t}} =_\beta t$ for any closed term~$t$.

\begin{exercise}
  Why is it \emph{not} the case that $\id[\Lambda] \preceq \delta \circ \gamma$?
\end{exercise}


\subsubsection{A retraction between $\comp{\Scott}$ and $\klone$}
\label{ex:pcamorphism_K1_RE}%

\index{retraction!between P and N@{between~$\comp{\Scott}$ and~$\klone$}}%

Let us look at a retraction $(\delta \dashv \gamma): \comp{\Scott} \pcato \klone$ between computable graph model and the first Kleene algebra~\cite[Proposition 3.3.7]{Longley:94}.
%
The inclusion $\delta: \klone \to \comp{\Scott}$ is defined by
% 
\begin{equation*}
  \delta(n) = \set{n},
\end{equation*}
% 
and the retraction $\gamma: \comp{\Scott} \pcato \klone$ by
% 
\begin{equation*}
  \gamma[A] = \set{n \in \NN \such \im{\xpr_n} = A}.
\end{equation*}
%
That is, a c.e.~set $A$ is simulated by the code of any partial computable map whose image is~$A$.

\begin{exercise}
  Verify that $\delta$ and $\gamma$ are simulations.
\end{exercise}

The relation $\gamma \circ \delta \preceq \id[\klone]$ holds because there is a partial computable map~$f : \NN \parto \NN$ which takes as input $k \in \NN$ such that $\im{\xpr_k} = \set{n}$ and outputs $n$. For example, $f$ can compute in parallel $\pr{k}{0}, \pr{k}{1}, \pr{k}{2}, \ldots$, and as soon as one of them terminates and outputs a number, $f$ terminates with the same number as its output.

The relation $\id[\klone] \preceq \gamma \circ \delta$ holds because from $n \in \NN$ we can compute $k \in \NN$ such that $\xpr_k$ is the constantly~$n$ map.

To establish $\delta \circ \gamma \preceq \id[\comp{\Scott}]$, note that $(\delta \circ \gamma)(A)$ is the index set of~$A$, i.e., the set of codes of those partial computable maps whose image is~$A$. The set
%
\begin{equation*}
  S = \set{ \code{\pair{\code{\set{m}}, n}} \such \some{k \in \NN} \pr{m}{k} = n }
\end{equation*}
%
is computably enumerable. The computable enumeration operator $\Lambda(S) : \comp{\Scott} \to \comp{\Scott}$, where $\Lambda$ is as in \eqref{eq:scott-section-retraction} is then a translation from $\delta \circ \gamma$ to $\id[\comp{\Scott}]$.

\begin{exercise}
  Why is it \emph{not} the case that $\id[\comp{\Scott}] \preceq \delta \circ \gamma$?
\end{exercise}



%%%%%%%%%%%%%%%%%%%%%%%%%%%%%%%%%%%%%%%%%%%%%%%%%%


\subsubsection{A retraction from $(\Scott, \comp{\Scott})$ to
  $(\Baire, \comp{\Baire})$}
\label{sec:retraction-PP-BB}%

\index{retraction!between P and B@{between~$(\Scott, \comp{\Scott})$ and~$(\Baire, \comp{\Baire})$}}%

Lietz~\cite{Lietz:99} compared realizability models over~$\Scott$ and
over~$\Baire$ and observed that there is an applicative retraction
$(\iota \dashv \delta): PP \pcato BB$. In this subsection we
describe it explicitly and show that it is in fact a retraction from
$(\Scott, \comp{\Scott})$ to $(\Baire, \comp{\Baire})$,
                                %
\begin{equation*}
  (\iota \dashv \delta): (\Scott, \comp{\Scott}) \pcato (\Baire, \comp{\Baire}).
\end{equation*}
                                %
Given a finite sequence of natural numbers $a = \seq{a_0, \ldots,
  a_{k-1}}$, let $\code{a}$ be the encoding of~$a$ as a natural
number, as defined in Section~\ref{sec:second_kleene}. Define the
embedding $\iota: \Baire \to \Scott$ by
                                %
\begin{equation*}
  \iota \alpha =
  \set{ \code{a} \such
    a \in \NN^{*} \land a  \sqsubseteq \alpha 
    }.
\end{equation*}
                                %
Observe that if $\alpha \in \comp{\Baire}$ then $\iota \alpha \in \comp{\Scott}$.
Let $\Baire' = \im{\iota}$ and define $p: \Baire' \times \Baire' \to \Scott$
by
                                %
\begin{equation*}
  p \pair{\iota \alpha, \iota \beta} =
  \begin{cases}
    \iota (\alpha \mid \beta) 
    & \text{if $\alpha \mid \beta$ defined} \,\\
    \emptyset
    & \text{otherwise}.
  \end{cases}
\end{equation*}
                                %
The map $p$ is continuous and it can be extended to an
r.e.~enumeration operator $p: \Scott \times \Scott \to \Scott$. Thus, $p$
realizes~$\iota$, which is therefore an applicative morphism.

Let $\delta: \Scott \pcato \Baire$ be the applicative morphism defined,
for  $x \in \Scott$, $\alpha \in \Baire$, by
                                %
\begin{equation*}
  \delta(x, \alpha)
  \iff
  x = \set{n \in \NN \such \some{k \in \NN} \alpha k = n + 1}.
\end{equation*}
                                %
In words, $\alpha$ is a $\delta$-implementation of~$x$ when it
enumerates~$x$. We added~$1$ to~$n$ in the above definition so that
the empty set is enumerated as well. Clearly, if $\alpha \in \comp{\Baire}$
then~$x \in \comp{\Scott}$. In order for~$\delta$ to be an applicative
morphism, it must have a realizer~$\rho \in \comp{\Baire}$ such that
                                %
\begin{equation*}
  \delta(x, \alpha) \land
  \delta(y, \beta)
  \lthen
  \delta(x \cdot y, \rho \mid \alpha \mid \beta).
\end{equation*}
                                %
Equivalently, we may require that $\delta(x \cdot y, \rho \mid
\pair{\alpha, \beta})$. Such a~$\rho$ can be obtained as follows.
To determine the value $(\rho \mid \pair{\alpha, \beta})
(\code{(m, n)})$, let $A = \set{\alpha_0, \ldots, \alpha_{m-1}}$
and $B = \set{\beta_0, \ldots, \beta_{m-1}}$. If there exists $k
\in B$ such that $k = 1 + \code{(n, j)}$ and $\mathsf{finset}(j) \subseteq
A$ then the value is~$n+1$, otherwise it is~$0$. Clearly, this is an
effective procedure, therefore it is continuous and realized by an
element~$\rho \in \comp{\Baire}$. If we compare the definition of~$\rho$ to
the definition of application in~$\Scott$, we see that they match.

Let us show that $\iota \dashv \delta$ is an applicative retraction.
Suppose $\alpha \in \Baire$, $x = \iota(\alpha)$, and $\delta(x,
\beta)$. We can effectively reconstruct $\alpha$ from~$\beta$,
because $\beta$ enumerates the initial segments of~$\alpha$. This
shows that $\delta \circ \iota \preceq \id[\Baire]$. Also, given
$\alpha$ we can easily construct a sequence~$\beta$ which enumerates
the initial segments of~$\alpha$, therefore $\id[\Baire] \preceq
\delta \circ \iota$, and we conclude that $\delta \circ \iota \sim
\id[\Baire]$.

To see that $\iota \circ \delta \preceq \id[\Scott]$, consider $x, y \in
\Scott$ and $\alpha \in \Baire$ such that $\delta(x, \alpha)$ and $y =
\iota(\alpha)$. The sequence~$\alpha$ enumerates~$x$, and $y$
consists of the initial segments of~$\alpha$. Hence, we can
effectively reconstruct $x$ from $y$, by
                                %
\begin{equation*}
  m \in x
  \iff
  \some{n \in y} n = 1 + \code{a} \land \some{i < |a|} m = a_i.
\end{equation*}


\subsubsection{Equivalence of Reflexive Continuous Lattices}
\label{sec:equivalence_reflexive_continuous_lattices}%

\index{equivalence!of reflexive continuous lattices}%
\index{continuous!lattice}%

In Subsection~\cref{sec:reflexive_cpo_model} we saw that a reflexive
CPO is a model of the untyped $\lambda$-calculus, hence a combinatory
algebra. So far we have considered two reflexive CPOs, the graph
model~$\Scott$ and the universal domain~$\UU$. In this subsection we show
that every countably based reflexive continuous lattice is equivalent
to~$\Scott$. Thus, as far as categories of modest sets on countably based
reflexive continuous lattices are concerned, we do not lose any
generality by considering only the graph model~$\Scott$.

We only consider countably based continuous lattices.
%
\index{reflexive!continuous lattice}%
\index{continuous!reflexive lattice}%
%
A continuous lattice~$L$ is \emph{reflexive} if it contains at least
two elements and its continuous function space~$L^L$ is a retract
of~$L$.

\begin{proposition}
  \label{th:PP_retract_of_reflexive_lattice}%
  The graph model is a continuous retract of every reflexive
  continuous lattice.
\end{proposition}

\begin{proof}
  \newcommand{\retr}[1]{\mathscr{R}(#1)}
  %
  Let $L$ be a reflexive continuous lattice. Then we have a
  section-retraction pair
  %
  \begin{equation*}
    \xymatrix@+1.5em{
      {L^L} \ar@<0.5ex>[r]^{\Gamma} &
      {L} \ar@<0.5ex>[l]^{\Lambda}.
      }
  \end{equation*}
  %
  The lattice~$L$ is a model of the untyped $\lambda$-calculus.
  The product $L \times L$ is a retract of~$L$. The section $p^{+}: L
  \times L \to L$ and the retraction $p^{-}: L \to L \times L$ can
  be most conveniently expressed as the untyped $\lambda$-terms as
  %
  \begin{align*}
    p^{+} \pair{x, y} &= \lam{z}{z x y},
    &
    p^{-} z &= \pair{\fst{z}, \snd{z}},
  \end{align*}
  %
  where $\fst{z} = z (\lam{x \, y} x)$ and $\snd{z} = z (\lam{x \,
    y} y)$. Let $p = p^{+} \circ p^{-}$.
  
  Let $\retr{L}$ be the continuous lattice of retractions on~$L$.
  There is a continuous pairing operation on~$\retr{L}$, defined by
  %
  \begin{equation*}
    A \times B =
    \lam{\annot{z}{L}} p^{+} \pair{
       A (\fst{(p z)}),
       B (\snd{(p z)})}.
  \end{equation*}
  %
  The Sierpinski space $\Sierpinski$ is a retract of~$L$, with the
  corresponding retraction $S: L \to L$
  %
  \begin{equation*}
    S x = 
    \begin{cases}
      \bot & \text{if $x = \bot$} \\
      \top & \text{if $x \neq \bot$}
    \end{cases}
  \end{equation*}
  %
  Let $P$ be the least retraction on $L$ satisfying the recursive
  equation
  %
  \begin{equation*}
    P = S \times P.
  \end{equation*}
  %
  The retraction $P$ is the directed supremum of the chain of
  retractions $P_0 \leq P_1 \leq \cdots$, defined by
  %
  \begin{align*}
    P_0 &= \bot,
    &
    P_{k+1} &= S \times P_k.
  \end{align*}
  %
  We abuse notation slightly and denote a retraction and its lattice
  of fixed points with the same letter. Clearly $P_k \cong
  \Sierpinski^k$ for every $k \in \NN$. Thus,~$P$ is isomorphic to the
  limit/colimit of the chain
  %
  \begin{equation*}
    \xymatrix@+1.5em{
      {\bot} \ar@<0.5ex>[r] &
      {\Sierpinski} \ar@<0.5ex>[l] \ar@<0.5ex>[r] &
      {\Sierpinski^2} \ar@<0.5ex>[l] \ar@<0.5ex>[r] &
      {\Sierpinski^3} \ar@<0.5ex>[l] \ar@<0.5ex>[r] &
      {\cdots} \ar@<0.5ex>[l],
    }
  \end{equation*}
  %
  where the pairs of arrows between the stages are the canonical
  section-retraction pairs between $\Sierpinski^k$ and
  $\Sierpinski^{k+1}$.  The limit/colimit is the
  lattice~$\Sierpinski^{\NN}$, which is isomorphic to~$\Scott$.
\end{proof}

\begin{corollary}
  \label{th:reflexive_lattices_equivalent}%
  Every two countably based reflexive continuous lattices are retracts 
  of each other, hence they are equivalent as combinatory algebras.
\end{corollary}

\begin{proof}
  If $L$ and $M$ are reflexive continuous lattices, then they are
  retracts of each other because each is a retract of~$\Scott$, and $\Scott$
  is a retract of each of them by
  Proposition~\ref{th:PP_retract_of_reflexive_lattice}.  There are
  section-retraction pairs
                                %
  \begin{align*}
    &\xymatrix@+1.5em{
      L \ar@<0.5ex>[r]^{\lambda^{+}} &
      M \ar@{->>}@<0.5ex>[l]^{\lambda^{-}}
      },
    &
    &\xymatrix@+1.5em{
      M \ar@<0.5ex>[r]^{\mu^{+}} &
      L \ar@{->>}@<0.5ex>[l]^{\mu^{-}}
      }.
  \end{align*}
                                %
  The equivalence $L \sim M$ is witnessed by the sections
  $\lambda^{+}$ and $\mu^{+}$. We omit the details.
\end{proof}


%%% Local Variables:
%%% mode: latex
%%% TeX-master: "notes-on-realizability"
%%% End:
