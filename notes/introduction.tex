\chapter{Introduction}
\label{chap:introduction}

\section{Background material}
\label{sec:background-material}

In this section we overview a selection of concepts which we need
later on. We also fix notation and a number of definitions. At the
moment the sections are not listed in any particular order.

\subsubsection*{Free and bound variables}

Occurrences of variables in an expression may be \defemph{free} or
\defemph{bound}. Variables are bound when they are used to indicate the
range over which an operator acts. For example, in expressions
%
\begin{equation*}
  \all{x}{\RR}{x^2 + y \geq 0},
  \qquad\qquad
  \sum_{k = 0}^n \frac{1}{k^2},
  \qquad\qquad
  \int_a^b f(t) \, dt,
\end{equation*}
%
the variables $x$, $k$, and $t$ are bound by the operators $\forall$,
$\sum$, and $\int$, respectively. The remaining variables are free. It
is really the \defemph{occurrence} of a variable that is bound or free,
not the variable itself. In
%
\begin{equation*}
  P(x) \lor \some{x} \lnot Q(x)
\end{equation*}
%
the left-most occurence of $x$ is free whereas the other two are bound
by~$\exists$.

\subsubsection*{Functions}

The set of all functions from $A$ to $B$ is denoted by $B^A$ as well as $A \to B$. The arrow associates to the right,
$A \to B \to C$ is $A \to (B \to C)$. We write $f : A \to B$ instead of $f \in A \to B$. If $f : A \to B$ and $x \in A$, the application $f(x)$ is also written as $f\, x$. We often work with \defemph{curried} functions which take several
arguments in succession, i.e., if $f : A \to B \to C$ then $f$ takes $x \in A$, and $y \in B$ to produce an element
$f(x)(y)$ in $C$, also written $f\, x\, y$.


\subsubsection*{Partial functions}

A \defemph{partial} function\sidenote{In the literature on Type Two
  Effectivity the common notation is $f \mathbin{{:}{\subseteq}} A \to
  B$.} $f: A \parto B$ is a function that is defined on a subset
$\dom{f} \subseteq A$, called the \emph{domain} of~$f$. Sometimes
there is confusion between the domain~$\dom{f}$ and the set~$A$, which
is also called the domain. We therefore call $\dom{f}$ the
\defemph{support} of~$f$. If $f: A \parto B$ is a partial function and $x
\in A$, we write $\defined{f\, x}$ to indicate that $f x$ is defined.
For an expression~$e$, we also write $\defined{e}$ to indicate
that~$e$ and all of its subexpressions are defined. The
symbol~$\downarrow$ is sometimes inserted into larger expressions, for
example, $\defined{f\, x} = y$ means that $f x$ is defined and is
equal to~$y$. If $e_1$ and $e_2$ are two expressions whose values are
possibly undefined, we write $e_1 \kleq e_2$ to indicate that either
$e_1$ and $e_2$ are both undefined, or they are both defined and
equal. The notation $e_1 \klgeq e_2$ means that if $e_1$ is defined
then $e_2$ is defined and they are equal. Thus we have
%
\begin{equation*}
  e_1 \kleq e_2 \iff e_1 \klgeq e_2 \land e_2 \klgeq e_1.
\end{equation*}

A partial map $f: X \parto Y$ between topological spaces~$X$ and~$Y$
is said to be \defemph{continuous} when it is continuous as a total map
$f: \dom{f} \to Y$, where the support $\dom{f} \subseteq
X$ is equipped with the subspace topology.



\subsubsection*{Primitive recursive and recursive function}

The \defemph{primitive recursive function} are those function $\NN^k \to
\NN$ that are built inductively from the following functions and operations:
%
\begin{enumerate}
\item constant functions $f(n_1, \ldots, n_k) = c$, where $c \in \NN$,
\item projections $p_i(n_1, \ldots, n_k) = n_i$, where $1 \leq i \leq k$,
\item the successor function $s(n) = n + 1$,
\item composition of functions,
\item primitive recursion: given primitive recursive $f : \NN^k \to
  \NN$ and $g : \NN^{k+2} \to \NN$, the function $h : \NN^{k+1} \to
  \NN$ defined by
  %
  \begin{align*}
    h(0, n_1, \ldots, n_k) &= f(n_1, \ldots, n_k), \\
    h(n+1, n_1, \ldots, n_k) &= g(h(n, n_1, \ldots, n_k), n, n_1,
    \ldots, n_k)    
  \end{align*}
  %
  is primitive recursive.
\end{enumerate}
%
Every primitive recursive function is computable, but not every
computable function is primitive recursive.\sidenote{The Ackermann
  function is computable but not primitive recursive.}


The \defemph{(general) partial recursive functions} are built from the above operations and \defemph{minimization}: given a partial recursive $f : \NN^{k+1} \parto \NN$ the function $g : \NN^k \parto \NN$, defined by
%
\begin{equation*}
  g(n_1, \ldots, n_k) = \min_n (f(n, n_1, \ldots, n_k) \neq 0),
\end{equation*}
%
is partial recursive as well. When no $n$ satisfies $f(n, n_1, \ldots, n_k) \neq 0$ the value $g(n_1, \ldots, n_k)$ is undefined.

The \defemph{general recursive functions} are those partial recursive functions whose domain and support coincide.


\subsubsection*{Order theory}

A \defemph{preorder} $(P, {\leq})$ is a set with a reflexive and
transitive relation.
%
A \defemph{partially ordered set (poset)} $(P, {\leq})$ is a set with a
reflexive, transitive, and anti-symmetric relation.

A function $f : P \to Q$ between posets is \defemph{monotone} if $x \leq
y$ in $P$ implies $f(x) \leq f(y)$ in $Q$.

A subset $S \subseteq P$ is an \defemph{upper set} if $x \in S$ and $x
\leq y$ implies $y \in S$. Similarly, it is a \defemph{lower set} if $y
\in S$ and $x \leq y$ implies $x \in S$.
%
A subset $S \subseteq P$ of a poset $(P, {\leq})$ is \defemph{directed}
if it is non-empty and for every $x, y \in S$ there exists $z \in S$
such that $x \leq z$ and $y \leq z$.
%
An \defemph{upper bound} of a subset $S \subseteq P$ in a poset is an
element $x \in P$ such that $y \leq x$ for all $y \in S$.
%
The \defemph{supremum} $\sup S$ of a subset $S \subseteq P$ in a poset is
its least upper bound, if it exists. More precisely, it is an upper
bound $x$ for $S$ such that if $y$ is also an upper bound for~$S$ then
$x \leq y$.

A \defemph{directed-complete partial order (dcpo)} is a poset in which
every directed set has a supremum. Let $(D, {\leq})$ be a dcpo. For
$x, y \in D$ we say that~$x$ is \defemph{way below} $y$, written $x \wayb
y$, when for every directed $S \subseteq D$ such that $y \leq \sup S$
there exists $z \in S$ for which $x \leq z$. An element $x \in D$ is
\defemph{compact} (or \defemph{finite}) when $x \wayb x$. A subset $U
\subseteq D$ is \defemph{Scott open} if it is an upper set and is
inaccessible by suprema of directed sets, which means that, for every
directed $S \subseteq D$, if $\sup S \in U$ then already $x \in U$ for
some $x \in S$. The Scott open sets form the \defemph{Scott topology}
of~$D$.

If $D$ and $E$ are dcpos then a function $f : D \to E$ is continuous
with respect to the Scott topologies precisely when it preserves
suprema of directed sets. It follows that such a function is monotone.


\subsubsection*{Topology}

A topological space~$X$ is \defemph{$T_0$-space} if each point is
uniquely determined by its open neighborhoods: for all $x, y \in
X$,
%
\begin{equation*}
  (\all{U \in \topol{X}} (x \in U \iff y \in U)) \lthen x = y.
\end{equation*}

A topological space is \defemph{zero-dimensional} if it has a basis
consisting of clopen sets.





%%% Local Variables: 
%%% mode: latex
%%% TeX-master: "notes-on-realizability"
%%% End: 

