\documentclass{article}

\usepackage[T1]{fontenc}
\usepackage[utf8]{inputenc}
\usepackage{amsmath,amsfonts,amsthm,amssymb}
\usepackage[colorlinks]{hyperref}
\usepackage{listings}
\lstset{basicstyle=\footnotesize\ttfamily}

\newcommand{\set}[1]{\{#1\}}
\newcommand{\such}{\mid}

%%% Blackboard bold letters
\newcommand{\NN}{\mathbb{N}}
\newcommand{\NNx}{{\NN^{+}}}
\newcommand{\ZZ}{\mathbb{Z}}
\newcommand{\QQ}{\mathbb{Q}}
\newcommand{\RR}{\mathbb{R}}
\newcommand{\CC}{\mathbb{C}}

%%% quantifiers
\newcommand{\all}[3]{\forall\, #1 \,{\in}\, #2\,.\left(#3\right)}
\newcommand{\some}[3]{\exists\, #1 \,{\in}\, #2\,.\left(#3\right)}
\newcommand{\exactlyone}[3]{\exists!\, #1 \,{\in}\, #2\,.\left(#3\right)}
\newcommand{\lam}[3]{\lambda #1 \,{\in}\, #2\,.\left(#3\right)}
\newcommand{\uall}[2]{\forall\, #1\,.\left(#2\right)}
\newcommand{\usome}[2]{\exists\, #1\,.\left(#2\right)}
\newcommand{\uexactlyone}[3]{\exists!\, #1\,.\left(#2\right)}
\newcommand{\ulam}[2]{\lambda #1 .\left(#2\right)}
\newcommand{\xall}[3]{\forall\, #1 \,{\in}\, #2\,.\,#3}
\newcommand{\xsome}[3]{\exists\, #1 \,{\in}\, #2\,.\,#3}
\newcommand{\xexactlyone}[3]{\exists!\, #1 \,{\in}\, #2\,.\,#3}
\newcommand{\xuall}[2]{\forall\, #1\,.\,#2}
\newcommand{\xusome}[2]{\exists\, #1\,.\,#2}
\newcommand{\xuexactlyone}[2]{\exists!\, #1,.\,#2}
\newcommand{\xlam}[3]{\lambda #1 \,{\in}\, #2\,.\,#3}
\newcommand{\xulam}[2]{\lambda #1 .\,#2}
\newcommand{\tlam}[3]{\lambda #1 \,{:}\, #2\,.\,\left(#3\right)}
\newcommand{\xtlam}[3]{\lambda #1 \,{:}\, #2\,.\,#3}


\title{Exercises on realizability\\
(Midlands Graduate School 2022)}
\author{Andrej Bauer}

\begin{document}

\maketitle

\section*{Instructions}
\label{sec:instructions}

These are the exercises accompanying the ``Realizability'' lectures, given at the Midlands Graduate School 2022.
During the exercise classes we shall solve them in the following way:
%
\begin{enumerate}
\item I will explain the exercise and review the background knowledge.
\item You will talk to your neighbor for 10 minutes about the exercise.
\item We will discuss the solution together.
\end{enumerate}
%
You are of course welcome to work on the exercises ahead of the class. Please consult the accompanying ``Notes on realizability'' (\href{http://www.andrej.com/zapiski/MGS-2022/notes-on-realizability.pdf}{PDF}, \href{https://github.com/andrejbauer/notes-on-realizability}{GitHub repository}) and discuss the exercises on the
MGS Discord server.

\section{Lecture 1: Models of computation}
\label{sec:models-computation}

\subsection{Dove-tailing}
\label{sec:dove-tailing}

Let $\xpr_n$ be the partial computable function computed by the Turing machine encoded by~$n \in \NN$.
Write $\defined{e}$ to indicate that the value of~$e$ is defined.
%
Show that there is a partial computable function $f : \NN \parto \NN$ such that, for all $n, k \in \NN$,
%
\begin{equation*}
f n = k \lthen \defined{\pr{n} k}
\quad\text{and}\quad
(\some{m \in \NN} \defined{\pr{n} m}) \lthen \defined{f n}.
\end{equation*}
%
You are not expected to construct an actual Turing machine, but you should describe in some detail how the machine works.

Is there a Haskell function $\mathtt{f} : (\mathtt{Int} \to \mathtt{Int}) \to \mathtt{Int}$ with an analogous property?

\subsection{Modulus of continuity}
\label{sec:modulus-continuity}


Write $\code{e}$ for the encoding of entity~$e$ as a natural number, and let $\Baire = \NNNN$ be the Baire space.
Given $\alpha \in \Baire$, let $\seg{\alpha}{k} = [\alpha 0, \ldots, \alpha (k-1)]$.
Write $0^{(\omega)}$ for the sequence of all zeroes and $0^{(k)}$ for the list of~$k$ zeroes $[0, \ldots, 0]$.

Recall how $\gamma \in \Baire$ encodes a map $\xfpr_\gamma : \Baire \parto \Baire$.
%
Given input $\alpha \in \Baire$ and $i \in \NN$, we compute the value of $\xfpr_\gamma \, \alpha \, i$ by looking up in succession
%
\begin{align*}
  & \gamma (\code{\cons{i}{[]}}), \\
  & \gamma (\code{\cons{i}{[\alpha 0]}}), \\
  & \gamma (\code{\cons{i}{[\alpha 0, \alpha 1]}}), \\
  & \gamma (\code{\cons{i}{[\alpha 0, \alpha 1]}}), \\
  & \gamma (\code{\cons{i}{[\alpha 0, \alpha 1, \alpha 2]}}), \\
  & \ \vdots
\end{align*}
%
until we find the first non-zero one $j > 0$ and output $j - 1$.
%
More precisely, define $\ell(\gamma, \alpha) : \NN \parto \NN$ by
%
\begin{equation*}
  \ell(\gamma, \alpha)(i) = \gamma(\code{\cons{i}{\seg{\alpha}{k}}}) - 1
  \quad\text{where}\quad
  k = \min\nolimits_k (\gamma(\code{\cons{i}{\seg{\alpha}{k}}}) \neq 0)
\end{equation*}
%
(if not such $k$ exists then $\ell(\gamma, \alpha)(i)$ is undefined), and let the map $\xfpr_\gamma : \Baire \parto \Baire$ encoded by $\gamma$ be
%
\begin{equation*}
  \fpr{\gamma}{\alpha} =
  \begin{cases}
    \ell(\gamma, \alpha) & \text{if $\ell(\gamma, \alpha)$ is a total map,} \\
    \text{undefined} & \text{otherwise.}
  \end{cases}
\end{equation*}

Suppose $f : \Baire \to \Baire$ is a \emph{total} map satisfying
%
\begin{equation}
  \label{eq:modulus}%
  \all{k \in \NN} \some{m \in \NN}
  \all{\alpha \in \Baire} \seg{\alpha}{m} = 0^{(k)} \lthen (f \alpha) k = (f 0^{(\omega)}) k.
\end{equation}
%
We say that $m$ is a \emph{modulus of continuity} for $f$ at $k$.

Verify that $f$ satisfies~\eqref{eq:modulus} if, and only if, it is continuous with respect to the product topology on~$\Baire$.

Construct a Type 2 Turing machine which accepts as input $\gamma \in \Baire$ and $k \in \NN$ and outputs a modulus of continuity for~$\xfpr_\gamma$ at~$k$. We assume that $\gamma$ encodes a \emph{total} map $\Baire \to \Baire$. Concretely, the read-only input tape contains~$k$ followed by~$\gamma$ (in Type 2 computability tape cells contain numbers or blanks). The machine should terminate and output a suitable~$m$. Of course, it suffices to describe the machine informally.

\medskip

Is there a corresponding Haskell map $\mathtt{modulus} : \mathtt{Int} \to (\mathtt{Int} \to \mathtt{Int}) \to \mathtt{Int}$?


\subsection{Programming in $\lambda$-calculus}
\label{sec:encod-gener-recurs}

Use the online untyped $\lambda$-calculus interpreter at
%
\begin{quote}
  \url{http://www.andrej.com/zapiski/MGS-2022/lambda/}
\end{quote}
%
to implement a function which tests whether a given natural number is prime. Consult the notes section on the $\lambda$-calculus to get basic arithmetic and Booleans going.
%
If you prefer to use the much faster off-line interpreter, you can get compile the language \texttt{lambda} from the \href{http://plzoo.andrej.com/language/lambda.html}{Programming Languages Zoo} -- but refrain from looking at examples because they contain the solution.

The actual programming should be done in the privacy of your computer. During the exercise class you should just write down a list of specific functions (zero testing, searching, division, etc.) that you need to implement. Assuming you have implemented those, write down the primality testing function. To get you started, consult Figure~\ref{fig:lambda} (where \lstinline|^| stands for $\lambda$):

\begin{figure}[htp]
  \centering
\begin{lstlisting}
-- the constant function
K := ^ x y . x ;

-- pairing
pair := ^ a b . ^p . p a b ;
first := ^ p . p (^x y . x) ;
second := ^ p . p (^x y. y) ;

-- Booleans
true  := ^x y . x ;
false := ^x y . y ;
if := ^u . u ;

-- recursive definitions
fix := ^f . (^x . f (x x)) (^x . f (x x)) ;

-- arithmetic
0  := ^f x . x ;
1  := ^f x . f x ;
2  := ^f x . f (f x) ;
3  := ^f x . f (f (f x)) ;
+ := ^n m f x . (n f) ((m f) x) ;
* := ^n m f x . (n (m f)) x ;
succ := ^n f x . f (n f x) ;
pred := ^n . second (n (^p. pair (succ (first p)) (first p)) (pair 0 0)) ;
iszero := ^n . (n (K false)) true ;
\end{lstlisting}
  \caption{Basic programming in \texttt{lambda}.}
  \label{fig:lambda}
\end{figure}


\section{Lecture 2: Assemblies}
\label{sec:assemblies}

\subsection{Assemblies of finite sets of numbers}
\label{sec:assembl-finite-sets}

We work over Kleene's first algebra, so all the realizers are numbers and we compute with Turing machines.
Recall that $\pr{m}{n}$ is the $m$-th Turing machine applied to~$n$.

A reasonable coding of pairs and lists with numbers is as follows:
%
\begin{align*}
  \code{(m, n)} &= 2^m (2 n + 1), \\
  \code{[]} &= 0, \\
  \code{\cons{m}{\ell}} &= 2^m (2 \code{\ell} +1).
\end{align*}
%
Let $$\FF = \set{ S \subseteq \NN \such \text{$S$ is finite} }$$ be the set of all finite subsets of~$\NN$. We consider 
different ways of equipping $\FF$ with a realizability relation:
%
\begin{enumerate}
\item[(a)] $F_1 = \nabla \FF$,

\item[(b)] $F_2 = (\FF, {\rz[F_2]})$ where
  %
  \begin{equation*}
    r \rz[F_2] S \iff
    \some{a_1, \ldots, a_n \in \NN}
    r = \code{[a_1, \ldots, a_n]} \land S = \set{a_1, \ldots, a_n}.
  \end{equation*}
  %
  In words, $r$ realizes $S$ when it encodes a finite enumeration of the elements of~$S$.

\item[(c)] $F_3 = (\FF, {\rz[F_3]})$ where
  %
  \begin{equation*}
    r \rz[F_3] S \iff
    \begin{aligned}[t]
      & \some{a_1, \ldots, a_n \in \NN} \\
      & \qquad a_1 < a_2 < \cdots < a_n \land {} \\
      & \qquad r = \code{[a_1, \ldots, a_n]} \land S = \set{a_1, \ldots, a_n}.
    \end{aligned}
  \end{equation*}
  %
  In words, $r$ realizes $S$ when it encodes the elements of~$S$ in a strictly increasing order.

\item[(d)] $F_4 = (\FF, {\rz[F_4]})$ where
  %
  \begin{equation*}
    r \rz[F_4] S \iff
    \begin{aligned}[t]
      & (\all{k \in \NN} \defined{\pr{r}{k}} \land (\pr{r}{k} > 0 \lthen \pr{r}{k} - 1\in S)) \land {} \\
      & (\all{m \in S} \some{k \in \NN} \pr{r}{k} = m + 1)
    \end{aligned}
  \end{equation*}
  %
  In words, $r$ realizes $S$ when it satisfies the following conditions:
  %
  \begin{enumerate}
  \item for all $k \in \NN$, $\pr{r}{k}$ is defined,
  \item for all $k \in \NN$, if $\pr{r}{k} > 0$ then $\pr{r}{k} - 1 \in S$,
  \item for every $m \in S$ there is $k \in \NN$ such that $\pr{r}{k} = m + 1$.
  \end{enumerate}
  %
  That is, $r$ is an infinite enumeration of the elements of~$S$, where $m + 1$ is understood to enumerate $m$, and $0$ does not enumerate anything. (This trick is needed so that $\emptyset$ has a realizer).
\end{enumerate}

The assemblies $F_1, F_2, F_3, F_4$ all share the same underlying set~$\FF$. For each $i, j \in \set{1, 2, 3, 4}$, determine whether $\id[\FF] : F_i \to F_j$ is realized.

\subsection{Cantor space}
\label{sec:cantor-space}

We work with the first Kleene algebra $\klone$. Define the following assemblies:
%
\begin{itemize}
\item Booleans: $\two = (\set{0,1}, {\rz[\two]})$ with $r \rz[\two] b \liff r = b$.
\item Natural numbers: $N = (\NN, {\rz[N]})$ with $r \rz[N] n \liff r = n$.
\end{itemize}
%
Explain in your own words what the elements of the exponential assembly $N \to \two$ are. Which well-known concept from computability theory does $N \to \two$ embody?


\subsection{Epis and regular epis}
\label{sec:epis-regular-epis}

In the category of sets a map $f : X \to Y$ is an \emph{epimorphism (epi)} if, and only if, it is surjective: $\all{y \in Y} \some{x \in X} f x = y$. (In general, the epimorphisms are those morphisms that can be canceled on the right.)

Verify that an assembly map $f : S \to T$ is epi if, and only if, it is epi as a map $f : \S{S} \to \T{T}$ between the underlying sets.

Without having heard the lecture on realizability logic, but using your intuition for what it means to have ``computational evidence'', formulate what it would mean for an assembly map $f : S \to T$ to be ``computationally surjective'', i.e., when would you expect a realizer $\R{r}$ to witness the fact ``$f$ is surjective''?


\section{Lecture 3: Realizability logic and type theory}
\label{sec:lecture-3}


\subsection{$\neg\neg$-stable propositions}
\label{sec:negn-stable-prop}

In intuitionistic logic negation $\neg P$ is an abbreviation for $P \lthen \bot$. A logical proposition $P$ is said to be \emph{$\neg\neg$-stable} when $\neg\neg P \lthen P$.

Prove that $P \lthen \neg\neg P$ holds intuitionistically by giving a proof. You may consult a proof assistant, but at the end you should be in possession of a human-readable proof.

We would like to characterize the $\neg\neg$-stable realizability predicates. Working with assemblies over an arbitrary pca $\AA$, complete the following characterizations, where $r \in \AA$, $p \in \Pred{S}$ and $x \in \S{S}$:
%
\begin{align*}
  r \rz (\neg p) x &\iff \text{(insert characterization here)} \\
  r \rz (\neg \neg p) x &\iff \text{(insert characterization here)}
\end{align*}
%
For reference, here is how $\bot$ and $\lthen$ are characterized:
%
\begin{align*}
  r \rz \bot &\qquad\text{never}, \\
  r \rz (p \lthen q) x &\iff
    \all{a, b \in \AA} \text{if $a \rz[S] x$ and $b \rz p x$ then $r \cdot a \cdot b \rz q x$}.
\end{align*}
%
After that, show that the following holds:
%
\begin{quote}
  \emph{A realizability predicate $p \in \Pred{S}$ is $\neg\neg$-stable if, and only if, there is $r \in \AA$ such that for all $x \in \S{S}$ and $a \in \AA$, if $a \rz[S] x$ and $p x \neq \emptyset$ then $r \cdot a \rz p x$.
}\end{quote}


\subsection{Markov's principle}
\label{sec:markov-principle}

Working over the first Kleene algebra~$\klone$, is the following statement realized?
%
\begin{equation*}
  \all{f \in \two^\NN}
  \neg\neg(\some{n \in \NN} f n = 1)
  \lthen
  \some{n \in \NN} f n = 1
\end{equation*}
%
Is it also realized when we use an arbitrary pca~$\AA$?

\subsection{Countable choice}
\label{sec:countable-choice}

The \emph{Axiom of Countable Choice} states that a family $A : \NN \to \Set$ of inhabited sets has a choice map:
%
\begin{equation*}
  \textstyle
  (\all{n \in \NN} \some{x \in A_n} \top) \lthen \some{u \in \prod_{n \in \NN} A_n} \top.
\end{equation*}
%
Verify that the axiom is inter-derivable with the statement that every total relation $R \subseteq \NN \times B$ has a choice function:
%
\begin{equation*}
  (\all{n \in \NN} \some{x \in B} R(n, x)) \lthen
  \some{v \in B^{\NN}} \all{n \in \NN} R(n, v n).
\end{equation*}
%
Finally, show that in realizability logic Countable Choice, in either one of the above forms, is realized.


\subsection{The enumeration axiom}
\label{sec:enumeration-axiom}

Say that an assembly $S$ is \emph{enumerable} if
%
\begin{equation*}
  \some{e : \NN \to \one + S}
  \all{x \in S}
  \some{n \in \NN}
  e n = \iota_2 x.
\end{equation*}
%
is realized. Give an explicit description of a realizer for the above statement.

Let $\mathcal{E}$ be the assembly of enumerable subsets of~$\NN$, which is defined as follows. For $r \in \AA$ and $U \subseteq \NN$, define
%
\begin{equation*}
  r \rz[\mathcal{E}] U
  \iff
  \begin{aligned}[t]
    & (\all{n \in \NN} \defined{r \cdot \numeral{n}}) \land {} \\
    & (\all{k \in \NN} k \in U \iff \some{n \in \NN} r \cdot \numeral{n} = \numeral{k + 1})
  \end{aligned}
\end{equation*}
%
and let $\S{\mathcal{E}} = \set{U \subseteq \NN \mid \some{r \in \AA} r \rz[\mathcal{E}] U}$.

Show that $\mathcal{E}$ is isomorphic to the assembly
%
\begin{equation*}
  \textstyle
  \Sigma_{u : \NN \to \nabla \set{0,1}}
  \some{e : \NN \to \NN}
  \all{k \in \NN}
  (u k = 1 \liff \some{n \in \NN} e n = k + 1).
\end{equation*}
%
(Recall that $\exists$ is defined as the propositional truncation of $\Sigma$.)

\medskip

What well-known concept from computability theory does $\mathcal{E}$ represent in $\Asm{\klone}$?
Show that in $\Asm{\klone}$ the statement ``$\mathcal{E}$ is countable'' is realized.

What well-known concept from computability theory does $\mathcal{E}$ represent in $\Asm{\comp{\Baire}}$, where $\comp{\Baire}$ is the \emph{computable} Kleene's second algebra? Does $\Asm{\comp{\Baire}}$ realize ``$\mathcal{E}$ is countable''?


\end{document}
