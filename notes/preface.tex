\chapter*{Preface}
\addcontentsline{toc}{chapter}{Preface} % Add the preface to the table of contents as a chapter

It is not an exaggeration to say that the invention of modern computers was a direct consequence of the great advances of the 20th century logic: Hilbert's putting a decision problem on his list, Gödel's amazing exercise in programming with numbers, Church's invention of $\lambda$-calculus, Gödel's of general recursive functions, and Turing's of his machines.
%
Unfortunately, by the time computers took over the world and demanded a fitting foundation of mathematics,
generations of mathematicians had been educated with little regard or sensitivity to questions of computability and constructivity.
%
Some even cherished living in a paradise removed from earthly matters and encouraged others to take pride in the uselessness of their activity.
%
Today such mathematics persists as the generally accepted canon.

How is the working mathematician to understand and study computable mathematics?
%
Given their unshakable trust in classical mathematics, it is only natural for them to ``bolt on computability as an afterthought'', as was put eloquently by a friend of mine.
%
Indeed, this is precisely how many experts practice computable mathematics, and so shall we.

A comprehensive account of realizability theory would be a monumental work, which we may hope to see one day in the form of a sketch of an elephant. These notes are at best a modest introduction that aims to strike a balance between approachable concreteness and inspiring generality. Because my purpose was to educate, I did not hesitate to include informal explanations and recollection of material that I could have relegated to background reading. Suggestions for further reading will hopefully help direct those who seek deeper knowledge of realizability.

Realizability theory weaves together computability theory, category theory, logic, topology and programming languages. I therefore recommend to the enthusiastic students the adoption of the Japanese martial art principle \begin{CJK}{UTF8}{min}修行\end{CJK}.

An early version of these lecture notes were written to support a graduate course on computable topology, which I taught in 2009 at the University of Ljubljana. I copiously reused part of my dissertation. In 2022 I updated the notes and added a chapter on type theory, on the occasion of my lecturing at the Midlands Graduate School, hosted by the University of Nottingham.



\bigskip

\begin{flushright}
Andrej Bauer\\
Ljubljana, March 2022
\end{flushright}

%%% Local Variables:
%%% mode: latex
%%% TeX-master: "notes-on-realizability"
%%% End:
