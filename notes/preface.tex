\chapter*{Preface}
\addcontentsline{toc}{chapter}{Preface} % Add the preface to the table of contents as a chapter


I sometimes think it is unfortunate that the modern mathematics of
20${}^{\text{th}}$ century came before modern computers. Perhaps it is
true that computers would have never been invented without Hilbert's
putting a decision problem on his list, G\"odel's unbelievable
exercise in programming with numbers, the discovery of
$\lambda$-calculus, recursive functions, and Turing's machines, but by
the time computers ruled the world, generations of mathematicians had
been educated with little regard to questions about computability of
mathematical structures. They were told their world was a paradise and
were encouraged to take pride in the uselessness of their activity.
Today classical mathematics is taken for granted by the vast majority
of mathematicians, despite overwhelming evidence that we have entered
an era of computable---and therefore non-classical---mathematics.

How does a classically trained mathematician approach computability of
real numbers and other structures in mathematical analysis? Given the
unshakable edifice of classical mathematics that he knows, it is only
natural for him to ``bolt on computability as an afterthought'', as a
friend of mine once put it. Indeed, this is how computable mathematics
is practiced by most experts, and this is also the way in which we
approach the subject. However, after introducing the basic models of
computability and the basics of realizability theory, we take a
``logical'' look at the setup which reveals the connection with
constructive mathematics. It turns out that computable mathematics is
just ordinary, albeit constructive, mathematics developed in
mathematical universes that have computability built in from their
conception. In the final chapter we put on programmer's hat and
actually implement some of the computable structures in a real-world
programming language.

These lecture notes were prepared for a graduate course at the Faculty of Mathematics and
Physics, University of Ljubljana, Slovenia. I have included more material than I could
hope to cover in 15 lectures, 90~minutes each. I am aware that proper understanding of the
subject requires background knowledge in computability theory, analysis, topology, logic,
category theory, and programming. Therefore, the first exercise for the vigilant students
is to look up the meaning of the Japanese word \begin{CJK}{UTF8}{min}修行\end{CJK}, as
used in the training of martial arts.


\bigskip

\begin{flushright}
Andrej Bauer\\
Ljubljana, January 2009
\end{flushright}
