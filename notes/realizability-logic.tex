\chapter{Realizability and logic}
\label{sec:realizability-interpretation}

The idea that the elements of a set are represented by values of a
datatype is familiar to programmers. In the previous section we
expressed the idea mathematically in terms of realizability relations
and assemblies. What is perhaps less obvious is that realizability
carries over to logic as well. To every statement $\phi$ in
first-order logic we may assign its realizers, which are computable
witnesses showing why $\phi$ holds.

Let us first recall how the usual interpretation of classical
first-order logic works. A \emph{predicate} on a set~$S$ is a function
$\phi : S \to \two$, where $\two = \set{0, 1}$ is the set of truth
values. Such a predicate determines the subset $\set{x \in S \such
  \phi(x) = 1} \subseteq S$, called the \emph{extension} of~$\phi$,
and conversely, every $T \subseteq S$ determines the predicate
$\chi_T(x) = (\cond{x \in T}{1}{0})$, called the \emph{characteristic
  map} of~$T$. This correspondence is a bijection between $\two^S$ and
$\pow{S}$. Either can be used to interpret logic, but $\pow{S}$ is the
more suitable option in our case. Thus we view predicates on a set
$S$ as elements of the powerset $\pow{S}$.

The poset $\pow{S}$, ordered by $\subseteq$, is a Boolean algebra. The
least element is $\emptyset$, the largest $S$, the infimum of $A \in
\pow{S}$ and $B \in \pow{S}$ is their intersection $A \cap B$, the
supremum is the union $A \cup B$, and the complement of~$A$ is the
set-theoretic complement $S \setminus A$. This structure is used in
the interpretation of classical propositional calculus. Suppose $\phi$
is a proposition built from $\bot$, $\top$, $\land$, $\lor$,
$\implies$, $\lnot$, and \emph{atomic propositions} $p_1, \ldots,
p_n$. As soon as we assign subsets $P_i \subseteq S$ to the atomic
propositions $p_i$, the meaning of $\phi$ as a subset $\sem{\phi}
\subseteq S$ is defined inductively as follows:
%
\begin{align*}
  \sem{\bot} &= \emptyset,\\
  \sem{\top} &= S,\\
  \sem{\phi_1 \land \phi_2} &= \sem{\phi_1} \cap \sem{\phi_2},\\
  \sem{\phi_1 \lor \phi_2} &= \sem{\phi_1} \cup \sem{\phi_2},\\
  \sem{\phi_1 \implies \phi_2} &= (S \setminus \sem{\phi_1}) \cup \sem{\phi_2},\\
  \sem{\lnot \phi} &= S \setminus \sem{\phi},\\
  \sem{p_i} &= P_i.
\end{align*}
%
The interpretation can be further extended to the classical predicate
calculus, which involves the quantifiers $\exists$ and $\forall$.
Suppose $\phi$ is a predicate with free variables $x$ and $y$ ranging
over sets $S$ and $T$, respectively. Then the interpretation of $\phi$
is a subset $\sem{\phi} \subseteq S \times T$. From it we define the
interpretation of $\some{x}{S}{\phi}$ and $\all{x}{S}{\phi}$ to be
the subsets of~$T$ given by
%
\begin{align*}
  \sem{\some{x}{S}{\phi}} &=
  \sem{b \in T \such \some{a}{S}{(a,b) \in \sem{\phi}}}, \\
  \sem{\all{x}{S}{\phi}} &=
  \sem{b \in T \such \all{a}{S}{(a,b) \in \sem{\phi}}}.
\end{align*}
%
Observe that $\sem{\some{x}{S}{\phi}}$ is the least subset of~$T$
such that $\sem{\phi} \subseteq S \times \sem{\some{x}{S}{\phi}}$,
and that $\sem{\all{x}{S}{\phi}}$ is the largest subset of~$T$ such
that $S \times \sem{\all{X}{S}{\phi}} \subseteq \sem{\phi}$. We shall
use these characterizations of $\exists$ and $\forall$ in the setting
of realizability.

\section{Realizability predicates}
\label{sec:realizability-predicates}


The interpretation of logic in the category of assemblies goes much
along the same lines, except that we use monos instead of subsets. For
monos $u : U \monoto S$ and $t : T \monoto S$,
write $u \leq t$, or somewhat less precisely $U \leq T$,
if $u$ factors through $t$, i.e., there exists a realized $f : U
\to T$ such that
%
\begin{equation*}
  \xymatrix{
    &
    {S}
    &
    \\
    {U}
    \ar@{ >->}[ur]^{u}
    \ar[rr]_{f}
    &
    &
    {T}
    \ar[ul]_{t}
  }
\end{equation*}
%
commutes. If~$f$ exists then it is unique and is a mono.\sidenote{Such
  basic category-theoretic observations are excellent exercises. You
  should prove them yourself.} If $u \leq t$ and $t \leq u$ we say
that $u$ and $t$ are \emph{isomorphic} and write $u \equiv t$. The
collection $\Mono{S}$ of all monos with codomain~$S$ is a
preorder because $\leq$ is reflexive and transitive.\sidenote{By
  quotienting $\Mono{S}$ by $\equiv$ we get the partial order
  $\Sub{S} = \Mono{S}/{\cong}$ of \emph{subobjects}
  of~$S$, but we shall not do that because it is cumbersome to
  work with equivalence classes.}

While the powerset $\pow{S}$ is a complete Boolean algebra,
$\Mono{S}$ is neither complete nor Boolean. However, it is a
\emph{Heyting prealgebra}, which is enough to interpret
\emph{intuitionistic} propositional calculus, and as we shall see, the
quantifiers may be interpreted too.

We would like the interpretation of logic in $\AsmA$ to be in specific
form, for which we first need to focus on a particular kind of monos.
Let $S$ be an assembly, $\T{p}$ a type, and $p : S \to
\pow{\Atyp{\T{p}}}$ a function. We call $p$ a \defemph{realizability
  predicate} on $S$ and $\T{p}$ its \defemph{underlying type}. The
map $p$ is equivalently given as a relation ${-} \rz p({-})$ between
$\Atyp{\T{p}}$ and $S$. The correspondence between the map and the
relation is
%
\begin{equation*}
  \R{r} \rz p(x) \iff \R{r} \in p(x).
\end{equation*}
%
Read $\R{r} \rz p(x)$ as ``$\R{r}$ realizes $p(x)$''. Now define the
assembly $S_p$ by
%
\begin{align*}
  S_p &= \set{x \in S \such \some{\R{r}}{\Atyp{\T{p}}}{\R{r} \in p(x)}}, \\
  \T{S_p} &= \T{S} \times \T{p}, \\
  \combPair\;\R{x}\;\R{r} \rz[S_p] x &\iff
  \R{r} \in p(x) \land \R{x} \rz[S] x.
\end{align*}
%
The subset inclusion $\iota_p : S_p \to S$ is realized by $\combFst$.
The mono $\iota_p : S_p \to S$ is called a \defemph{standard
  mono}.

\begin{proposition}
  In $\AsmA$ every mono is equivalent to a standard one.
\end{proposition}

\begin{proof}
  Let $u : U \monoto S$ be a mono, realized by $\R{u}$.
  Let $\T{p} = \T{U}$ and
  %
  \begin{equation*}
    p(x) = \tbigcup_{y \in \inv{u}(x)} \Ex_U(y) =
    \set{\R{y} \in \Atyp{\T{p}} \such \some{y}{U}{u(y) = x \land \R{y} \rz[U] y}}.
  \end{equation*}
  %
  We claim that the standard mono $\iota_p : S_p \to S$ is
  isomorphic to~$u$. The function $u : U \to S$ restricts to a
  bijection $u : U \to S_p$, which is realized as a map from $U$
  to $S_p$ by
  $\tpcalam{y}{\T{U}} \combPair\;(\R{u}\;y)\;y$. The inverse
  $\inv{u} : S_p \to U$ is realized as a map from $S_p$ to
  $U$ by $\combSnd$.
\end{proof}

For the purposes of logic we may limit attention to standard monos,
because the interpretation of a predicate is determined only up to
isomorphism of monos. Furthermore, since a standard mono $S_p
\to S$ is uniquely determined by the corresponding realizability
predicate $p : S \to \pow{\Atyp{\T{p}}}$, we may express the
realizability intepretation of logic directly in terms of
realizability predicates.

Let $p$ and $q$ be realizability predicates on $S$, with the
corresponding monos $\iota_p : S_p \monoto S$ and $\iota_q
: S_q \monoto S$. We define $p \leq q$ to mean that
$\iota_p \leq \iota_q$.

\begin{lemma}
  \label{lemma:realizability-predicate-leq}%
  For realizability predicates $p$ and $q$ on an assembly $S$,
  $p \leq q$ if, and only if, there exists $\R{f} \in \subAtyp{\T{S}
    \to \T{p} \to \T{q}}$ such that, for all $x \in S$, whenever $\R{x}
  \rz[S] x$ and $\R{r} \rz p(x)$ then $\defined{\R{f}\;\R{x}\;\R{r}}$
  and $\R{f}\;\R{x}\;\R{r} \rz q(x)$. We say that $\R{f}$
  \defemph{witnesses} $p \leq q$.
\end{lemma}

\begin{proof}
  Suppose first that $p \leq q$. Then $\iota_p \leq \iota_q$, which
  means that $S_p \subseteq S_q$ and that the subset inclusion $i :
  S_p \to S_q$ is realized, say by $\R{i} \in \subAtyp{\T{S} \times \T{p}
    \to \T{S} \times \T{q}}$. Let $\R{f} = \tpcalam{x}{\T{S}}
    \tpcalam{r}{\T{p}}{\combSnd\;(\R{i}\;(\combPair\;x\;r))}$. Suppose $x
  \in S$, $\R{x} \in \Atyp{\T{S}}$, $\R{r} \rz p(x)$, and $\R{x} \rz[S]
  x$. Then $\combPair\;\R{x}\;\R{r} \rz[S_p] x$, therefore
  $\R{i}\;(\combPair\;\R{x}\;\R{r}) \rz[S_q] x$, which implies
  $\R{f}\;\R{x}\;\R{r} \rz q(x)$ because $\R{f}\;\R{x}\;\R{r} =
  \combSnd\;(\R{i}\;(\combPair\;\R{x}\;\R{r}))$.

  Conversely, let $\R{f}$ be a realizer as in the statement of the
  lemma, and suppose $\R{x} \rz[S] x$ and $\R{r} \rz p(x)$. Because
  $\R{f}\;\R{x}\;\R{r} \rz q(x)$, we see that $x \in S_q$. Hence $S_p
  \subseteq S_q$. A realizer for the inclusion map $i : S_p \to S_q$
  is $\R{i} = \tpcalam{u}{\T{S} \times
      \T{p}}{\combPair\;(\combFst\;u)\;(\R{f}\;(\combFst\;u)\;(\combSnd\;u))}$,
  as is easily verified.
\end{proof}

The lemma is our main tool for analyzing the preorder $\leq$. For
example, let us show that it is reflexive and transitive, i.e., that
the family $\Pred{S}$ of all realizability predicates on
$S$ is a \emph{preorder}. Reflexivity $p \leq p$ is witnessed by
the realizer $\tpcalam{x}{\T{S}} \tpcalam{u}{\T{p}}{u}$. For
transitivity, suppose $p \leq q$ and $q \leq r$ are witnessed by
$\R{f}$ and $\R{g}$. Then $p \leq r$ is witnessed by
$\tpcalam{x}{\T{S}} \tpcalam{u}{\T{p}}{\R{g}\;x\;(\R{f}\;x\;u))}$.


\section{The Heyting prealgebra of realizability predicates}
\label{sec:heyting-prealgebra}

We would like to give an interpretation of the intuitionistic
propositional calculus in the preorder $\Pred{S}$. Such an
interpretation is sound only if $\Pred{S}$ is a Heyting
prealgebra. Let us recall the definition.

\begin{definition}
  A \defemph{Heyting prealgebra} $(H, {\leq})$ is a preorder (reflexive
  and transitive) with elements $\bot$, $\top$ and binary operations
  meet $\land$, join $\lor$, and implication $\lthen$ such that, for all
  $p, q, r \in H$:
  %
  \begin{enumerate}
  \item $\bot$ is a smallest and $\top$ a largest element: $\bot \leq
    p \leq \top$,
  \item $p \land q$ is an infimum of $p$ and $q$: $p \land q \leq p$,
    $p \land q \leq q$, and whenever $r \leq p$ and $r \leq q$ then $r
    \leq p \land q$,
  \item $p \lor q$ is a supremum of $p$ and $q$: $p \leq p \lor q$, $q
    \leq p \lor q$, and whenever $p \leq r$ and $q \leq r$ then $p
    \lor q \leq r$,
  \item $p \lthen {-}$ is right adjoint to $p \land {-}$: $(p \lthen q)
    \land p \leq q$ and whenever $r \land p \leq q$ then $r \leq p
    \lthen q$.
  \end{enumerate}
  %
  A \defemph{Heyting algebra} is a Heyting prealgebra which is partially
  ordered.
\end{definition}

We say that elements $p, q \in H$ are \defemph{isomorphic}, written $p
\equiv q$, if $p \leq q$ and $q \leq p$. The operations of a Heyting
prealgebra are determined only up to isomorphism. For example, there
may be many smallest elements, but they are all isomorphic to each
other. We are not bothered by this. On the contrary, as programmers we
actually worry about choosing the most efficient realizer among many
equivalent ones.

\begin{proposition}
  The preorder $\Pred{S}$ of realizability predicates on an
  assembly $S$ is a Heyting prealgebra.
\end{proposition}

\begin{proof}
  Let $S$ be an assembly and $\unit$ any type with an element
  $\ttunit \in \subAtyp{\unit}$. Define the predicates $\bot, \top :
  S \to \pow{\Atyp{\unit}}$ by
  % 
  \begin{equation*}
    \bot(x) = \emptyset
    \qquad\text{and}\qquad
    \top(x) = \Atyp{\unit}.
  \end{equation*}
  % 
  That is, $\bot$ is realized by nothing and $\top$ by everything. It is
  easy to check that $\bot \leq p \leq \top$ for all $p \in
  \Pred{S}$.

  For predicates $p$ and $q$ on $S$, let $p \land q$ be the
  predicate whose realizers have the type $\T{p \land q} = \T{p} \times
  \T{q}$ and
  %
  \begin{equation*}
    p(x) \land q(x) = \set{\combPair\;\R{u}\;\R{v} \such
    \text{$\R{u} \rz p(x)$ and $\R{v} \rz q(x)$}}.
  \end{equation*}
  % 
  Here we wrote $p(x) \land q(x)$ instead of $(p \land q)(x)$. Let us
  verify that $p \land q$ is the infimum of $p$ and $q$. The
  inequalities $p \land q \leq p$ and $p \land q \leq q$ are witnessed
  by $\tpcalam{x}{\T{S}} \tpcalam{u}{\T{p} \times \T{q}}{\combFst\;u}$
  and $\tpcalam{x}{\T{S}} \tpcalam{u}{\T{p} \times
      \T{q}}{\combSnd\;u}$, respectively. Suppose $r \leq p$ and $r
  \leq q$ are witnessed by $\R{f}$ and $\R{g}$, respectively. Then $r
  \leq p \land q$ is witnessed by $\tpcalam{x}{\T{S}}
    \tpcalam{u}{\T{r}}{\combPair\; (\R{f}\;x\;u)\; (\R{g}\;x\;u)}$.

  Next we consider suprema. Define $p \lor q$ to be the predicate
  whose realizers have the type $\T{p \land q} = \T{p} + \T{q}$. There are
  two ways to realize $p(x) \lor q(x)$:
  %
  \begin{equation*}
    p(x) \lor q(x) =
    \set{\combLeft\;\R{u} \such \R{u} \rz p(x)} \cup
    \set{\combRight\;\R{v} \such \R{v} \rz q(x)}.
  \end{equation*}
  %
  The inequalities $p \leq p \lor q$ and $q \leq p \lor q$ are
  witnessed by $\tpcalam{x}{\T{S}} \tpcalam{u}{\T{p}}{\combLeft\;u}$
  and $\tpcalam{x}{\T{S}} \tpcalam{v}{\T{q}}{\combRight\;v}$,
  respectively. If $p \leq r$ and $q \leq r$ are witnessed by $\R{f}$
  and $\R{g}$ then $p \lor q \leq r$ is witnessed by
  %
  \begin{equation*}
    \tpcalam{x}{\T{S}} \tpcalam{w}{\T{p}+\T{q}}{
      \combCase\;w\;(\R{f}\;x)\;(\R{g}\;x)
    }.
  \end{equation*}
  
  Finally, define $p \lthen q$ to be the predicate whose realizers have
  the type $\T{p \lthen q} = \T{p} \to \T{q}$ and
  %
  \begin{equation*}
    p(x) \lthen q(x) = \set{
      \R{f} \in \Atyp{\T{p} \to \T{q}} \such
      \all{\R{u} \in \Atyp{\T{p}}}
        (\R{u} \rz p(x)
        \implies
        \defined{\R{f}\;\R{u}} \land
        \R{f}\;\R{u} \rz q(x))
    }.
  \end{equation*}
  %
  That is, $\R{f}$ maps realizers for $p(x)$ to realizers for $q(x)$.
  Note that $\R{f}$ need not be computable. Let us show that $p \lthen
  q$ has the desired properties. The inequality $(p \lthen q) \land p
  \leq q$ is witnessed by $\tpcalam{x}{\T{S}} \tpcalam{w}{(\T{p} \to
      \T{q}) \times \T{p}}{(\combFst\;w)\; (\combSnd\;w)}$. If $r \land p
  \leq q$ is witnessed by $\R{f}$ then $r \leq p \lthen q$ is witnessed
  by $\tpcalam{x}{\T{S}} \tpcalam{w}{\T{r}}
    \tpcalam{u}{\T{p}} \R{f}\;(\combPair\;w\;u)$.
\end{proof}

\section{Quantifiers}
\label{sec:quantifiers}

To get an interpretation of the quantifiers we apply methods of
categorical logic. In general the existential and universal
quantifiers are defined as left and right adjoints to weakening. Let
us explain what this means.

Weakening along the projection $\pi_1 : S \times T \to
S$ is an operation which maps a mono $V \monoto S$
to the mono $V \times T \monoto S \times T$.
This is a monotone map from $\Mono{S}$ to $\Mono{S \times
  T}$. Existential quantification is its left adjoint, i.e., a
monotone map $\exists_{T} : \Mono{S \times T} \to
\Mono{S}$ such that, for all monos $U \monoto S
\times T$ and $V \monoto S$,
%
\begin{equation*}
  U \leq V \times T
  \iff
  \exists_{T} U \leq V.
\end{equation*}
%
Similarly, universal quantification is the right adjoint:
%
\begin{equation*}
  V \times T \leq U
  \iff
  V \leq \forall_{T} U.
\end{equation*}
%
We verify that such adjoints really exist in $\AsmA$ but work with
realizability predicates instead of monos.

Weakening along $\pi_1 : S \times T \to S$ takes $p
\in \Pred{S}$ to the predicate $p \times T \in
\Pred{S \times T}$ defined by
%
\begin{equation*}
  \T{p \times T} = \T{p}
  \qquad\text{and}\qquad
  (p \times T)(x,y) = p(x),
\end{equation*}
%
where $x \in S$ and $y \in T$. We claim that the left adjoint
$\exists_{T}$ maps $q \in \Pred{S \times T}$ to
$\exists_{T} q \in \Pred{S}$ defined by
%
\begin{align*}
  \T{\exists_{T} q} &= \T{T} \times \T{q} \\
  (\exists_{T} q)(x) &=
  \set{ \combPair\;\R{y}\;\R{r} \in \Atyp{\T{T} \times \T{q}} \such
    \some{y}{T}{\R{y} \rz[T] y \land \R{r} \rz q(x,y)}
  }.
\end{align*}
%
Let us verify that we have a left adjoint to weakening. Suppose $p \in
\Pred{S \times T}$ and $q \in \Pred{S}$. If $p \leq
q \times T$ is witnessed by $\R{f}$ then $\exists_{T} p
\leq q$ is witnessed by $\tpcalam{x}{\T{S}} \tpcalam{e}{\T{T} \times
    \T{p}}{\R{f}\; (\combPair \; x \; (\combFst\; e)) \; (\combSnd \; e)}$.
Conversely, if $\exists_{T} p \leq q$ is witnessed by $\R{g}$
then $p \leq q \times T$ is witnessed by $\tpcalam{r}{\T{S}
    \times \T{T}} \tpcalam{s}{\T{p}}{\R{g} \;
  (\combFst\;r)\; (\combPair \; (\combSnd \; r) \; s)}$.

Similarly, the universal quantifier $\forall_{T}$ maps $q \in
\Pred{S \times T}$ to $\forall_{T} q \in
\Pred{S}$ defined by
%
\begin{align*}
  \T{\forall_{T} q} &= \T{T} \to \T{q} \\
  (\forall_{T} q)(x) &=
  \set{ \R{r} \in \Atyp{\T{T} \to \T{q}} \such
    \all{\R{y}}{\Atyp{\T{T}}}{
      \all{y \in T}
        (\R{y} \rz[T] y \implies
        \defined{\R{r}\;\R{y}} \land
        \R{r}\;\R{y} \rz q(x,y))
    }
  }.
\end{align*}
%
To verify that this is the right adjoint, suppose $p \in \Pred{S
  \times T}$ and $q \in \Pred{S}$. If $q \times T
\leq p$ is witnessed by $\R{f}$ then $q \leq \forall_{T} p$ is
witnessed by $\tpcalam{x}{\T{S}} \tpcalam{r}{\T{q}}
  \tpcalam{y}{\T{T}}{\R{f} \; (\combPair \; x \; y) \; r}$. Conversely,
if $q \leq \forall_{T} p$ is witnessed by $\R{g}$ then $q \times
T \leq p$ is witnessed by $\tpcalam{s}{\T{S} \times \T{T}}
  \tpcalam{r}{\T{q}}{\R{g} \; (\combFst \; s) \; r \; (\combSnd \; s)}$.

It is customary to write $\some{y}{T}{q(x,y)}$ and
$\all{y}{T}{q(x,y)}$ instead of $\exists_{T} q$ and
$\forall_{T} q$.

\begin{proposition}
  The operations $\exists_{T}$ and $\forall_{T}$
  respectively give a sound interpretation of the existential and
  universal quantifier with respect to first-order intuitionistic
  logic.
\end{proposition}

\begin{proof}
  This follows from the fact that $\exists_{T}$ and
  $\forall_{T}$ are the left and the right adjoint to weakening.
\end{proof}

\section{Substitution and the Beck-Chevalley condition}
\label{sec:beck-chevalley}

We have so far ignored the most basic logical operation of all, which
is \emph{substitution}. There are two kinds of substitution in
first-order logic: of terms into terms and of terms into formulas.

Substitution of terms into terms is interpreted as composition.
Suppose $s$ is a term of type $T$ with a free variable $x$ of
type $S$, and $t$ is a term of type $U$ with a free
varaible $y$ of type $U$. The terms $s$ and $t$ determine
realized maps
%
\begin{equation*}
  \xymatrix{
    {S}
    \ar[r]^{\sem{s}}
    &
    {T}
    \ar[r]^{\sem{t}}
    &
    {U}
  }
\end{equation*}
%
A fairly routine induction on the structure of~$t$ shows that
$\sem{\subst{t}{x \subto s}} = \sem{t} \circ \sem{s}$, i.e.,
substitution of terms into terms correponds to composition.
Simultaneous substitution of several variables is treated similarly.

Substition of terms into predicates is more interesting. To see what
operation it corresponds to, we first look at predicates in $\Set$.
Suppose $P \subseteq T$ is a subset of a set~$S$ and $t : S \to T$ is
a function. If we think of $P$ as a predicate $P(y)$ in variable $y
\in T$ and $t$ as a term $t(x)$ with a free variable $x \in S$ then
$P(t(x))$ should be a subset of $S$. A moment's thought shows that it
is the inverse image $\invim{t}(P) = \set{x \in S \such t(x) \in P}$.
The category-theoretic equivalent of inverse image is pullback.

Now let $p \in \Pred{T}$ be a realizability predicate, $\iota_p
: T_p \monoto T$ the corresponding mono, and $t : S
\to T$ a realized map. If we think of~$p$ as a formula $p(y)$
with variable $y : T$ and $t(x)$ as a term with a variable $x :
S$ then the substitution of $t(x)$ for $y$ gives us a formula
$p(t(x))$ with variable $x : S$. As in the case of sets,
$p(t(x))$ corresponds the pullback $\invim{t}(T_p) \monoto
S$, as in the diagram
%
\begin{equation*}
  \xymatrix{
    {\invim{t}(T_p)}
    \pbcorner
    \ar[r]
    \ar@{ >->}[d]
    &
    {T_p}
    \ar@{ >->}[d]
    \\
    {S}
    \ar[r]_{t}
    &
    {T}
  }
\end{equation*}
%
The realizability predicate $\invim{t}(p) \in \Pred{S}$ which
determined the mono $\invim{t}(T_p) \monoto S$ is given by
%
\begin{equation*}
  \T{\invim{t}(p)} = \T{p}
  \qquad\text{and}\qquad
  \invim{t}(p)(x) = p(t(x)).
\end{equation*}
%
\newcommand{\xtot}{\overline{x} \to \overline{t}}%
This defines substitution as an operation on realizability predicates.
However, substitution is also a syntactic transformation and we need
to check that the two notions fit together. Let $\xtot$ stand for a
simultaneous substitution $x_1 \subto t_1, \ldots, x_n \subto t_n$.
Substitution into first-order formulas is defined inductively on the
structure of the formula:
%
\begin{align*}
  \subst{\top}{\xtot} &= \top,\\
  \subst{\bot}{\xtot} &= \bot,\\
  \subst{(a = b)}{\xtot} &= \subst{a}{\xtot} = \subst{b}{\xtot},\\
  \subst{(\phi \land \psi)}{\xtot} &= 
  (\subst{\phi}{\xtot}) \land (\subst{\psi}{\xtot}), \\
  \subst{(\phi \lor \psi)}{\xtot} &= 
  (\subst{\phi}{\xtot}) \lor (\subst{\psi}{\xtot}), \\
  \subst{(\phi \implies \psi)}{\xtot} &= 
  (\subst{\phi}{\xtot}) \implies (\subst{\psi}{\xtot}), \\
  \subst{(\some{y}{T}{\phi})}{\xtot} &= 
  \some{y \in T} \subst{\phi}{\xtot},\\
  \subst{(\all{y}{T}{\phi})}{\xtot} &= 
  \all{y \in T} \subst{\phi}{\xtot}.
\end{align*}
%
In the last two cases we assume that $x_1, \ldots, x_n$ and the free
variables appearing in $t_1, \ldots, t_n$ are distinct from~$y$. The
above rules are sound for the realizability interpretation only if the
corresponding equations hold for realizability predicates and
substitution along a realized map $t$:
%
\begin{align*}
  \invim{t}(\top) &\equiv \top,\\
  \invim{t}(\bot) &\equiv \bot,\\
  \invim{t}(a = b) &\equiv (a \circ t = b \circ t),\\
  \invim{t}(p \land q) &\equiv \invim{t}(p) \land \invim{t}(q),\\
  \invim{t}(p \lor q) &\equiv \invim{t}(p) \lor \invim{t}(q),\\
  \invim{t}(p \implies q) &\equiv \invim{t}(p) \implies \invim{t}(q),\\
  \invim{t}(\exists_{T} p) &\equiv \exists_{T} \invim{t}(p),\\
  \invim{t}(\forall_{T} p) &\equiv \forall_{T} \invim{t}(p).
\end{align*}
%
Pullback is a finite limit, hence it commutes with those operations
that are given as finite limits, namely $\top$, $=$ and $\land$. The
remaining equations must be checked by hand. [NOT FINISHED.]


\section{Validity in the realizability interpretation}
\label{sec:realizability-validity}

Let us summarize the realizability interpretation of first-order
intuitionistic logic. For each formula $\phi$ we define its underlying
type $\T{\phi}$ as follows:
%
\begin{align*}
  \T{\bot} &= \unit \\
  \T{\top} &= \unit \\
  \T{t_1 = t_2} &= \unit \\
  \T{\phi_1 \land \phi_2} &= \T{\phi_1} \times \T{\phi_2}\\
  \T{\phi_1 \lor \phi_2} &= \T{\phi_1} + \T{\phi_2}\\
  \T{\phi_1 \implies \phi_2} &= \T{\phi_1} \to \T{\phi_2}\\
  \T{\all{x}{T}{\phi}} &= \T{T} \to \T{\phi} \\
  \T{\some{x}{T}{\phi}} &= \T{T} \times \T{\phi}.
\end{align*}
%
The realizability relation $\R{r} \rz \phi$, where $\R{r} \in
\Atyp{\T{\phi}}$ is defined inductively as follows:
%
\begin{align*}
  \R{r} \rz \bot &\iff \bot\\
  \R{r} \rz \top &\iff \top\\
  \ttunit \rz t_1 = t_2 &\iff t_1 = t_2\\
  \combPair\;\R{r}_1\;\R{r}_2 \rz \phi_1 \land \phi_2 &\iff
  \R{r}_1 \rz \phi_1 \land \R{r}_2 \rz \phi_2\\
  \combLeft\;\R{r} \rz \phi_1 \lor \phi_2 &\iff \R{r} \rz \phi_1\\
  \combRight\;\R{r} \rz \phi_1 \lor \phi_2 &\iff \R{r} \rz \phi_2\\
  \R{r} \rz \phi_1 \implies \phi_2 &\iff
  \all{\R{s} \in \Atyp{\T{\phi_1}}} (\R{s} \rz \phi_1 \implies
    \defined{\R{r}\;\R{s}} \land \R{r}\;\R{s} \rz \phi_2)
  \\
  \combPair\;\R{u}\;\R{r} \rz \some{x}{T}{\phi} &\iff
  \some{u \in T} \R{u} \rz[T] u \land \R{r} \rz \subst{\phi}{x \subto u}
  \\
  \R{r} \rz \all{x}{T}{\phi} &\iff
  \all{\R{u}}{\Atyp{\T{T}}}{
    \all{u \in T}
      (\R{u} \rz[T] u \implies
      \defined{\R{r}\;\R{u}} \land \R{r}\;\R{u} \rz \subst{\phi}{x \subto u})
    }.
\end{align*}
%

\section{Realizability toposes}
\label{sec:realizability-toposes}


\section[\texorpdfstring{$\neg\neg$-stable predicates}{Not-not-stable predicates}]{$\neg\neg$-stable predicates}
\label{sec:decidable-predicates}

\subsection{Excluded middle}
\label{sec:excluded-middle}

\subsection{Decidable predicates}
\label{sec:decidable-predicates-1}

\subsection{Semidecidable predicates}
\label{sec:semid-pred}

\subsection{Stability of equality}
\label{sec:stability-equality}

\subsection{Negative and almost negative formulas}
\label{sec:negat-almost-negat}



%%% Local Variables: 
%%% mode: latex
%%% TeX-master: "notes-on-realizability"
%%% End: 
