\chapter{Realizability and logic}
\label{cha:realizability-logic}

The idea that the elements of a set are represented by values of a datatype is familiar to programmers. In the previous chapter we expressed the idea mathematically in terms of realizability relations and assemblies. Programmers are less aware of, but still use, the fact that realizability carries over to logic as well: a logical statement can be validated by realizers.

\section{The set-theoretic interpretation of logic}
\label{sec:interpr-logic-set-theoretic}

Let us first recall how the usual interpretation of classical
first-order logic works.
%
A \emph{predicate} on a set~$S$ is a Boolean function $S \to \two$, where $\two = \set{\bot, \top}$ is the Boolean algebra on two elements. The Boolean algebra structure carries over from~$\two$ to predicates, e.g., the conjunction of $p, q : S \to \two$ is computed element-wise as
%
\begin{equation*}
  (p \land q) x = p x \land q x,
\end{equation*}
%
and similarly for other predicates. With this much structure we can interpret the propositional calculus. The quantifiers $\exists$ and $\forall$ can be interpreted too, because~$\two$ is complete: given a predicate $p : S \times T \to \two$, define $\exists_S p : T \to \two$ and $\forall_S p : T \to \two$ by
%
\begin{equation*}
  (\exists_S p) y = \bigvee\nolimits_{x \in S} p(x, y)
  \qquad\text{and}\qquad
  (\forall_S p) y = \bigwedge\nolimits_{x \in S} p(x, y).
\end{equation*}
%
Categorical logic teaches us that the essential characteristic of the quantifiers is not their relation to infima and suprema, but rather an \emph{adjunction}: $\exists_S p$ is the least predicate on~$T$ such that $p(x, y) \leq (\exists_S p) y$ for all $x \in S$, $y \in T$, and $\forall_S p$ is the largest predicate on~$T$ such that $(\forall_S p) y \leq p(x, y)$ for all $x \in S$, $y \in T$. The adjunction will carry over to the realizability logic, but completeness will not.


\section{Realizability predicates}
\label{sec:realizability-predicates}

In a category a predicate on an object $S$ is represented by a mono with codomain~$S$. These form a preorder, where $u : T \monoto S$ is below $t : U \monoto S$, written $u \leq t$ or somewhat less precisely $U \leq T$, whhen $u$ factors through $t$, i.e., there exists a morphism $f : U \to T$ such that
%
\begin{equation*}
  \xymatrix{
    &
    {S}
    &
    \\
    {U}
    \ar@{ >->}[ur]^{u}
    \ar[rr]_{f}
    &
    &
    {T}
    \ar[ul]_{t}
  }
\end{equation*}
%
commutes. Such an~$f$ is unique if it exists and is a mono.\sidenote{Such basic category-theoretic observations are excellent exercises. You should prove them yourself.} If $u \leq t$ and $t \leq u$ we say that $u$ and $t$ are \emph{isomorphic} and write $u \equiv t$. The induced partial order $\Sub{S} = \Mono{S}/{\cong}$ of \emph{subobjects} can be used if one cares about antisymmetry, which we do not.

In the case of assemblies $\Mono{S}$ forms a \emph{Heyting prealgebra}, which is enough to interpret \emph{intuitionistic} propositional calculus, and there is enough additional structure to interpret the quantifiers too.
%

However, rather than working with the Heyting prealgebra of monos, we shall replace it with an equivalent one that expresses the predicates as maps into Heyting prelagebras.

\begin{definition}
  A \defemph{realizability predicate} on an assembly $S$ is given by a type $\T{p}$ and a
  map $p : \S{S} \to \pow{\Atyp{\T{p}}}$.
\end{definition}

It is customary to write $\R{r} \rz p x$ instead of $\R{r} \in p x$ and to read this as ``$\R{r}$ realizes $p x$''. Realizability predicates are more informative than Boolean predicates. The latter only express truth and falshood, whereas the former provide \emph{computatational evidence} for validity of statements.

The set $\Pred{S} = \pow{\Atyp{\T{p}}}^{\S{S}}$ of all realizability predicates on~$S$ is a preorder for the \defemph{entailment} relation~$\entails$, defined as follows. Given  $p, q \in \Pred{S}$, we define $p \entails q$ to hold when there exists $\R{i} \in \subAtyp{\T{S} \to \T{p} \to \T{q}}$ such that whenever $\R{x} \rz[S] x$ and $\R{r} \rz p x$ then $\defined{\R{i}\,\R{x}\,\R{r}}$ and $\R{i}\,\R{x}\,\R{r} \rz q x$.
Thus, $\R{i}$ converts computational evidence of $p x$ to computational evidence of~$q x$. Note that $\R{i}$ recieves as input both a realizer for~$x$ and the evidence of~$p x$.

\begin{exercise}
  Verify that $\entails$ is reflexive and transitive.
\end{exercise}

\begin{theorem}
  \label{thm:mono-equiv-pred}%
  The preorders $\Mono{S}$ and $\Pred{S}$ are equivalent.
\end{theorem}

\begin{proof}
  Given $p \in \Pred{S}$, define the assembly $S_p$ by
  %
  \begin{align*}
    \S{S_p} &= \set{x \in \S{S} \such \some{\R{r} \in \Atyp{\T{p}}}{\R{r} \in p x}}, \\
    \T{S_p} &= \T{S} \times \T{p}, \\
    q \rz[S_p] x &\iff \combFst \, q \rz[S] x \land \combSnd\,q \rz p x.
  \end{align*}
  %
  The subset inclusion $\S{S_p} \subseteq \S{S}$ is realized by $\combFst$.
  The corresponding assembly map $i_p : S_p \to S$ is called the \defemph{extension} of~$p$.

  It is easy to check that $p \entails q$ is equuvalent to $i_p \leq i_q$, therefore the assignment $p \mapsto S_p$ consistutes a monotone embedding $\Pred{S} \to \Mono{S}$.
  We still have to show that it is essentially surjective, i.e., that every mono
  $u : U \monoto S$, realized by $\R{u}$, is equivalent to the extension of a predicate.
  %
  Define the predicate~$p_u$ on~$S$ by $\T{p_u} = \T{U}$ and
  %
  \begin{equation*}
    \R{r} \rz p_u \, x
    \iff
    \some{y \in U} u\,y = x \land \R{y} \rz[U] y.
  \end{equation*}
  %
  We claim that~$i_{p_u}$ and~$u$ are isomorphic monos.
  %
  The injection $u : \S{U} \to \S{S}$ restricts to a
  bijection $u : \S{U} \to \S{S_{p_u}}$, which is realized as a morphism $U \to
  S_{p_u}$ by
  $\tpcalam{y}{\T{U}} \combPair\,(\R{u}\,y)\,y$. Its inverse
  $\inv{u} : \S{S_{p_u}} \to \S{U}$ is realized by $\combSnd$.
\end{proof}


\section{The Heyting prealgebra of realizability predicates}
\label{sec:heyting-prealgebra}

Recall that a \defemph{Heyting prealgebra} $(H, {\entails})$ is a preorder (reflexive and transitive) with elements $\bot$, $\top$ and binary operations $\land$, $\lor$, and $\lthen$ governed by the following rules of inference:\sidenote{These ``fractions'' are inference rules stating that the bottom statement follows from the conjunction of the top ones. The double line indicates a two-way rule which additionally states that the bottom statement implies the conjunction of the top ones.}
%
\begin{mathpar}
  \inferrule{ }{\bot \entails p}

  \inferrule{ }{p \entails \top}

  \mprset{fraction={===}}
  \inferrule
  { r \entails p \\ r \entails q}
  { r \entails p \land q }

  \inferrule
  { p \entails r \\ q \entails r}
  { p \lor q \entails r }

  \inferrule
  { r \land p \entails q}
  { r \entails p \Rightarrow q }
\end{mathpar}
%
We say that elements $p, q \in H$ are \defemph{equivalent}, written $p \bientails q$, if
$p \entails q$ and $q \entails p$. A \defemph{Heyting algebra} is a Heyting prealgebra in which equivalence is equality, which just means that the preorder is also antisymmetric.

In a Heyting prealgebra there may be many smallest elements, but they are all equivalent to~$\bot$. Similarly, the binary infima and suprema may exist in many copies, all of which are equivalent.


\begin{proposition}
  The preorder $\Pred{S}$ is a Heyting prealgebra.
\end{proposition}

\begin{proof}
  Define the predicates $\bot, \top : S \to \pow{\Atyp{\unit}}$ on an assembly~$S$ by
  % 
  \begin{equation*}
    \bot x= \emptyset
    \qquad\text{and}\qquad
    \top x = \Atyp{\unit}.
  \end{equation*}
  % 
  That is, $\bot$ is realized by nothing and $\top$ by everything. It is
  easy to check that $\bot \entails p \entails \top$ for all $p \in
  \Pred{S}$.

  For predicates $p$ and $q$ on $S$, let $p \land q$ be the
  predicate with $\T{p \land q} = \T{p} \times
  \T{q}$ and
  %
  \begin{equation*}
    \R{r} \rz (p \land q) \, x
    \iff
    \combFst\,\R{r} \rz p \land \combSnd\,\R{r} \rz q.
  \end{equation*}
  %
  It is customary to write $p \, x \land q \, x$ instead of $(p \land q) \, x$, which we shall do henceforth, including for other connectives.
  %
  Let us verify that $p \land q$ is the infimum of $p$ and $q$. If $r \entails p$ and
  $r \entails q$ are witnessed by $\R{a}$ and $\R{b}$, respectively, then $r \entails p \land q$
  is witnessed by $\tpcalam{x}{\T{S}} \tpcalam{u}{\T{r}} \combPair\,(\R{a}\,x\,u)\,(\R{b}\,x\,u)$.
  %
  Conversely, if $\R{c}$ witnesses $r \entails p \land q$ then
  $\tpcalam{x}{\T{S}} \tpcalam{u}{\T{r}} \combFst\,(\R{c}\,x\,u)$ witnesses $r \entails p$ and $\tpcalam{x}{\T{S}} \tpcalam{u}{\T{r}} \combSnd\,(\R{c}\,x\,u)$ witnesses $r \entails q$.

  Next, define $p \lor q$ to be the predicate with $\T{p \land q} = \T{p} + \T{q}$ and
  %
  \begin{equation*}
    \R{r} \rz p x \lor q x
    \iff
    \begin{aligned}[t]
      & (\some{\R{u}} \R{r} = \combLeft\,\R{u} \land \R{u} \rz p x)
      \lor {} \\
      & (\some{\R{v}} \R{r} = \combRight\,\R{v} \land \R{v} \rz q x).
    \end{aligned}
  \end{equation*}
  %
  If $\R{a}$ and $\R{b}$ witness $p \entails r$ and $q \entails r$, respectively, then
  $p \lor q \entails r$ is witnessed by
  %
  \begin{equation*}
    \tpcalam{x}{\T{S}} \tpcalam{w}{\T{p}+\T{q}}{
      \combCase\,w\,(\R{a}\,x)\,(\R{b}\,x)
    }.
  \end{equation*}
  %
  Conversely, if $\R{c}$ witnesses $p \lor q \entails r$ then
  $p \entails r$ and $q \entails r$ are witnessed by
  $\tpcalam{x}{\T{S}} \tpcalam{u}{\T{p}}{\R{c}\,x\,(\combLeft\,u})$
  and
  $\tpcalam{x}{\T{S}} \tpcalam{v}{\T{q}}{\R{c}\,x\,(\combRight\,v})$,
  respectively.

  Finally, define $p \lthen q$ to be the predicate with $\T{p \lthen q} = \T{p} \to \T{q}$ and
  %
  \begin{equation*}
    \R{r} \rz (p x \lthen q x)
    \iff
      \all{\R{u} \in \Atyp{\T{p}}}
        \R{u} \rz p x
        \lthen
        \R{r}\,\R{u} \rz q x.
  \end{equation*}
  %
  That is, $\R{r}$ maps realizers for $p x$ to realizers for $q x$.
  %
  Note that $\R{r} \in \Atyp{\T{p} \to \T{q}}$ and \emph{not} $\R{u} \in \subAtyp{\T{p} \to \T{q}}$, so we have an example of entailment~$\entails$ and implication~$\lthen$ not ``being the same thing'' (about which students of logic often wonder about).
\end{proof}

\begin{exercise}
  Finish the proof by checking that the definition of $\lthen$ validates the inference rules for implication.
\end{exercise}

In intuitionistic logic \defemph{negation} $\neg p$ is an abbreviation for $p \lthen \bot$. The following holds:
%
\begin{align*}
  r \rz \neg p x
  &\iff
  \lnot \some{\R{s} \in \Atyp{\T{p}}} \R{s} \rz p x, \\
  r \rz \neg \neg p x
  &\iff
  \some{\R{s} \in \Atyp{\T{p}}} \R{s} \rz p x.
\end{align*}


\section{Quantifiers}
\label{sec:quantifiers}

Categorical logic teaches us that the existential and universal quantifiers are defined as the left and right adjoints to weakening. Let us explain what this means.

Weakening along the projection $\pi_1 : S \times T \to S$ is an operation which maps a
mono $V \monoto S$ to the mono $V \times T \monoto S \times T$. This is a monotone map
from $\Mono{S}$ to $\Mono{S \times T}$. Existential quantification is its left adjoint,
i.e., a monotone map $\exists_{T} : \Mono{S \times T} \to \Mono{S}$ such that, for all
monos $U \monoto S \times T$ and $V \monoto S$,
%
\begin{equation*}
  U \leq V \times T
  \iff
  \exists_{T} U \leq V.
\end{equation*}
%
Similarly, universal quantification is the right adjoint:
%
\begin{equation*}
  V \times T \leq U
  \iff
  V \leq \forall_{T} U.
\end{equation*}
%
From the above adjunctions the usual laws of inference for the existential and universal quantifiers follow. We verify that such adjoints for the Heyting prealgebras of realizability predicates.

Weakening along $\pi_1 : S \times T \to S$ takes $p \in \Pred{S}$ to the predicate
$p \times T \in \Pred{S \times T}$ defined by
%
\begin{equation*}
  \T{p \times T} = \T{p}
  \qquad\text{and}\qquad
  (p \times T)(x,y) = p x,
\end{equation*}
%
where $x \in S$ and $y \in T$. We claim that the left adjoint
$\exists_{T}$ maps $q \in \Pred{S \times T}$ to
$\exists_{T} q \in \Pred{S}$ with
%
$\T{\exists_{T} q} = \T{T} \times \T{q}$
%
and
%
\begin{equation*}
  \R{r} \rz (\exists_{T} q)(x)
  \iff
  \some{y \in \S{T}}
    \combFst\,R{r} \rz[T] y \land \combSnd\,\R{r} \rz q(x,y).
\end{equation*}
%
Let us verify that we have a left adjoint to weakening. Suppose $p \in
\Pred{S \times T}$ and $q \in \Pred{S}$. If $p \leq
q \times T$ is witnessed by $\R{f}$ then $\exists_{T} p
\leq q$ is witnessed by $\tpcalam{x}{\T{S}} \tpcalam{e}{\T{T} \times
    \T{p}}{\R{f}\, (\combPair \, x \, (\combFst\, e)) \, (\combSnd \, e)}$.
Conversely, if $\exists_{T} p \leq q$ is witnessed by $\R{g}$
then $p \leq q \times T$ is witnessed by $\tpcalam{r}{\T{S}
    \times \T{T}} \tpcalam{s}{\T{p}}{\R{g} \,
  (\combFst\,r)\, (\combPair \, (\combSnd \, r) \, s)}$.

Similarly, the universal quantifier $\forall_{T}$ maps $q \in
\Pred{S \times T}$ to $\forall_{T} q \in
\Pred{S}$ with $\T{\forall_{T} q} = \T{T} \to \T{q}$ and
%
\begin{align*}
  \R{r} \rz (\forall_{T} q)(x)
  \iff
    \all{y \in \S{T}} \all{\R{y} \in \Atyp{\T{T}}}
     \R{y} \rz[T] y \lthen \R{r}\,\R{y} \rz q(x,y).
\end{align*}
%
To verify that this is the right adjoint, suppose $p \in \Pred{S
  \times T}$ and $q \in \Pred{S}$. If $q \times T
\leq p$ is witnessed by $\R{f}$ then $q \leq \forall_{T} p$ is
witnessed by $\tpcalam{x}{\T{S}} \tpcalam{r}{\T{q}}
  \tpcalam{y}{\T{T}}{\R{f} \, (\combPair \, x \, y) \, r}$. Conversely,
if $q \leq \forall_{T} p$ is witnessed by $\R{g}$ then $q \times
T \leq p$ is witnessed by $\tpcalam{s}{\T{S} \times \T{T}}
  \tpcalam{r}{\T{q}}{\R{g} \, (\combFst \, s) \, r \, (\combSnd \, s)}$.

It is customary to write $\some{y \in T}{q(x,y)}$ and
$\all{y \in T}{q(x,y)}$ instead of $\exists_{T} q$ and
$\forall_{T} q$.

\begin{exercise}
  Show that the usual inference rules for quantifiers follow form them being adjoint to weakneing.
\end{exercise}


\section{Substitution}
\label{sec:beck-chevalley}

We have so far ignored the most basic logical operation of all, which
is \emph{substitution}. Terms may be substituted into terms and into formulas.


Substitution of terms into terms is interpreted as composition.
Think of a term $s(x)$ of type~$S$ with a free variable~$x$ of type~$T$ as a map $s : T \to S$. Given $t : U \to T$, construed as a term $t(y)$ of type $T$ with a free variable~$y$ of type~$U$, the substitution of $t(y)$ for $x$ in $s(x)$ yields $s(t(y))$, which is just $(s \circ t)(y)$, so $s \circ t : U \to S$.

Substitution of terms into predicates corresponds to composition too, by an analogous argument. Think about how it works in $\Set$. Given a predicate $p : T \to \two$ on a set~$S$ and a term $t : S \to T$, the composition $p \circ t : S \to \two$ corresponds to substituting~$t$ into~$p$. Indeed, if we replace $y$ with $t(x)$ in the formula $p(x)$ we obtain $p (t(y))$, which is just $(p \circ t)(y)$.

\begin{exercise}
  You may have heard the slogan ``substitution is pullback''.
  Explain how the slogan arises when predicates are viewed as subsets, rather than maps into~$\two$
\end{exercise}

Let us introduce a notation for substitution. Given assembly maps $t : U \to T$ and $s : T \to S$, and $p \in \Pred{T}$, define the \defemph{substitution} of~$t$ into~$s$ and $p$ to be precomposition with~$t$:
%
\begin{equation*}
 \invim{t} s = s \circ t : U \to S
 \qquad\text{and}\qquad
 \invim{t} p = p \circ t \in \Pred{U}.
\end{equation*}
%
We still have some work to do, namely check that substitution is functorial and that it commutes with the logical connective and the quantifiers. The former guarantees that the identity substitution and compositions of substitutions act in the expected way, and the latter that substituting preserves the logical structure of a formula. Functoriality means that
%
\begin{equation*}
  \invim{\id[S]} s = s
  \qquad\text{and}\qquad
  \invim{(u \circ t)} s = \invim{t} (\invim{u} s),
\end{equation*}
%
which is of course the case.\sidenote{Substitution is \emph{contravariant} because it reverses the order of composition.}

\begin{proposition}
  Realizability predicates and substitution constitute a functor
  %
  \begin{equation*}
    \opcat{\AsmA} \to \mathsf{Heyt}
  \end{equation*}
  %
  from (the opposite category of) assemblies to the category of Heyting prealgebras.
  Moreover, substitution preserves the quantifiers.
\end{proposition}

\begin{proof}
  We already checked that substitution is functorial, but we still need to verify
  that it yields a homomorphism of Heyting prealgebras, i.e.,
  %
  \begin{align*}
    \invim{t} \top &= \top,\\
    \invim{t} \bot &= \bot,\\
    \invim{t} (p \land q) &= \invim{t} p \land \invim{t} q,\\
    \invim{t} (p \lor q) &= \invim{t} p \lor \invim{t} q,\\
    \invim{t} (p \lthen q) &= \invim{t} p \lthen \invim{t} q.
  \end{align*}
  % 
  These hold because they are defined pointwise. For instance, given $p, q \in \Pred{T}$, define $\land' : \pow{\Atyp{\T{p}}} \times \pow{\Atyp{\T{q}}} \to \pow{\Atyp{\T{p} \times \T{q}}}$ by
  % 
  \begin{equation*}
    A \land' B =
    \set{ \R{r} \in \Atyp{\T{p} \times \T{q}} \such
      \combFst\,\R{r} \in A \land \combSnd\,\R{r} \in B },
  \end{equation*}
  % 
  and observe that $p \land q = \land' \circ \pair{p, q}$ therefore, for any $t : U \to T$,
  % 
  \begin{equation*}
    \invim{t}(p \land q) = \land' \circ \pair{p, q} \circ t
    = \land' \circ \pair{p \circ t, q \circ t} = \invim{t} p \land \invim{t} q.
  \end{equation*}
  % 
  The other connectives are dealt with analogously.

  Preservation of quantifier amounts to checking the \defemph{Beck-Chevalley conditions}:
  %
  \begin{align*}
    \invim{t} (\exists_{T} p) &= \exists_{T} (\invim{t} p),\\
    \invim{t} (\forall_{T} p) &= \forall_{T} (\invim{t} p). \qedhere
  \end{align*}
\end{proof}

\begin{exercise}
  Verify the Beck-Chevalley conditions for $\forall_T$ and $\exists_T$.
\end{exercise}

\section{Equality}
\label{sec:equality}

Equality qua binary predicate on an object $S$ is represented by the diagonal morphism $\Delta : S \to S \times S$. In assemblies this is the map $\Delta x = (x, x)$, which is realized by $\tpcalam{x}{\T{S}} \combPair\,x\,x$.
%
We claim that the corresponding realizability predicate $\mathrm{eq} \in \Pred{S \times S}$ is given by $\T{e} = \unit$ and
%
\begin{equation*}
  \star \rz \mathrm{eq} (x, y)
  \iff
  x = y.
\end{equation*}
%
To verify the claim, we need to factor $\Delta$ and $i_\mathrm{eq} : S_\mathrm{eq} \to S$ from \cref{thm:mono-equiv-pred} through each other. We have
%
\begin{align*}
  \S{S_\mathrm{eq}} &= \set{(x, y) \in \S{S} \times \S{S} \mid x = y},
  \\
  i_\mathrm{eq} (x,y) &= (x, y),
  \\
  \T{S_\mathrm{eq}} &= \T{S} \times \T{S} \times \unit,
  \\
  q \rz[S_\mathrm{eq}] x &\liff \combFst q \rz[S] x \land \combSnd\,q = \star.
\end{align*}
%
Thus $\Delta$ factors through $i_\mathrm{eq}$ via $x \mapsto (x,x)$ and $i_\mathrm{eq}$ through $\Delta$ via $(x, y) \mapsto x$, each of which is easily seen to be realized.

\section{Summary of realizability logic}
\label{sec:realizability-summary}

We summarize the realizability interpretation of intuitionistic logic for easy lookup.

A realizability predicate~$p$ on an assembly $S$ is given by a type $\T{p}$ and a map $p : \S{S} \to \pow{\T{p}}$. The collection of all realizability predicates on~$S$ is denoted $\Pred{S}$. Given $p, q \in \Pred{S}$, $p$ entails $q$, written $p \entails q$, when there is $\R{i} \in \subAtyp{\T{S} \to \T{p} \to \T{q}}$ such that
%
\begin{equation*}
  \all{x \in \S{S}}
  \all{\R{x} \in \T{S}}
  \all{\R{r} \in \T{p}}
  \R{r} \rz[S] x
  \land
  \R{r} \rz p
  \lthen
  \R{i}\,\R{x}\,\R{r} \rz q x.
\end{equation*}
%
The entailment relation is a preorder.

The underlying types of connectives are computed as follows, where $p, q \in \Pred{S}$ and $r \in \Pred{T \times S}$:
%
\begin{align*}
  \T{\bot} &= \unit, \\
  \T{\top} &= \unit, \\
  \T{s = t} &= \unit, \\
  \T{p \land q} &= \T{p} \times \T{q},\\
  \T{p \lor q} &= \T{p} + \T{q},\\
  \T{p \lthen q} &= \T{p} \to \T{q}.\\
  \T{\all{x \in T}{r(x, {-})}} &= \T{T} \to \T{r}, \\
  \T{\some{x \in T}{r(x, {-})}} &= \T{T} \times \T{r}.
\end{align*}
%
We write $\R{r} \rz p x$ instead of $\R{r} \in p x$.
The Heyting prealgebra structure on $\Pred{S}$ is as follows, where $p, q \in \Pred{S}$,
$r \in \Pred{T \times S}$, and $s, t \in \S{S}$:
%
\begin{align*}
  \R{r} \rz \bot &\liff \bot\\
  \R{r} \rz \top &\liff \top\\
  \R{r} \rz s = t &\liff s = t \land \R{r} = \star\\
  \R{r} \rz p x \land q x &\liff
  \combFst\,\R{r} \rz p \land \combFst\,\R{r} \rz q\\
  \R{r} \rz p x \lor q x
    &\liff
    \begin{aligned}[t]
      & (\some{\R{u}} \R{r} = \combLeft\,\R{u} \land \R{u} \rz p x)
      \lor {} \\
      &(\some{\R{v}} \R{r} = \combRight\,\R{v} \land \R{v} \rz q x).
    \end{aligned}
  \\
  \R{r} \rz (p \lthen q) &\liff
  \all{\R{s} \in \Atyp{\T{p}}} \R{s} \rz p \lthen \R{r}\,\R{s} \rz q
  \\
  \R{r} \rz \some{y \in T} p(x,y) &\liff
  \some{y \in \S{T}} \combFst\,\R{r} \rz[T] y \land \combSnd\,\R{r} \rz p(x,y)
  \\
  \R{r} \rz \all{y \in T} p(x, y) &\liff
    \all{y \in \S{T}} \all{\R{y}}
     \R{y} \rz[T] y \lthen \R{r}\,\R{y} \rz q(x,y).
\end{align*}
%
We have the following soundness theorem.

\begin{theorem}
  The realizability interpretation of intuitionistic logic is sound.
\end{theorem}

\begin{proof}
  We already proved the theorem when we checked that the predicates form Heyting prealgebras and that the quantifiers validate the desired inference rules.
\end{proof}
 
Let us spell out what the theorem says: if a formula $\phi$ is intuitionistically
provable then there exists $\R{r} \in \subAtyp{\T{\phi}}$ such that $\R{r} \rz \phi$.
Moreover, the realizer~$\R{r}$ can be constructed from the proof of~$\phi$, one just has
to follow the inference steps and apply the corresponding realizability constructions.

The above clauses are reminiscent of the Curry-Howard correspondence between logic and type theory. The similarity is not accidental, as both realizability and Martin-Löf type theory aim to formalize the Brouwer-Heyting-Kolmogorov explanation of intuitionistic logic. Let us not forget, however, that realizability was the first such rigorous explanation.


\section{Classical and decidable predicates}
\label{sec:special-predicates}

Some classes of predicates are of special interest.

\subsection[\texorpdfstring{$\neg\neg$-stable predicates}{Not-not-stable predicates}]{Classical predicates}
\label{sec:stable-predicates}

A formula $\phi$ is \defemph{classical}\sidenote{The terminology is non-standard but is vindicated below by the fact that $\nabla\two$ classifies the classical predicates. In any case, almost any terminology seems better than ``$\lnot\lnot$-stable''.} or \defemph{$\lnot\lnot$-stable} when $\lnot\lnot \phi \lthen \phi$. (We do not require $\phi \lthen \lnot\lnot \phi$ because it holds anyway.)

\begin{proposition}
  \label{prop:when-stable}
  A predicate $p \in \Pred{S}$ is classical if, and only if, there exists $\R{q} \in \subAtyp{\T{S} \to \T{p}}$ such that, if $\R{x} \rz[S] x$ and  $p x \neq \emptyset$ then $\R{q}\,\R{x} \rz p x$.
\end{proposition}

\begin{exercise}
  Prove \cref{prop:when-stable}. More precisely, you need to show that the stated condition is equivalent to $\top \entails \lnot\lnot p \lthen p$.
\end{exercise}

Thus a predicate is classical when its realizers can be computed at will, when they exist.
%
An important fact about assemblies is the following.

\begin{proposition}
  Equality on an assembly is classical.
\end{proposition}

\begin{proof}
  Apply \cref{prop:when-stable} with $\R{r} = \combK\,\ttunit$.
\end{proof}

It is useful to have a simple syntactic criterion for recognizing classical formulas. Say that a formula is \defemph{almost negative}\sidenote{A \defemph{negative} formula is like an almost negative one, except that the antecedent in an implication is required to be negative, too.} when it has one of the following forms:
%
\begin{itemize}
\item $\bot$, $\top$, $s = t$,
\item $\phi \land \psi$ where $\phi$ and $\psi$ are almost negative,
\item $\all{x \in S} \phi$ where $\phi$ is almost negative,
\item $\phi \lthen \psi$ where $\psi$ is almost negative.
\end{itemize}

\begin{proposition}
  \label{prop:almost-negative-stable}
  An almost negative formula is classical.
\end{proposition}

\begin{exercise}
  Prove \cref{prop:almost-negative-stable} \emph{without} resorting to realizability. You should just give suitable intuitionistic proofs. For extra credit, do it in a proof assistant.
\end{exercise}

The classical predicates can be used to interpret classical logic.

\begin{exercise}
  Consider the following preorders:
  %
  \begin{itemize}
  \item the sub-preorder $\mathsf{RegMono}(S) \subseteq \Mono{S}$ on the \emph{regular} monos,
  \item the sub-preorder $\mathsf{Pred}_{\lnot\lnot}(S) \subseteq \Pred{S}$ on the classical predicates,
  \item the powerset $\pow{\S{S}}$ ordered by~$\leq$.
  \end{itemize}
  %
  Show that these preorders are equivalent, and hence complete Boolean algebras.
  For extra credit, explain in what sense the equivalence is \emph{functorial}.
\end{exercise}


\subsection{Decidable predicates}
\label{sec:decidable-predicates}

Recall that a formula $\phi$ is \defemph{decidable}\sidenote{
  Every decidable statement is classical, but the converse cannot be shown intuitionistically.
  %
  If you are confused, you might be thinking of the equivalence between
  excluded middle (``all predicates are decidable'') and double-negation elimination (``all predicates are $\lnot\lnot$-stable'').
} when $\phi \lor \lnot \phi$.

\begin{proposition}
  \label{prop:decidable-when}
  A predicate $p \in \Pred{S}$ is decidable if, and only if, there exists
  $\R{d} \in \subAtyp{\T{S} \to \ttbool}$ such that, for all $x \in \S{S}$ and $\R{x} \in \T{S}$, if $\R{x} \rz[S] x$ then $\defined{\R{d}\,\R{x}}$ and
  %
  \begin{equation*}
    \R{d}\,\R{x} =
    \begin{cases}
      \combTrue & \text{if $p x \neq \emptyset$,} \\
      \combFalse & \text{if $p x = \emptyset$.}
    \end{cases}
  \end{equation*}
\end{proposition}

\begin{proof}
  If $\R{r} \rz p \lor \lnot p$ then we may take
  %
  \begin{equation*}
    \R{d} = \tpcalam{\R{x}}{\T{S}} \combCase\,(\R{r}\,\R{x})\,\combTrue\,\combFalse.
  \end{equation*}
  %
  Conversely, suppose $\R{d}$ with the stated property exists. Because a decidable predicate is stable, by \cref{prop:when-stable} there exists $\R{q} \in \subAtyp{\T{S} \to \T{p}}$ such that, if $\R{x} \rz[S] x$ and $p x \neq \emptyset$ then $\R{r}\,\R{x} \rz p x$. Now the realizer
  %
  \begin{equation*}
    \tpcalam{\_}{\unit}
    \tpcalam{\R{x}}{\T{S}}
      \combIf\,
        (\R{d}\,\R{x})\,
        (\combLeft (\R{q}\,\R{x}))\,
        (\combRight (\tpcalam{\_}{\T{S}}{\combK}))
  \end{equation*}
  %
  witnesses $\top \entails p \lor \lnot p$.
\end{proof}

The realizer $\R{d}$ from \cref{prop:decidable-when} can be thought of as a decision procedure for~$p$, although what that means depends on the underlying model of computation. If~$\AA$ retracts to~$\klone$, see \cref{sec:decidable-simulations}, then~$\R{d}$ does indeed correspond to an algorithm in the usual sense of the word. But a topological model of computation, such as the graph model or Kleene's second algebra, $\R{d}$ amounts to having a \emph{topological separation} of the underlying space into two disjoint open sets.


\subsection{Predicates classified by two-element assemblies}
\label{sec:predicates-classified}

Consider a two-element assembly~$T$ with $\S{T} = \set{0,1}$, like we did in \cref{sec:two-elem-asm}. Say that a predicate $p \in \Pred{S}$ is \defemph{classified} by an assembly map $c : S \to T$, called the \defemph{characteristic map} of~$p$, when
%
\begin{equation*}
  \top \entails \all{x \in S} p x \liff c x = 1
\end{equation*}
%
is realized. Because the above is a negative formula, it is $\neg\neg$-stable and so its realizers are uninformative. In other words, the statement is realized if, and only if,
%
\begin{equation*}
  \all{x \in \S{S}} p x \neq \emptyset \liff c x = 1.
\end{equation*}

\begin{exercise}
  Show that a predicate has at most one characteristic map.
\end{exercise}

For a specific choices of two-element assemblies we obtain known classes of predicates.

\begin{proposition}
  $\nabla \two$ and $\Bool$ classify the classical and decidable predicates, respectively.
\end{proposition}

\begin{proof}
  Let us show that $\nabla \two$ classifies the classical predicates on an assembly~$S$.
  %
  Given an assembly map $f : S \to \nabla \two$, defined $p_f \in \Pred{S}$ by $\T{p_f} = \unit$ and
  % 
  \begin{equation*}
    \R{r} \rz p_f x
    \iff
    f x = 1.
  \end{equation*}
  % 
  Obviously, $p_f$ is $\neg\neg$-stable. Conversely, given a classical $p \in \Pred{S}$, define $f_p : S \to \nabla \two$ by
  % 
  \begin{equation*}
    f_p x =
    \begin{cases}
      1 & \text{if $p x \neq \emptyset$,} \\
      0 & \text{if $p x = \emptyset$.}
    \end{cases}
  \end{equation*}
  % 
  The map $f_p$ is realized because every map into a constant assembly is.
  %
  We claim that $f \mapsto p_f$ and $p \mapsto f_p$ form a bijective correspondence.
  It is easy to check that $f = f_{p_f}$ for all $f : S \to \nabla \two$.
  For the other direction, we need to show that a classical $p \in \Pred{S}$ satisfies $p \bientails p_{f_p}$. One direction is easy, and other one not much harder with the help of \cref{prop:when-stable}.

  It remains to be checked that $\Bool$ classifies the decidable predicates.
  The hard part of the proof was already done in \cref{prop:decidable-when}, and we leave the rest as an exercise.
\end{proof}

The \defemph{law of excluded middle} states that all predicates are decidable. It is never realized.

\begin{proposition}
  There exists a non-decidable predicate.
\end{proposition}

\begin{proof}
  Consider the predicate classified by $\id[\nabla \two]$.
  More precisely, it is the predicate $p \in \Pred{\nabla \two}$ defined by
  $\T{p} = \unit$ and $p 0 = \emptyset$, $p 1 = \Atyp{\unit}$.
  It is not decidable because every assembly map $\nabla \two \to \Bool$ is constant, since $\Bool$ is modest.
\end{proof}

The \defemph{law of double negation} states that all predicates are classical. Because it is inter-derivable with excluded middle, it cannot be valid in realizability logic.

\begin{exercise}
  Say that a predicate $p \in \Pred{S}$ is \defemph{semidecidable} if it is classified by the Rosolini dominance $\Sigma^0_1$. 
  %
  Suppose $\R{r} \in \subAtyp{\T{S} \to \ttnat \to \ttbool}$ is such that
  whenever $\R{x} \rz[S] x$ and $n \in \NN$ then $\R{r}\,\R{x}\,\overline{n} \in \set{\combTrue, \combFalse}$. Define $q_{\R{r}} \in \Pred{S}$ by $\T{q_{\R{r}}} = \unit$ and
  %
  \begin{equation*}
    \ttunit \rz q_{\R{r}}\,x
    \iff \some{n \in \NN} \R{r}\,\R{x}\,\overline{n} = \combTrue.
  \end{equation*}
  %
  Show that a predicate is semidecidable if, and only if, it is equivalent to some~$q_{\R{r}}$.
\end{exercise}

\begin{exercise}
  Is there a pca~$\AA$ such that in $\Asm{\AA}$ the decidable and semidecidable predicates coincide?
\end{exercise}

One might ask whether there is a two-element assembly~$\Omega$ which classifies \emph{all} predicates, as that would imply that assemblies form a topos. Alas, this is not the case.

\begin{proposition}
  If a predicate is classified by a two-element assembly then it is $\lnot\lnot$-stable.
\end{proposition}

\begin{proof}
  If a predicate is classified by $T$ then it is also classified by~$\nabla \two$, therefore it is $\lnot\lnot$-stable.
\end{proof}

The point is that not all predicates are $\lnot\lnot$-stable.

\begin{exercise}
  Give an example of a realizability predicate which is not $\lnot\lnot$-stable.
\end{exercise}



%%% Local Variables: 
%%% mode: latex
%%% TeX-master: "notes-on-realizability"
%%% End: 
