\section{Schools of Computable Mathematics}
\label{sec:schools}

Realizability is a unifying framework for several ``schools'' of computable mathematics.
To get a particular variation we just choose an appropriate tpca with sub-tpca. We look at some of them and relate the traditional terminology and notions to ours.

\subsection{Recursive Mathematics}
\label{sec:recursive-math}

\emph{Recursive Mathematics}~\sidecite{type-1}, also known as \emph{type
  one effectivity} or \emph{Russian constructivism}~\sidecite{russ}, is
computable mathematics done with type 1 machines, cf.\
\cref{sec:type-1}. In our settings it corresponds to the
category $\Rep{\klone}$ of representations over the first Kleene algebra.

An object of $\Rep{\klone}$ is called a \defemph{numbered set}. It is a pair
$(S, \delta_S)$, where $S$ is a set $\delta_S : \NN \parto S$ a
partial surjection, called a \defemph{numbering} of~$S$. A function $f :
S \to T$ is \defemph{realized} by $n \in \NN$ when, for all $m \in
\dom{\delta_S}$,
%
\begin{equation*}
  \text{$\defined{\pr{n}{m}}$ and $\delta_T(\pr{n}{m}) = f(\delta_S(m))$.}
\end{equation*}
%
A numbered set $(S, \delta_S)$ has countably many elements because it is covered by the
countable set~$\dom{\delta_S}$. This is sometimes considered a disadvantage and a reason
for preferring type~2 machines, which are able to compute with uncountable structures such
as real numbers. However, \emph{internally} to the category the reals form a
Cauchy-complete archimedean ordered field, on top of which it is perfectly possible to
develop a version of compute analysisg. One gets an unusual variant which is a rich source
of counter-examples.

\subsection{Equilogical spaces}
\label{sec:equilogical-spaces}

An equilogical space is a topological space with an
equivalence relation~\sidecite{BauerA:equs}. We study equilogical spaces
in some detail because they give us a general theory of computable
maps between countably-based spaces. In \cref{sec:tte} we
relate equilogical spaces to Type Two Effectivity, which is another
school of computability on general topological spaces.

Recall that a topological space is \defemph{countably based} or \defemph{2-countable} if it has a countable topological basis. Equivalently, a space is countably based
when it has a countable \emph{subbasis}. We prefer to work with
subbases because they simplify the treatment of computable maps
between spaces. Thus we define a countably based space to be a pair
$(X, (U_i)_{i \in \NN})$ where $X$ is a topological space and
$(U_i)_{i \in \NN}$ is an enumeration of \emph{subbasic} open sets.
These generate the topology of $X$ by taking finite intersections and
arbitrary unions. While we usually omit an explicit mention of the
subbasis $(U_i)_{i \in \NN}$, we do insist that a countably based
space always be given \emph{together with} a particular subbasis. This
allows us to avoid the axiom of choice.

The graph model~$\Scott$ is a countably based space. We always take its
subbasic open sets to be $\upper{n} = \set{A \subseteq \NN \such n \in A}$, for $n \in \NN$.

A \defemph{(countably based) equilogical space} $(X, (U_i)_{i \in \NN},
{\equiv_X})$ is a countably based topological space~$(X, (U_i)_{i \in
  \NN})$ with an equivalence relation~$\equiv_X$. We usually do not
bother writing the subbasis $(U_i)_{i \in \NN}$. The canonical
quotient map $q_X : X \to X/{\equiv_X}$ maps each $x \in X$ to its
equivalence class $[x]_X$. A morphism $f : (X,{\equiv_X}) \to
(Y,{\equiv_Y})$ is a map $f : X/{\equiv_X} \to Y/{\equiv_Y}$ between
equivalence classes for which there exists a continuous $g : X \to Y$
such that
%
\begin{equation*}
  \xymatrix@+1em{
    {X} \ar[d]_{q_X} \ar[r]^g
    &
    {Y} \ar[d]^{q_Y}
    \\
    {X/{\equiv_X}}
    \ar[r]_f
    &
    {Y/{\equiv_Y}}
  }
\end{equation*}
%
commutes, i.e., $f([x]_X) = [g(x)]_Y$ for all $x \in X$. Morphisms
compose as expected. The category of equilogical spaces and morphisms
between them is denoted by~$\Equ$.

A countably based topological space $X$ may be construed as an
equilogical space $(X, {=_X})$ with equality as the equivalence
relation. A morphism $f : (X, {=_X}) \to (Y, {=_Y})$ is the same
thing as a continuous map $f : X \to Y$ so that we have a full and
faithful embedding $\wTop \to \Equ$ of the category $\wTop$ of
countably based spaces into the category $\Equ$.

Let us show that $\Equ$ and $\Asm{\Scott}$ are equivalent. A
\defemph{pre-embedding} $e : X \to Y$ between topological spaces is a
continuous map such that $\inv{e} : \topol{Y} \to \topol{X}$ is
surjective. For $T_0$-spaces this is equivalent to~$e$ being an
embedding. If $(U_i)_{i \in I}$ is a topological subbasis for $Y$ and
$e : X \to Y$ is a pre-embedding, then $(\invim{e}(U_i))_{i \in I}$ is
a topological subbasis for~$X$.

\begin{theorem}[Embedding Theorem for $\Scott$]
  \label[embeddingTheorem]{thm:scott-embedding}%
  A space $X$ may be pre-emebedded in~$\Scott$ if, and only if, it is
  countably based.
\end{theorem}

\begin{proof}
  Here $\Scott$ is equipped with the Scott topology. If $e : X \to
  \Scott$ is a pre-embedding then the open sets $U_n =
  \invim{e}(\upper{n})$ form a countable subbasis for~$X$.

  Conversely, suppose $(U_n)_{n \in \NN}$ is a countable subbasis
  for~$X$. Define the map $e_X : X \to \Scott$ by
  %
  \begin{equation*}
    e_X(x) = \set{n \in \NN \such x \in U_n}.
  \end{equation*}
  %
  We claim that $e_X$ is a pre-embedding. It is continuous because
  $\invim{e_X}(\upper{n}) = U_n$. Let $V \subseteq X$ be open. Then 
  $V$ is a union of finite intersections of subbasic opens,
  %
  \begin{equation*}
    V = \tbigcup_i U_{n_{i,1}} \cap \cdots \cap U_{n_{i,k_i}}
  \end{equation*}
  %
  Now
  %
  \begin{align*}
    \invim{e_X}\left(\tbigcup_i \upper{\set{n_{i,1}, \ldots, n_{i,k_i}}}\right) &=
      \tbigcup_i \invim{e_X}(\upper{\set{n_{i,1}, \ldots, n_{i,k_i}}})
      \\
      &=
      \tbigcup_i \invim{e_X}(\upper{n_{i,1}} \cap \cdots \cap \upper{n_{i,k_i}})
      \\
      &=
      \tbigcup_i \invim{e_X}(\upper{n_{i,1}}) \cap \cdots \cap \invim{e_X}(\upper{n_{i,k_i}})
      \\
      &=
      \tbigcup_i U_{n_{i,1}} \cap \cdots \cap U_{n_{i,k_i}} \\
      &= V,
  \end{align*}
  %
  therefore $\invim{e_X}$ is surjective, as required.
\end{proof}

\noindent
%
The pre-embedding $e_X : X \to \Scott$ is called the \defemph{(subbasic)
  neighborhood filter} because $e_X(x)$ is just the set of (indices
of) subbasic neighborhoods of~$x$. Henceforth $e_X : X \to \Scott$
will always denote the subbasic neighobrhood filter.

\begin{theorem}[Extension Theorem for $\Scott$]
  \label[extensionTheorem]{thm:scott-extension}%
  Suppose $e : X \to Y$ is a pre-embedding and $f : X \to \Scott$
  continuous. Then~$f$ has a continuous extension $g : Y \to \Scott$
  along~$e$.
\end{theorem}

\begin{proof}
  Consider the map $g : Y \to \Scott$ defined by
  %
  \begin{equation*}
    g(y) = \tbigcup_{U \in \topol{Y}} \left\{
      \tbigcap_{z \in \invim{e}(U)} f(z)
      \such
      y \in U
    \right\}.
  \end{equation*}
  %
  It is continuous because
  %
  \begin{align*}
    \invim{g}(\upper{n}) &=
    \set{y \in Y \such n \in g(y)} \\
    &=
    \set{y \in Y \such \some{U}{
        \topol{Y}}{y \in U \land
        \all{z}{\invim{e}(U)}{n \in f(z)}
      }
    } \\
    &=
    \tbigcup_{U \in \topol{Y}} \left\{
        U \such
        \all{z}{\invim{e}(U)}{n \in f(z)}
      \right\}.
  \end{align*}
  %
  Let us show that $g(e(x)) = f(x)$ for all $x \in X$. Consider the
  value
  %
  \begin{align*}
    g(e(x)) =
    \tbigcup_{U \in \topol{Y}} \left\{
      \tbigcap_{z \in \invim{e}(U)} f(z)
      \such
      e(x) \in U \right\}.
  \end{align*}
  %
  Because $f(x)$ appears in every intersection, each of them is
  contained in~$f(x)$, which shows that $g(e(x)) \subseteq f(x)$.
  Suppose $n \in f(x)$. Because $e$ is a pre-embedding there exists $W
  \in \topol{Y}$ such that $\invim{f}{\upper{n}} = \invim{e}(W)$. If
  $z \in X$ and $e(z) \in W$ then $z \in \invim{e}(W) =
  \invim{f}(\upper{n})$, hence $n \in f(z)$. The intersection
  $\tbigcap \left\{f(z) \such z \in X \land e(z) \in W \right\}$
  contains~$n$ and so $n \in g(e(x))$. We proved that $f(x) \subseteq
  g(e(x))$, therefore $f(x) = g(e(x))$.
\end{proof}

\noindent
%
The Embedding and Extension theorems now give us the desired
equivalence~\sidecite{Simpson-Menni}.

\begin{proposition}
  \label{prop:equ-equiv-asm-scott}%
  The categories $\Equ$ and $\Asm{\Scott}$ are equivalent.
\end{proposition}

\begin{proof}
  Suppose $(X, {\equiv_X})$ is an equilogical space, and let $e_X : X
  \to \Scott$ be the subbasic neighborhood filter pre-embedding.
  Define the assembly $F(X) = (X/{\equiv_X}, \Ex_{F(X)})$ by
  $\Ex_{F(x)} = \set{e_X(y) \such x \equiv_X y}$. To make $F$ into a
  functor we define $F(f) = f$ for a morphism $f : (X, {\equiv_X}) \to
  (Y, {\equiv_Y})$. If $f$ is tracked by $g : X \to Y$, then $F(f)$ is
  realized by a continuous extension of $e_Y \circ g : X \to \Scott$
  along~$e_X$, which exists by \cref{thm:scott-extension}.

  The functor $G : \Asm{\Scott} \to \Equ$ is defined as follows. An
  assembly $(S, \Ex_S)$ is mapped to the equilogical space $G(S) =
  (X_S, \equiv_{G(S)})$ whose underlying space is the set $X_S =
  \set{(x,A) \in S \times \Scott \such A \in \Ex_S(x)}$, equipped with
  the unique topology that makes the projection $p : X_S \to \Scott$,
  $p : (x,A) \mapsto A$, a pre-embedding. Explicitly, the open subsets
  of $X_S$ are the inverse images $\invim{p}(U)$ of open subsets $U
  \subseteq \Scott$. Let $\equiv_{G(S)}$ be the equivalence relation
  defined by
  %
  \begin{equation*}
    (x,A) \equiv_{G(S)} (y,B) \iff x = y.
  \end{equation*}
  %
  A morphism $f : (S, {\rz_S}) \to (T, {\rz_T})$ which is tracked by
  $B \in \comp{\Scott}$ is mapped to $G(f) : X_S/{\equiv_{G(S)}} \to
  X_T/{\equiv_{G(T)}}$ defined by $G(f)([(x,A)]_{G(S)}) = [(f(x), B
  \cdot A)]_{G(T)}$.

  We leave the verification that $F$ and $G$ form an equivalence of
  categories as exercise.
\end{proof}

By restricting to the $T_0$-spaces we obtain another equivalence. Let
$\Equ_0$ be the full subcategory of~$\Equ$ in which the underlying
topological spaces are $T_0$-spaces.


\begin{proposition}
  \label{prop:equ0-equiv-mod-scott}
  The categories $\Equ_0$ and $\Mod{\Scott}$ are equivalent.
\end{proposition}

\begin{proof}
  We verify that the equivalence functors $F$ and $G$ from the proof
  of \cref{prop:equ-equiv-asm-scott} restrict to $\Equ_0$
  and $\Mod{\Scott}$. If $(X, {\equiv_X})$ is an equilogical space
  whose underlying space $X$ is $T_0$, then the pre-embedding $e : X
  \to \Scott$ is actually an embedding. Because it is injective the
  assembly $F(X)$ is modest. This shows that $F$ restricts to a
  functor $\Equ_0 \to \Mod{\Scott}$.
  %
  To see that $G$ restricts to a functor $\Mod{\Scott} \to \Equ_0$,
  observe that, for a modest assembly $(S, \Ex_S)$, the projection
  $X_S \to \Scott$ is an embedding, therefore $X_S$ is a $T_0$-space.
\end{proof}

%%%%%%%%%%%%%%%%%%%%%%%%%%%%%%%%%%%%%%%%%%%%%%%%%%
\subsection{Computable countably based spaces}
\label{sec:computable-countably based-spaces}

We have so far studied the \emph{continuous} version of realizability
over the graph model in which the realizers for morphisms may be
arbitrary continuous maps. But what about the \emph{mixed}
version~$\Asm{\Scott, \comp{\Scott}}$, is it also equivalent to a
version of equilogical spaces? To see that the answer to the question
is affirmative, we first need to define computable maps between
countably based spaces.

Recall from \cref{sec:graph-model} that an enumeration operator
$g: \Scott \to \Scott$ is computable when its graph $\Gamma(g)$ is a
c.e.~set. By \cref{thm:scott-embedding}, every
countably based $T_0$-space $X$ can be embedded in $\Scott$, and every
continuous map $f: X \to Y$ can be extended to an enumeration operator
$g: \Scott \to \Scott$, so that the following diagram commutes:
%
\begin{equation*}
  \xymatrix{
    {X}   \ar@{ >->}[d]_{e_X} \ar[r]^f  &
    {Y} \ar@{ >->}[d]^{e_Y} \\
    {\Scott} \ar[r]^{g} &
    {\Scott}
  }
\end{equation*}
%
We can define a \defemph{computable continuous map} $f: X \to Y$ to be a
continuous map for which there exists a computable enumeration
operator $g: \Scott \to \Scott$ which makes the above diagram commute.
This idea gives the following definition of computable continuous
maps.

\begin{definition}
  \label{def:computable-map}%
  %
  \indexdef{computable continuous map}%
  %
  A continuous map $f : X \to Y$ between countably based spaces $(X,
  (U_i)_{i \in \NN})$ and $(Y, (V_j)_{j \in \NN})$ is
  \defemph{computable} when there exists a c.e.~set $F \subseteq \NN
  \times \NN$ such that:
  %
  \begin{enumerate}
  \item
    %
    $F$ is monotone in the first argument: if $A \subseteq B \wayb
    \NN$ and $\pair{\code{A}, m} \in F$ then $(\code{B}, m) \in
    F$.
  \item
    %
    $F$ approximates~$f$: if $(\code{\set{i_1, \ldots, i_n}}, m) \in
    F$ then $f(U_{i_1} \cap \cdots \cap U_{i_n}) \subseteq V_m$.
  \item
    %
    $F$ converges to~$f$: if $f(x) \in V_m$ then there exist $i_1,
    \ldots, i_n$ such that $x \in U_{i_1} \cap \cdots \cap U_{i_n}$
    and $(\code{\set{i_1, \ldots, i_n}}, m) \in F$.
  \end{enumerate}
  %
  \indexdef{realizer!for computable continuous map}%
  %
  The relation~$F$ is called a~\defemph{c.e.~realizer} for~$f$. We also
  say that~$F$ \defemph{tracks}~$f$.
  %
  \indexdef{category!of effective topological spaces}%
  %
  The category of countably based spaces and computable continuous
  maps is denoted by $\compTop$.
\end{definition}

The category $\compTop$ is well-defined. The identity map $\id[X]:
X \to X$ is tracked by the relation $I_X$, defined by
%
\begin{equation*}
  I_X = \set{
    (\code{\set{i_1, \ldots, i_n}}, m) \in \NN \times \NN \such
    m \in \set{i_1, \ldots, i_n}
  }.
\end{equation*}
%
The composition of computable maps $f: X \to Y$ and $g: Y \to Z$,
which are tracked by $F$ and $G$ respectively, is again a computable
map $g \circ f: X \to Z$ because it has a c.e.~realizer $H$ defined by
%
\begin{multline*}
  (\code{\set{j_1, \ldots, j_k}}, \ell) \in H
  \iff {} \\
    \some{i_1, \ldots, i_n}{\NN}{
      (\code{\set{i_1, \ldots, i_n}}, \ell) \in G \land
      \bigwedge\nolimits_{s = 1}^{n}
      (\code{\set{j_1, \ldots, j_k}}, i_s) \in F
    }.
\end{multline*}
%
The monotonicity condition in \cref{def:computable-map} is
redundant, for if $F$ is an c.e.~relation that satisfies the second
and the third condition, then we can recover monotonicity by defining
a new relation $F'$ by
%
\begin{equation*}
  F' = \set{(\code{A}, m) \in \NN \times \NN \such
    \bigvee\nolimits_{B \subseteq A} (\code{B}, m) \in F
  }.
\end{equation*}
%
It is easy to see that $F'$ satisfies all three conditions and
realizes the same function as~$f$.

A point $x \in X$ is \defemph{computable} when the map $\set{\star} \to
X$ which maps $\star$ to~$x$ is computable. This is equivalent to
requiring that $e_X(x) = \set{i \in \NN \such x \in U_i}$ is a
c.e.~set.

Next, we prove effective versions of the Embedding and Extension
Theorems.

\begin{theorem}[Computable Embedding Theorem]
  \label[embeddingTheorem]{th:computable_embedding_theorem}%
  %
  \index{Theorem!Computable Embedding Theorem}%
  \index{computable!Computable Embedding Theorem}%
  %
  Every countably based space can be computably pre-embedded
  into~$\Scott$.
\end{theorem}

\begin{proof}
  We just need to provide a c.e.~realizer~$E_X$ for the neighborhood
  filter $e_X : X \to \Scott$. It is
  %
  \begin{equation*}
    E_X = \set{
      (\code{\set{i_1, \ldots, i_n}}, m) \in \NN \times \NN \such
      m \in \set{i_1, \ldots, i_n}
    }.
  \end{equation*}
  %
  This is obviously a c.e.~relation which is monotone in the first
  argument. The second condition for the c.e.~realizer~$E_X$ is
  %
  \begin{equation*}
     (\code{\set{i_0, \ldots, i_n}}, m) \in E_X
     \implies
     e_X(U_{i_1}, \ldots, U_{i_n}) \subseteq \upper{m},
  \end{equation*}
  %
  which clearly holds.
  %
  Suppose $e_X(x) \in \upper{m}$. Then $x \in U_m$, and
  $(\code{\set{m}}, m) \in E_X$, which proves the third condition.
\end{proof}

\begin{theorem}[Computable Extension Theorem]
  \label[extensionTheorem]{th:computable_extension_theorem}%
  %
  \index{Theorem!Computable Extension Theorem}%
  \index{computable!Computable Extension Theorem}%
  %
  Let $X$ and $Y$ be countably based topological spaces and $f: X \to
  Y$ a computable map between them. Then there exists a computable map
  $g: \Scott \to \Scott$ such that the following diagram commutes:
  %
  \begin{equation*}
    \xymatrix{
      {X} \ar[r]^{f} \ar@{ >->}[d]_{e_X}  &
      {Y} \ar@{ >->}[d]^{e_Y} \\
      {\Scott} \ar[r]_{g} &
      {\Scott} 
    }
  \end{equation*}
  %
\end{theorem}

\begin{proof}
  The maps $e_X$ and $e_Y$ are the computable embeddings from
  \cref{th:computable_embedding_theorem}. Let $F$ be a
  c.e.~realizer for $f$. We define the map $g: \Scott \to \Scott$ by
  specifying its graph to be~$F$, i.e.,
  %
  \begin{equation*}
    g(A) = \set{m \in \NN \such
      \some{i_1, \ldots, i_n}{\NN}{
        (\code{\set{i_1, \ldots, i_n}}, m) \in F
      }}.
  \end{equation*}
  %
  All we have to show is that this choice of $g$ makes the diagram
  commute. For any $x \in X$,
  %
  \begin{align*}
    m &\in g(e_X(x)) \\
      &\liff
      \some{i_1, \ldots, i_n \in \NN}{
        x \in U_{i_1} \cap \cdots \cap U_{i_n} \land
        (\code{\set{i_1, \ldots, i_n}}, m) \in F
      } \\
      &\liff
      f(x) \in V_m
      \\
      &\liff
      m \in e_Y(f(x)).
  \end{align*}
  %
  The second equivalence is implied from left to right by the second
  condition in \cref{def:computable-map}, and from right to
  the left by the third condition.
\end{proof}

We should point out that the computable continuous maps, as defined
here, work at the level of open sets, i.e., a c.e.~realizer $F$ for $f
: X \to Y$ operates on the (indices) of subbasic open sets. Therefore,
$F$ does not distinguish points that share the same neighborhoods,
even though $f$ might. This is not an issue with $T_0$-spaces in which
points are distinguished by their neighborhoods.


%%%%%%%%%%%%%%%%%%%%%%%%%%%%%%%%%%%%%%%%%%%%%%%%%%

\subsection{Computable equilogical spaces}
\label{sec:computable-equ}


With a notion of computable maps between spaces at hand, we can define
the computable equilogical spaces just like the ordinary ones, except
that we replace continuous maps by their computable versions.

\begin{definition}
  \label{def:computable-equ}%
  %
  \indexdef{computable!equilogical space}%
  \indexsee{computable!equilogical space}{equilogical space, comuptable}%
  \indexdef{equilogical space!computable}%
  %
  A morphism $f : (X, {\equiv_X}) \to (Y, {\equiv_Y})$ between
  equilogical spaces is \defemph{computable} if there exists a computable
  continuous map $g : X \to Y$ which tracks~$f$.
  %
  The category of equilogical spaces and computable morphisms between
  them is denoted by $\comp{\Equ}$.
\end{definition}

We check that we got the definition right by proving
that~$\comp{\Equ}$ is equivalent to $\Asm{\Scott, \comp{\Scott}}$.

\begin{proposition}
  \label{th:equivalence_compEqu_AsmScott}%
  %
  \index{equilogical space!computable!equivalent definitions}%
  %
  The categories $\Asm{\Scott, \comp{\Scott}}$ and $\comp{\Equ}$ are
  equivalent.
\end{proposition}

\begin{proof}
  The proof goes just as the proof of
  \cref{prop:equ-equiv-asm-scott} that $\Equ$ and
  $\Asm{\Scott}$ are equivalent. The only difference is that we refer
  to \cref{th:computable_embedding_theorem} and \cref{th:computable_extension_theorem} in order to
  extend computable maps to computable enumeration operators.
\end{proof}

The category $\comp{\wTop}$ of countably based spaces and computable
maps is embedded fully and faithfully into $\comp{\Equ}$. The
embedding works as in the continuous case: a topological space~$X$ is
mapped to the equilogical space $(X, {=_X})$, and a computable
continuous map $f : X \to Y$ is the same thing as a morphism $f : (X,
{=_X}) \to (Y, {=_Y})$.


%%%%%%%%%%%%%%%%%%%%%%%%%%%%%%%%%%%%%%%%%%%%%%%%%%
\subsection{Type Two Effectivity}
\label{sec:tte}

A popular realizability model is Kleene's~\sidecite{KV65} \defemph{function
  realizability}, which is better known by its newer name \defemph{Type
  Two Effectivity (TTE)}~\sidecite{Wei00}. As the names say, it is the
model of realizability based on functions and type 2 machines. 

TTE is traditionally expressed as a theory of representations. There
are actuallly three variations:
%
\begin{enumerate}
\item $\Rep{\Baire, \Baire}$ is the \emph{continuous} version in which
  maps are realized by continuous realizers.
\item $\Rep{\Baire, \comp{\Baire}}$ is the \emph{relative} version.
\item $\Rep{\comp{\Baire}, \comp{\Baire}}$ is the \emph{computable}
  version in which all realizers must be computable.
\end{enumerate}
%
Mostly only the first two of these are used. Sometimes multi-valued
representations are considered also, and for these we need to move to
the larger category of assemblies $\Asm{\Baire, \comp{\Baire}}$.

Specifically, a representation $(S, \delta_S)$ over the Baire space is
a partial surjection $\delta_S : \Baire \to S$. When $\delta_S(\alpha)
= x$ we say that $\alpha$ is a \defemph{$\delta_S$-name} of~$x$. A
\defemph{(continuously) realized map} $f :(S, \delta_S) \to (T,
\delta_T)$ is a map $f : S \to T$ for which there exists a partial
continuous map $g : \Baire \parto \Baire$ such that
$f(\delta_S(\alpha)) = \delta_T(g(\alpha))$ for all $\alpha \in
\dom{\delta_S}$. If the realizer $g$ is computable we say that $f$ is
\defemph{computably realized}. Recall that a computable $g$ corresponds
to a type~2 machine which converts a $\delta_S$-name of~$x$ to a
$\delta_T$-name of $f(x)$.

In the case of equilogical spaces there was a straightforward way of
turning a topological space into an equilogical space. The present
situation is less obvious. One obvious idea is to represent a
topological space~$X$ by a representation $\delta_X : \Baire \parto X$
for which $\delta_X$ is a quotient map. However, this is too weak a
requirement. To see this, suppose $\delta_X : \Baire \parto X$ and
$\delta_Y : \Baire \parto Y$ are representations of topological spaces
with $\delta_X$ and $\delta_Y$ topological quotient maps. Then a
continuous map $f : X \to Y$ may be lifted to $g : \Baire \parto Y$,
as in the diagram
%
\begin{equation*}
  \xymatrix@+1em{
    {\Baire}
    \ar[d]_{\delta_X}
    \ar[dr]^{g}
    \ar@{..>}[r]^{h?}
    &
    {\Baire}
    \ar[d]^{\delta_Y}
    \\
    {X}
    \ar[r]_f
    &
    {Y}
  }
\end{equation*}
%
because $\delta_X$ is a quotient map. But to make~$f$ into a morphism
we need a continuous realizer $h : \Baire \parto \Baire$ on the top
line, which however might not exist. A stronger property is required.

\begin{definition}
  Suppose $X$ is a topological space and $\delta_X : \Baire \parto X$
  is a representation and a topological quotient map. Then $\delta_X$
  is \defemph{admissible} if every continuous $g : \Baire \parto X$ has a
  continuous lifting $h : \Baire \parto \Baire$ such that
  $\delta_X(h(\alpha)) = g(\alpha)$ for all $\alpha \in \dom{g}$.
\end{definition}

Admissible representations are a central concept in TTE because they
are very well behaved. For example, if $\delta_X : \Baire \parto X$
and $\delta_Y : \Baire \parto Y$ are admissible, the continuous maps
$f : X \to Y$ coincide with the realized maps.

The spaces which have admissible representations have been studied in
depth~\sidecite{Schroeder,Weihrauch}. We only mention a basic result whose
proof is not too complicated.

\begin{proposition}
  Every countably based $T_0$-space has an admissible representation.
\end{proposition}

\begin{proof}
  Suppose $(X, (U_i)_{i \in \NN})$ is a countably based $T_0$-space,
  and let $e_X : X \to \Scott$ be the neighborhood filter. Define the
  representation $\delta_X : \Baire \parto X$ by
  %
  \begin{equation*}
    \delta_X(\alpha) = x \iff
    e_X(x) = \set{\alpha(n) \such n \in \NN}.
  \end{equation*}
  %
  In words, $\alpha$ is a $\delta_X$-name for $x$ when it enumerates
  the (indices of) subbasic open neighborhoods of~$x$. Because $X$ is
  a $T_0$-space, any~$\alpha$ enumerates the subbasic neighborhood
  filter of at most one~$x$, hence $\delta_X$ is single-valued. We
  leave admissibility of~$\delta_X$ as an exercise.
\end{proof}

Further reading: Weihrauch degrees.


%%%%%%%%%%%%%%%%%%%%%%%%%%%%%%%%%%%%%%%%%%%%%%%%%%
\subsection{A relationship between equilogical spaces and TTE}
\label{sec:equ-tte}

Type~2 representations, or more precisely $\Asm{\Baire,
  \comp{\Baire}}$ are embedded in $\Asm{\Scott, \comp{\Scott}}$ by a
full and faithful embedding. The embedding has a right adjoint. We
outline the relationship here and refer to~\sidecite{Bauer:equ-tte} for
details.\sidenote{The general topic of comparing realizability models
  is important but excluded from these lecture notes, for the time
  being.}

We would like to translate assemblies over $\Scott$ to asemblies over
$\Baire$, and vice versa. This can be done systematically if can
``simulate'' the pca $\Baire$ within the pca $\Scott$, and vice versa.
%
Define the map $\iota : \Baire \to \Scott$ by
%                                %
\begin{equation*}
  \iota(\alpha) =
  \set{ \code{\seg{\alpha}{n}} \such n \in \NN}.
\end{equation*}
                                %
An infinite sequence $\alpha$ is represented as the set (of codes) of
its finite prefixes. This is a topological embedding, as can be easily
verified. Observe also that if $\alpha \in \comp{\Baire}$ then
$\iota(\alpha) \in \comp{\Scott}$, hence $\iota$ restricts to a map
$\comp{\Baire} \to \comp{\Scott}$.
%
The map $\iota$ is computable if we equip $\Baire$ with the subbasic
open sets $U_{\code{i,j}} = \set{\alpha \in \Baire \such \alpha(i) =
  j}$. Next we seek a computable continuous map $p : \Scott \times
\Scott \to \Scott$ which simulates application in~$\Baire$, i.e.,
whenever $\fpr{\alpha}{\beta}$ is defined then
%
\begin{equation*}
  p (\iota(\alpha), \iota(\beta)) = \iota(\fpr{\alpha}{\beta}).
\end{equation*}
%
The existence of~$p$ may be inferred from \cref{th:computable_extension_theorem} by noting that the map
$(\alpha, \beta) \mapsto \fpr{\alpha}{\beta}$ is continuous and
computable, where defined.

The map $\iota$ transfers realizers in~$\Baire$ to realizers
in~$\Scott$ and so induces a functor
%
\begin{equation*}
  I : \Asm{\Baire, \comp{\Baire}} \to \Asm{\Scott, \comp{\Scott}}
\end{equation*}
%
which maps an assembly $(S, {\rz_S})$ to $I(S) = (S, {\rz_{I(S)}})$
where $\rz_{I(S)}$ is characterized as
%
\begin{equation*}
  \alpha \rz_S x \iff
  \iota(\alpha) \rz_{I(S)} x,
\end{equation*}
%
for $x \in S$ and $\alpha \in \Baire$.
%
Suppose $f : (S, {\rz_S}) \to (T, {\rz_T})$ is a morphism in
$\Asm{\Baire, \comp{\Baire}}$, realized by $\gamma \in \comp{\Baire}$.
Then $f$ is realized as a morphism $I(f) = f : I(S) \to I(T)$ by the
computable enumeration operator $A \mapsto p(\iota(\gamma), A)$
because
%
\begin{multline*}
  \iota(\alpha) \rz_{I(S)} x \implies
  \alpha \rz_S x \implies
  \fpr{\gamma}{\alpha} \rz_T f(x) \implies {}\\
  \iota(\fpr{\gamma}{\alpha}) \rz_T f(x) \implies
  p(\iota(\gamma), \iota(\alpha)) \rz_T f(x).
\end{multline*}
%
The embedding $I$ is full and faithful.

To get a functor $D : \Asm{\Scott, \comp{\Scott}} \to \Asm{\Baire,
  \comp{\Baire}}$ we use a similar idea, except this time the
simulation $\delta : \Scott \multito \Baire$ is a multi-valued map.
Define
%
\begin{equation*}
  \delta(A) = \set{\alpha \in \Baire \such
    \all{n \in \NN} (n \in A \iff \some{m}{\NN}{\alpha(m) = n + 1})
  }.
\end{equation*}
%
A set $A \subseteq \NN$ is represented by any enumeration $\alpha$ of
its elements, where we used the usual trick of adding~$1$ so that the
empty set is represented. Again, it does not take much effort to see
that there is a computable map $q : \Baire \times \Baire \to \Baire$
which simulates application in~$\Scott$,
%
\begin{equation*}
  \alpha \in \delta(A) \land
  \beta \in \delta(B) \implies q(\alpha, \beta) \in \delta(A \cdot B).
\end{equation*}
%
The functor $D$ maps an assembly $(S, {\rz_S})$ over~$\Scott$ to the
assembly $D(S) = (S, {\rz_{D(S)}})$ over $\Baire$ whose realizability
relation is characterized by
%
\begin{equation*}
  \alpha \rz_{D(S)} x \iff
  \alpha \in \delta(A) \land A \rz_S x,
\end{equation*}
%
for $\alpha in \Baire$, $A \in \Scott$, and $x \in S$. If $f : (S,
{\rz_S}) \to (T, {\rz_T})$ is realized by a computable enumeration
operator $g : \Scott \to \Scott$, then it is realized as a morphism
$D(f) = f : D(S) \to D(T)$ by the computable map $\alpha \mapsto
q(\gamma, \alpha)$ where $\gamma$ is any member of
$\delta(\Gamma(g))$. The functor $D$ is faithful but is not full.


%%%%%%%%%%%%%%%%%%%%%%%%%%%%%%%%%%%%%%%%%%%%%%%%%%
\subsection{Other schools}
\label{sec:other-schools}

There are other schools of computable mathematics which fall within
our framework. We only mention two:
%
\begin{enumerate}
\item
  %
  Closely related to the previous case is $\xasm{U, \comp{U}}$ where
  $U$ is a universal Scott domain~\sidecite{GunterScott} and $\comp{U}$ is
  its computable counterpart. This model has been studied under the
  name of \defemph{domain representations}~\sidecite{Bla97,Bla97a}.
\item
  The case $A = \PCFinf$ and $\comp{A} = \PCF$ was used by
  Mart{\'\i}n Escard\'o in his investigations of compact, exhaustible
  and searchable sets~\sidecite{Escardo}.
\end{enumerate}


%%% Local Variables:
%%% mode: latex
%%% TeX-master: "notes-on-realizability"
%%% End:
