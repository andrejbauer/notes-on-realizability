\documentclass[a4paper,11pt]{article}

\usepackage{a4wide}
\usepackage{amsmath}
\usepackage{amssymb}
\usepackage[utf8]{inputenc}
\usepackage{theorem}

\newcommand{\set}[1]{\{#1\}}
\newcommand{\such}{\mid}

%%% Blackboard bold letters
\newcommand{\NN}{\mathbb{N}}
\newcommand{\NNx}{{\NN^{+}}}
\newcommand{\ZZ}{\mathbb{Z}}
\newcommand{\QQ}{\mathbb{Q}}
\newcommand{\RR}{\mathbb{R}}
\newcommand{\CC}{\mathbb{C}}

%%% quantifiers
\newcommand{\all}[3]{\forall\, #1 \,{\in}\, #2\,.\left(#3\right)}
\newcommand{\some}[3]{\exists\, #1 \,{\in}\, #2\,.\left(#3\right)}
\newcommand{\exactlyone}[3]{\exists!\, #1 \,{\in}\, #2\,.\left(#3\right)}
\newcommand{\lam}[3]{\lambda #1 \,{\in}\, #2\,.\left(#3\right)}
\newcommand{\uall}[2]{\forall\, #1\,.\left(#2\right)}
\newcommand{\usome}[2]{\exists\, #1\,.\left(#2\right)}
\newcommand{\uexactlyone}[3]{\exists!\, #1\,.\left(#2\right)}
\newcommand{\ulam}[2]{\lambda #1 .\left(#2\right)}
\newcommand{\xall}[3]{\forall\, #1 \,{\in}\, #2\,.\,#3}
\newcommand{\xsome}[3]{\exists\, #1 \,{\in}\, #2\,.\,#3}
\newcommand{\xexactlyone}[3]{\exists!\, #1 \,{\in}\, #2\,.\,#3}
\newcommand{\xuall}[2]{\forall\, #1\,.\,#2}
\newcommand{\xusome}[2]{\exists\, #1\,.\,#2}
\newcommand{\xuexactlyone}[2]{\exists!\, #1,.\,#2}
\newcommand{\xlam}[3]{\lambda #1 \,{\in}\, #2\,.\,#3}
\newcommand{\xulam}[2]{\lambda #1 .\,#2}
\newcommand{\tlam}[3]{\lambda #1 \,{:}\, #2\,.\,\left(#3\right)}
\newcommand{\xtlam}[3]{\lambda #1 \,{:}\, #2\,.\,#3}


\begin{document}

\title{Izračunljivost v topologiji}
\author{Homework 3}
\date{}

\maketitle


\subsection*{Instructions}

You are \emph{encouraged} to use the literature and research papers,
but any mathematical text that your solution relies on must come from
a peer-reviewed source (published book or journal article).
%
If you use information from Wikipedia or random PDF files on the
internet, then the correctness of the source becomes \emph{your}
responsibility, i.e., you should personally verify the proofs or write
them if they are missing.
%
Thus you may solve a problem, or part of a problem, by finding the
solution in a book or journal article. In this case you may simply
refer to the text you found (there is no need to copy it). If a
problem asks for a proof, you must refer to an actual proof (not a
paper which claims that the proof is trivial or left as exercise).

I prefer a hand-written solution to one written with Microsoft Word in
which mathematics is messed up. You may but do not have to use
{\LaTeX}.
%
If the problems are too hard, come to me and we will discuss them.


\subsection*{Problem 1}

Let $\asm{S}$ be an assembly and $p \in \Pred{\asm{S} \times \asm{S}}$
a realizability predicate on $\asm{S} \times \asm{S}$. The following
statement is known as \emph{dependent choice}:
%
\begin{equation*}
  (\xall{x}{\asm{S}}{\xsome{y}{\asm{S}}{p(x,y)}}) \implies
  \xall{z}{\asm{S}}{
    \some{f}{\asm{S}^{\asm{N}}}{
      f(0) = z \land \xall{n}{\asm{N}}{p(f(n), f(n+1))}
    }
  }.
\end{equation*}
%
Write a realizer for this statement. You can either do it in a general
N-TPCA, or in your favorite programming language. Explain what your
program does. How does the program simplify if we assume that $p$ is
$\lnot\lnot$-stable?


\subsection*{Problem 2}

In this problem we work in $\Asm{K_1}$ where $K_1$ is the first Kleene
algebra, i.e., we work with type 1 Turing machines. Recall the
definitions of standard assemblies for booleans $\two$, natural
numbers $\asm{N}$, integers $\asm{Z}$, and real numbers $\asm{R}$:
%
\begin{itemize}
\item $\two = (\set{0,1}, {\rz_2})$ where $0 \rz_2 0$ and
  $1 \rz_2 1$,
\item $\asm{N} = (\NN, {\rz_{\NN}})$ where $n \rz_\NN n$ for every $n
  \in \NN$,
\item $\asm{Z} = (\ZZ, {\rz_\ZZ})$ where $\pair{m,n} \rz_\ZZ k$ if $k
  = m - n$ for all $m, n \in \NN$ and $k \in \ZZ$,
\item $\asm{R} = (R, {\rz_R})$ where, for $n \in \NN$ and $x \in \RR$,
  $n \rz_R x$ if, and only if, for every $k \in \NN$ there are $a, b,
  c \in \NN$ such that $\pr{n}{k} = \pair{\pair{a, b}, c}$ and $|x -
  (a - b)/c| \leq 2^{-k}$. In other words, a real number~$x$ is
  represented by a Turing machine which accepts a number~$k$ and
  outputs a rational number $q = (a-b)/c$ such that $|x - q| \leq
  2^{-k}$. The set $R = \set{x \in \RR \such \xsome{n}{\NN}{n \rz_R
      x}}$ consists of those real numbers that have such a realizer.
\end{itemize}
%
For each of the following statements determine whether it is realized
in $\Asm{K_1}$:
%
\begin{enumerate}
\item $\all{f}{\two^\NN}{(\xall{n}{\asm{N}}{f(n)=0}) \lor
    (\xsome{n}{\asm{N}}{f(n) = 1})}$,
\item $\all{x}{\asm{R}}{x \leq 0 \lor 0 < x}$,
\item $\all{x}{\asm{R}}{x \neq 0 \implies x < 0 \lor 0 < x}$,
\item $\xall{x}{\asm{R}}{\some{k}{\asm{Z}}{k \leq x \land x < k + 1}}$,
\item $\xall{x}{\asm{R}}{\some{k}{\asm{Z}}{k \leq x \land x < k + 2}}$.
\end{enumerate}
%
You may argue informally when you claim that a realizer exists, but
you should provide a rigorous proof when you claim a realizer does not
exist. Hint: if you show that $\phi$ does not have a realizer and
intuitionistically prove $\psi \implies \phi$, then $\psi$ does not
have a realizer either.

\subsection*{Problem 2}

A map $f : \RR \to \RR$ is a \emph{contraction} if there exists $0
\leq \alpha < 1$ such that $|f(x) - f(y)| \leq \alpha \cdot |x - y|$
for all $x, y \in \RR$. Banach fixed-point theorem says that every
contraction has a (unique) fixed point, i.e., a point $x \in \RR$ such
that $x = f(x)$.

State Banach fixed-point theorem as a formula in first-order logic to
make it precise. Prove the theorem intuitionistically and eplain how
a realizer based on your proof works: what are the inputs, the
outputs, and the underlying algorithm?


\subsection*{Problem 4}

Prove the following statement in synthetic topology:
%
\begin{enumerate}
\item If $\asm{X}$ is compact and $\asm{Y}$ discrete then
  $\asm{Y}^\asm{X}$ is discrete.
\item A compact subspace of a Hausdorff space is closed.
\item The image of a compact space is compact.
\item If $\two^\asm{N}$ is compact then $[0,1]$ is compact.
\end{enumerate}



\end{document}
