\documentclass[a4paper,11pt]{article}

\usepackage{a4wide}
\usepackage{amsmath}
\usepackage{amssymb}
\usepackage[utf8]{inputenc}
\usepackage{theorem}

\newcommand{\set}[1]{\{#1\}}
\newcommand{\such}{\mid}

%%% Blackboard bold letters
\newcommand{\NN}{\mathbb{N}}
\newcommand{\NNx}{{\NN^{+}}}
\newcommand{\ZZ}{\mathbb{Z}}
\newcommand{\QQ}{\mathbb{Q}}
\newcommand{\RR}{\mathbb{R}}
\newcommand{\CC}{\mathbb{C}}

%%% quantifiers
\newcommand{\all}[3]{\forall\, #1 \,{\in}\, #2\,.\left(#3\right)}
\newcommand{\some}[3]{\exists\, #1 \,{\in}\, #2\,.\left(#3\right)}
\newcommand{\exactlyone}[3]{\exists!\, #1 \,{\in}\, #2\,.\left(#3\right)}
\newcommand{\lam}[3]{\lambda #1 \,{\in}\, #2\,.\left(#3\right)}
\newcommand{\uall}[2]{\forall\, #1\,.\left(#2\right)}
\newcommand{\usome}[2]{\exists\, #1\,.\left(#2\right)}
\newcommand{\uexactlyone}[3]{\exists!\, #1\,.\left(#2\right)}
\newcommand{\ulam}[2]{\lambda #1 .\left(#2\right)}
\newcommand{\xall}[3]{\forall\, #1 \,{\in}\, #2\,.\,#3}
\newcommand{\xsome}[3]{\exists\, #1 \,{\in}\, #2\,.\,#3}
\newcommand{\xexactlyone}[3]{\exists!\, #1 \,{\in}\, #2\,.\,#3}
\newcommand{\xuall}[2]{\forall\, #1\,.\,#2}
\newcommand{\xusome}[2]{\exists\, #1\,.\,#2}
\newcommand{\xuexactlyone}[2]{\exists!\, #1,.\,#2}
\newcommand{\xlam}[3]{\lambda #1 \,{\in}\, #2\,.\,#3}
\newcommand{\xulam}[2]{\lambda #1 .\,#2}
\newcommand{\tlam}[3]{\lambda #1 \,{:}\, #2\,.\,\left(#3\right)}
\newcommand{\xtlam}[3]{\lambda #1 \,{:}\, #2\,.\,#3}


\begin{document}

\title{Izračunljivost v topologiji}
\author{Homework 2}
\date{Due date: April 9th, 2009}

\maketitle

\subsection*{Instructions}

You are \emph{encouraged} to use the literature and research papers,
but any mathematical text that your solution relies on must come from
a peer-reviewed source (published book or journal article).

If you use information from Wikipedia or random PDF files on the
internet, then the correctness of the source becomes \emph{your}
responsibility, i.e., you should personally verify the proofs or write
them if they are missing.

Thus you may solve a problem, or part of a problem, by finding the
solution in a book or journal article. In this case you may simply
refer to the text you found (there is no need to copy it). If a
problem asks for a proof, you must refer to an actual proof (not a
paper which claims that the proof is trivial or left as exercise).

\subsection*{Problem 1}

Suppose $\AA$ is a TPCA and $\compAA$ a sub-TPCA. Let $t$ be a type
and $r_0 \in \compAtyp{t}$. Define the functor $\nabla' : \Set \to
\AsmA$ as follows. Given a set $X$, let $\nabla' X = (X, \Atyp{t},
{\rz_{\nabla' X}})$ with
%
\begin{equation*}
  r \rz_{\nabla' X} x \iff r = r_0.
\end{equation*}
%
A function $f : X \to Y$ is mapped to $\nabla' f = f$, which is
obviously realized.

Recall that we defined $\nabla X = (X, \Atyp{t}, {\rz_{\nabla X}})$
where $r \rz_{\nabla X} x$ for all $r \in \Atyp{t}$ and $x \in X$.

\begin{enumerate}
\item State the definition of ``functors $F$ and $G$ are naturally
  isomorphic''.
\item Verify that $\nabla$ and $\nabla'$ are naturally isomorphic.
\item Now suppose we picked $r_0 \in \Atyp{t}$ so that $r_0 \not\in
  \compAtyp{t}$. What would go wrong?
\end{enumerate}


\subsection*{Problem 2}

\begin{enumerate}
\item Prove that in any category with binary products $A \times B
  \cong B \times A$.
\item 
  Prove that in $\AsmA$ the following distributive law holds:
  % 
  \begin{equation*}
    (\asm{S} + \asm{T}) \times \asm{U} \cong
    (\asm{S} \times \asm{U}) + (\asm{S} \times \asm{U}).
  \end{equation*}
\item Find a category which has binary products and coproducts, but
  the distributive law does not hold.
\end{enumerate}

\end{document}
