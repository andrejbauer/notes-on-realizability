\documentclass[a4paper,11pt]{article}

\usepackage{a4wide}
\usepackage{amsmath}
\usepackage{amssymb}
\usepackage[utf8]{inputenc}
\usepackage{theorem}

%\newenvironment{proof}{\medskip\noindent\emph{Proof.}}{\hfill$\Box$\medskip}

\newcommand{\defemph}[1]{\emph{\textbf{#1}}}

%%% Blackboard bold letters
\newcommand{\NN}{\mathbb{N}}
\newcommand{\NNx}{{\NN^{+}}}
\newcommand{\ZZ}{\mathbb{Z}}
\newcommand{\QQ}{\mathbb{Q}}
\newcommand{\RR}{\mathbb{R}}
\newcommand{\CC}{\mathbb{C}}

%%% PCAs
\renewcommand{\AA}{\mathbb{A}}
\newcommand{\subAA}{\mathbb{A}'}
\newcommand{\EE}{{\mathbb{E}}}
\newcommand{\subEE}{{\mathbb{E}'}}
\newcommand{\FF}{{\mathbb{F}}}
\newcommand{\subFF}{{\mathbb{F}'}}
\newcommand{\GG}{{\mathbb{G}}}
\newcommand{\subGG}{{\mathbb{G}'}}

\newcommand{\klone}{\mathbb{K}_1}
\newcommand{\UU}{\mathbb{U}}

\newcommand{\CL}{\mathbb{CL}}

%%% Applicative morphisms
\newcommand{\ff}[1]{\widehat{#1}}  % functor induced by an applicative morphism



%%% quantifiers
\newcommand{\all}[1]{\forall #1 .\,}
\newcommand{\some}[1]{\exists #1 .\,}
\newcommand{\lam}[2]{\lambda #1 .\,#2}
\newcommand{\of}{{:}}

%% Grammar
\newcommand{\bnfis}{\mathbin{{:}{:}{=}}}
\newcommand{\bnfor}{\mid}

%%% Substitution
\newcommand{\subst}[2]{#1[#2]}

%%% Arrows
\newcommand{\subto}{{\shortrightarrow}} % for substitution
\newcommand{\oneto}{\mapsto}
\newcommand{\manyto}{\oneto^{*}}
\newcommand{\multito}{\rightrightarrows}
\newcommand{\parto}{\mathbin{\rightharpoonup}}
\newcommand{\epito}{\twoheadrightarrow}
\newcommand{\into}{\hookrightarrow}
\newcommand{\monoto}{\rightarrowtail}
\newcommand{\natto}{\Rightarrow} % natural transformation

\newcommand{\curry}[1]{\hat{#1}}
\newcommand{\uncurry}[1]{\check{#1}}

%%% Sets
\newcommand{\set}[1]{\{#1\}}
\newcommand{\such}{\mid}
\newcommand{\pow}[1]{\mathcal{P}(#1)}

\newcommand{\im}[1]{\mathrm{im}(#1)}

\newcommand{\tbigcup}{\bigcup\nolimits}
\newcommand{\tbigcap}{\bigcap\nolimits}

\newcommand{\dsum}{\Sigma}
\newcommand{\dprod}{\Pi}

\newcommand{\zero}{\mathsf{0}}
\newcommand{\one}{\mathsf{1}}
\newcommand{\two}{\mathsf{2}}

\newcommand{\NNNN}{\NN^{\NN}}
\newcommand{\Sierpinski}{\mathbb{S}}
\newcommand{\Baire}{\mathbb{B}}
\newcommand{\Cantor}{\two^{\NN}}
\newcommand{\Scott}{\mathbb{P}}

%%% Topology
\newcommand{\topol}[1]{\mathcal{O}(#1)}

%%% Functions
\newcommand{\id}[1][]{\mathrm{id}_{#1}}

\newcommand{\dom}[1]{\mathsf{dom}(#1)}
\newcommand{\invim}[1]{#1^{*}}
\newcommand{\inv}[1]{{#1}^{-1}}

\newcommand{\defined}[1]{#1{\downarrow}}
\newcommand{\divergent}[1]{#1{\uparrow}}
\newcommand{\place}{{-}}

\newcommand{\restrict}[2]{#1{\restriction}_{#2}}

%%% Pairing
\newcommand{\pair}[1]{\langle #1 \rangle}
\newcommand{\xfst}{\mathtt{fst}}
\newcommand{\fst}[1]{\xfst,#1}
\newcommand{\xsnd}{\mathtt{snd}}
\newcommand{\snd}[1]{\xsnd,#1}

%%% Coding
\newcommand{\code}[1]{\ulcorner #1 \urcorner}

%%% Standard enumerations
\newcommand{\enumstage}[2]{#1{\mid}_{#2}}

\newcommand{\xpr}{\text{\boldmath{$\varphi$}}}
\newcommand{\pr}[2]{\xpr_{#1}(#2)}
\newcommand{\prm}[3]{\xpr^{(#1)}_{#2}(#3)}

\newcommand{\iitm}[1]{\text{\boldmath{$\psi$}}_{#1}}

\newcommand{\xfpr}{\text{\boldmath{$\eta$}}}
\newcommand{\fpr}[2]{\xfpr_{#1}(#2)}
\newcommand{\fprm}[3]{\xfpr^{(#1)}_{#2}(#3)}

\newcommand{\cons}[2]{#1 {:}{:} #2}
\newcommand{\append}[2]{#1 \mathbin{{+}\!\!{+}} #2}
\newcommand{\basicBB}[1]{#1{{+}\!\!{+}}\Baire}
\newcommand{\seq}[1]{[#1]}
\newcommand{\seg}[2]{\overline{#1}(#2)}

%%% Lambda calculus
\newcommand{\unit}{\mathtt{unit}}
\newcommand{\ttunit}{{\star}}
\newcommand{\ttfst}[1]{\mathtt{fst}\,#1}
\newcommand{\ttsnd}[1]{\mathtt{snd}\,#1}
\newcommand{\FV}[1]{\mathsf{FV}(#1)}
\newcommand{\ttnat}{\mathtt{nat}}
\newcommand{\ttbool}{\mathtt{bool}}

%% Denotational semantics
\newcommand{\sem}[1]{[\![#1]\!]}

%% Logic
\newcommand{\ctx}{\mid}

\newcommand{\lthen}{\Rightarrow}
\newcommand{\liff}{\Leftrightarrow}


% Axiom
\newcommand{\axiom}[1]{\dfrac{}{#1}}

% Axiom with a side condition
\newcommand{\axiomd}[2]{\dfrac{}{#1} \; #2}

% Inference rule
\newcommand{\infer}[2]{\begin{gathered}\dfrac{#1}{#2}\end{gathered}}
\newcommand{\inferr}[3]{\begin{gathered}\dfrac{#1\quad #2}{#3}\end{gathered}}
\newcommand{\inferrr}[4]{\begin{gathered}\dfrac{#1\quad #2 \quad #3}{#4}\end{gathered}}

\newcommand{\sep}{\qquad}
\newcommand{\fromassumption}[2]{
  \begin{gathered}[b]
    {\displaystyle #1} \\
    \vdots \\
    {#2}
  \end{gathered}}

% Inference rule with a side condition
\newcommand{\inferd}[3]{\begin{gathered}\dfrac{#1}{#2} \; #3\end{gathered}}

%%% Domain theory
\newcommand{\upper}[1]{{\uparrow}#1}
\newcommand{\wayb}{\ll}

%%%% PCAs
\renewcommand{\AA}{\mathbb{A}}
%\newcommand{\compAA}{\comp{\AA}}

\newcommand{\pcalam}[1]{\langle #1 \rangle\,}
\newcommand{\annot}[2]{#1^{#2}}
\newcommand{\tpcalam}[2]{\langle \annot{#1}{#2} \rangle\,}
\newcommand{\kleq}{\simeq}
\newcommand{\klgeq}{\succeq}
\newcommand{\numeral}[1]{\overline{#1}}
\newcommand{\JJ}{\mathbb{J}}

\newcommand{\pcacomb}[1]{\mathtt{#1}}

%\newcommand{\pcato}{\stackrel{\scriptscriptstyle\mathsf{PCA}}{\longrightarrow}}
\newcommand{\pcato}{\xrightarrow{\scriptscriptstyle\mathsf{pca}}{}}

\newcommand{\combK}{\pcacomb{K}}
\newcommand{\combS}{\pcacomb{S}}
\newcommand{\combI}{\pcacomb{I}}

\newcommand{\combY}{\pcacomb{Y}}
\newcommand{\combZ}{\pcacomb{Z}}
\newcommand{\combW}{\pcacomb{W}}
\newcommand{\combFix}{\pcacomb{fix}}

\newcommand{\combPair}{\pcacomb{pair}}
\newcommand{\combFst}{\pcacomb{fst}}
\newcommand{\combSnd}{\pcacomb{snd}}

\newcommand{\combLeft}{\pcacomb{left}}
\newcommand{\combRight}{\pcacomb{right}}
\newcommand{\combCase}{\pcacomb{case}}

\newcommand{\combSucc}{\pcacomb{succ}}
\newcommand{\combPred}{\pcacomb{pred}}

\newcommand{\combRec}{\pcacomb{rec}}
\newcommand{\combMin}{\pcacomb{min}}

\newcommand{\combIf}{\pcacomb{if}}
\newcommand{\combTrue}{\pcacomb{true}}
\newcommand{\combFalse}{\pcacomb{false}}
\newcommand{\combIsZero}{\pcacomb{iszero}}
\newcommand{\cond}[3]{\mathtt{if}\,#1\,\mathtt{then}\,#2\,\mathtt{else}\,#3}

\newcommand{\case}[5]{\mathtt{case}\,#1\,\mathtt{of}\,#2 \mapsto #3 \mid #4 \mapsto #5}
\newcommand{\xcase}[5]{\begin{aligned}[t]\mathtt{case}\,&#1\;\mathtt{of}\\&#2 \mapsto #3 \\&#4 \mapsto #5\end{aligned}}


% PCF
\newcommand{\PCF}{\mathsf{PCF}}
\newcommand{\PCFinf}{\mathsf{PCF}^\infty}


% Realizability
\newcommand{\comp}[1]{#1_{\#}}
\newcommand{\rz}[1][]{\Vdash_{#1}}
\newcommand{\Ex}[1][]{\mathsf{E}}
\newcommand{\per}{\approx}

\newcommand{\typ}[2]{#1_{#2}}
\newcommand{\Atyp}[1]{\typ{\AA}{#1}}
\newcommand{\xAtyp}[1]{\typ{\AA}{|#1|}}
\newcommand{\subAtyp}[1]{\typ{\subAA}{#1}}

\newcommand{\effsym}{\#}
\newcommand{\eff}[1]{\effsym #1}

\newcommand{\R}[1]{\mathtt{#1}} % realizer
\renewcommand{\S}[1]{|#1|} % underlying set 
\newcommand{\T}[1]{\|#1\|} % underlying type
\newcommand{\xasm}[1]{(\S{#1|}, \T{#1}, {\rz[#1]})}
%\newcommand{\asm}[1]{#1}

\newcommand{\rep}[1]{\mathbf{#1}}
\newcommand{\xrep}[1]{(#1, \delta_{#1})}

% Categories

\newcommand{\cat}[1]{\mathcal{#1}}
\newcommand{\Hom}[1]{\mathsf{Hom}(#1)}

\newcommand{\Sub}[1]{\mathsf{Sub}(#1)}
\newcommand{\Mono}[1]{\mathsf{Mono}(#1)}
\newcommand{\Pred}[1]{\mathsf{Pred}(#1)}

%\newcommand{\subcat}[1]{\mathsf{Subset}(#1)}

\newcommand{\Set}{\mathsf{Set}}
\newcommand{\Asm}[1]{\mathsf{Asm}(#1)}
\newcommand{\AsmA}{\Asm{\AA,\subAA}}

\newcommand{\Mod}[1]{\mathsf{Mod}(#1)}
\newcommand{\ModA}{\Mod{\AA,\subAA}}

\newcommand{\Rep}[1]{\mathsf{Rep}(#1)}
\newcommand{\Per}[1]{\mathsf{Per}(#1)}
\newcommand{\Er}[1]{\mathsf{Er}(#1)}

\newcommand{\wTop}{\mathsf{\omega Top}}
\newcommand{\compTop}{\comp{\wTop}}
\newcommand{\Equ}{\mathsf{Equ}}

\newcommand{\CanProj}[1]{\mathsf{Proj}(#1)}

% Adjunctions

% Adjunction as a two-way rule
\newcommand{\adjunction}[2]{%
  \begin{tabular}{c}
    $#1$ \\
    \noalign{
      \vskip 2pt      
      \hrule
      \vskip 1pt      
      \hrule
      \vskip 2pt      
      }
    $#2$
  \end{tabular}
  }

\newcommand{\adjunctionx}[3]{%
  \begin{tabular}{c}
    $#1$ \\
    \noalign{
      \vskip 2pt      
      \hrule
      \vskip 1pt
      \hrule
      \vskip 2pt      
      }
    $#2$ \\
    \noalign{
      \vskip 2pt      
      \hrule
      \vskip 1pt
      \hrule
      \vskip 2pt      
      }
    $#3$
  \end{tabular}
  }

\newcommand{\adjrule}{\noalign{\vskip 2pt \hrule \vskip 1pt \hrule \vskip 2pt}}

\newcommand{\longadjunction}[1]{
\begin{tabular}{>{$}c<{$}}
#1
\end{tabular}
}



%%% Local Variables: 
%%% mode: latex
%%% TeX-master: "notes-on-realizability"
%%% End: 


\begin{document}

\title{Izračunljivost v topologiji}
\author{Homework 3}
\date{}

\maketitle


\subsection*{Instructions}

You are \emph{encouraged} to use the literature and research papers,
but any mathematical text that your solution relies on must come from
a peer-reviewed source (published book or journal article).
%
If you use information from Wikipedia or random PDF files on the
internet, then the correctness of the source becomes \emph{your}
responsibility, i.e., you should personally verify the proofs or write
them if they are missing.
%
Thus you may solve a problem, or part of a problem, by finding the
solution in a book or journal article. In this case you may simply
refer to the text you found (there is no need to copy it). If a
problem asks for a proof, you must refer to an actual proof (not a
paper which claims that the proof is trivial or left as exercise).

I prefer a hand-written solution to one written with Microsoft Word in
which mathematics is messed up. You may but do not have to use
{\LaTeX}.
%
If the problems are too hard, come to me and we will discuss them.


\subsection*{Problem 1}

Let $\asm{S}$ be an assembly and $p \in \Pred{\asm{S} \times \asm{S}}$
a realizability predicate on $\asm{S} \times \asm{S}$. In your
favorite programming language (a functional programming language such
as Haskell or Ocaml might work well), write a realizer for the
statement known as ``dependent choice'':
%
\begin{equation*}
  (\xall{x}{\asm{S}}{\xsome{y}{\asm{S}}{p(x,y)}}) \implies
  \xall{z}{\asm{S}}{
    \some{f}{\asm{S}^{\asm{N}}}{
      f(0) = z \land \xall{n}{\asm{N}}{p(f(n), f(n+1))}
    }
  }.
\end{equation*}
%
Explain what your program does. How does the program simplify if we
assume that $p$ is $\lnot\lnot$-stable?


\subsection*{Problem 2}

In this problem we work in $\Asm{K_1}$ where $K_1$ is the first Kleene
algebra, i.e., we work with type 1 Turing machines. Recall the
definitions of standard assemblies for booleans $\two$, natural
numbers $\asm{N}$, integers $\asm{Z}$, and real numbers $\asm{R}$:
%
\begin{itemize}
\item $\two = (\set{0,1}, {\rz_2})$ where $0 \rz_2 0$ and
  $1 \rz_2 1$,
\item $\asm{N} = (\NN, {\rz_{\NN}})$ where $n \rz_\NN n$ for every $n
  \in \NN$,
\item $\asm{Z} = (\ZZ, {\rz_\ZZ})$ where $\pair{m,n} \rz_\ZZ k$ if $k
  = m - n$ for all $m, n \in \NN$ and $k \in \ZZ$,
\item $\asm{R} = (R, {\rz_R})$ where, for $n \in \NN$ and $x \in \RR$,
  $n \rz_R x$ if, and only if, for every $k \in \NN$ there are $a, b,
  c \in \NN$ such that $\pr{n}{k} = \pair{\pair{a, b}, c}$ and $|x -
  (a - b)/c| \leq 2^{-k}$. In other words, a real number~$x$ is
  represented by a Turing machine which accepts a number~$k$ and
  outputs a rational number $q = (a-b)/c$ such that $|x - q| \leq
  2^{-k}$. The set $R = \set{x \in \RR \such \xsome{n}{\NN}{n \rz_R
      x}}$ consists of those real numbers that have such a realizer.
\end{itemize}
%
For each of the following statements determine whether it is realized
in $\Asm{K_1}$:
%
\begin{enumerate}
\item $\all{f}{\two^\NN}{(\xall{n}{\asm{N}}{f(n)=0}) \lor
    (\xsome{n}{\asm{N}}{f(n) = 1})}$,
\item $\all{x}{\asm{R}}{x \leq 0 \lor 0 < x}$,
\item $\all{x}{\asm{R}}{x \neq 0 \implies x < 0 \lor 0 < x}$,
\item $\xall{x}{\asm{R}}{\some{k}{\asm{Z}}{k \leq x \land x < k + 1}}$,
\item $\xall{x}{\asm{R}}{\some{k}{\asm{Z}}{k \leq x \land x < k + 2}}$.
\end{enumerate}
%

\subsection*{Problem 3}

Prove the following statement in synthetic topology:
%
\begin{enumerate}
\item If $\asm{X}$ is compact and $\asm{Y}$ discrete then
  $\asm{Y}^\asm{X}$ is discrete.
\item A compact subspace of a Hausdorff space is closed.
\item Suppose $\asm{M}$ is an assembly and $d : \asm{M} \times \asm{M}
  \to \asm{R}$ a realized map such that $(M, d)$ is a metric space.
  Show that, for all $x \in M$ and $r \in R$, the ``open'' ball
  $B(x,r) = \set{y \in M \such d(x,y) < r}$ is intrinsically open
  in~$\asm{M}$. (You may use the fact that $<$ is an open relation on
  $\asm{R}$.)
\end{enumerate}



\end{document}
