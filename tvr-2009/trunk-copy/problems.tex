\documentclass[a4paper]{article}

\usepackage[utf8]{inputenc}
\usepackage{hyperref}

\begin{document}

\title{Topologija v računalništvu -- programerske naloge}
\author{Andrej Bauer}

\maketitle

Spodaj so opisane programerske naloge za drugi del predmeta Topologija
v računalništvu. Vsak si izbere eno nalogo, kdor prej pride prej
melje. Spodnji opisi so le informativne narave, ko boste izbrali
nalogo, se pridete k meni pogovorit o podrobnostih.

Nekatere naloge so preobsežne za eno osebo in jih lahko skupaj
rešujeta dva.

\section*{Naloga 1: MPFR (za eno ali dve osebi)}

Implementacija realnih števil v Haskellu sloni na diadičnih ulomkih iz
modula \texttt{Dyadic}, ki so implementirani neučinkovito. V tej
nalogi je treba modul \texttt{Dyadic} zamenjati z implementacijo, ki
uporablja knjižnico MPFR. Za Haskell že obstaja vmesnik do knjižnice.

Knjižnica MPFR ima tudi implementirane osnovne elementarne funkcije
(trigonometrične, logaritemske itn.), zato bi lahko v sklopu te naloge
dodatno implementirali še elementarne funkcije na realnih številih.

Reference:
%
\begin{itemize}
\item Knjižnica MPFR: \url{http://www.mpfr.org/}
\item Haskell interface za MPFR: \url{http://hackage.haskell.org/package/hmpfr}
\item Elementarne funkcije v Haskellu, glej \texttt{class Floating},
  npr. \url{http://www.zvon.org/other/haskell/Outputprelude/Floating_c.html}
\end{itemize}


\section*{Naloga 2: Eksistenčni in univerzalni kvantifikatorji}

Po zgledu univerzalenga kvantifikatorja na zaprtem intervalu (modul
\texttt{Reals}), implementiraj še eksistenčni kvantifikator za:
%
\begin{enumerate}
\item zaprti interval
\item odprti interval
\item realna števila
\end{enumerate}
%
Dodatno lahko implementiramo še nekatere druge kvantifikatorje:
%
\begin{enumerate}
\item Ker je slika kompaktne podmnožice kompaktna, lahko
  implementiramo $\forall$ za sliko kompaktne podmnožice.
\item Ker je slika odkrite podmnožice odkrita, lahko
  implementiramo $\exists$ za sliko odkrite podmnožice.
\item Produkt kompaktnih prostorov je kompakten.
\item Produkt odkritih prostorov je odkrit.
\end{enumerate}

\section*{Naloga 3: Manjkajoča struktura na realnih številih}

Sedanja implementacija realnih števil še ni popolna, saj manjka
operator \texttt{lim}, ki izračuna limito Caucheyevega zaporedja. Prav
tako manjka operator \texttt{approx\_to}, ki izračuna približek
realnega števila na dano natančnost.

Ko implementiramo \texttt{lim}, lahko iz njega izpeljemo še nekatere
druge operatorje, na primer operator, ki izračuna vrednost neskončne
vrste.

Podani naj bodo tudi primeri limit, recimo za $e$ in $\pi$.

\section*{Naloga 4: Boljši prikaz realnih števil na zaslonu}

Implementirati bi bilo treba boljši prikaz diadičnih števil na
zaslonu, kakor tudi prikaz realnih števil. Pričakujem, da boste
implementirali instanco razreda \texttt{Show} za \texttt{Dyadic},
\texttt{Interval} in \texttt{RealNum}, in to na ``inteligenten''
način.

Dodatno bi bilo dobro implementirati tudi instanco razreda
\texttt{Read}, da bomo lahko diadična števila tudi prebrali z
datoteke.

Dodatno bi bilo dobro implementirai tudi instanco za implementacijo,
ki uporablja MPFR (MPFR že ima vgrajene funkcije za pretvorbo števil
iz in v stringe, tako da s tem ne bi smelo biti veliko dela).

Referenca:
%
\begin{enumerate}
\item \url{ls.fi.upm.es/~jjmoreno/prog_dec/haskell_EN_read_show.pdf}
\end{enumerate}

\section*{Naloga 5: Risanje grafov funkcij}

Za propagandne namene je koristno imeti demo, ki zna narisati kako
slikico. V tej nalogi bi implementirali preprost demo, ki nariše graf
funkcije na danem intervalu. Demo izračuna graf tako natančno, da je
prikazani graf funkcije zagotovo pravilen. Ker ima lahko funkcija
pole in območja, kjer ni definirana, mora program inteligentno risati
sliko tako, da jo postopoma izboljšuje.

Pričakujem, da bo demo deloval še kje druge kot na Microsoft Windows.
Izbiro knjižnice in metode za risanje prepuščam kandidatu. Lahko na
primer napiše demo, ki generira SVG ali kak drug standardni format za
2D grafiko in ne prikazuje slik neposredno na zaslonu. Seveda je
programček, ki ga lahko ineraktivno uporabljamo, bolj zabaven.

Referenca: grafika v Haskellu \url{http://www.haskell.org/haskellwiki/Libraries_and_tools/Graphics}.

\section*{Benchmarki in testi}

Da bi lahko ocenili kvaliteto implementacije realnih števil,
potrebujemo smiselne teste, s katerimi preverjamo pravilnost
implementacije, in benchmarke, s katerimi merimo hitrost.

Naloga zajemo implementacijo osnovnih testov in benchmarkov. Ti naj
bodo enostavni za uporabo. Dodatno lahko uporabite profiler in
ugotovite, katere dele kode se najbolj splača optimizirati.

Reference:
%
\begin{enumerate}
\item Haskell unit testing: \url{http://hunit.sourceforge.net/}
\item GHC profiler: \url{http://www.haskell.org/ghc/docs/latest/html/users_guide/profiling.html}
\end{enumerate}

\end{document}
