\chapter{Models of Computation\label{cha:models}}


In this chapter we review basic notions of computability theory. We
start with the notions of Turing machine and computable function and
prove their basic properties. We then consider other models of
computability, such as Turing machines with infinite input and output,
Scott's graph model, and the general notion of partial combinatory
algebras.


\section{Turing machines}
\label{sec:turing-machines}

A \emph{Turing machine} is a given by:
%
\begin{enumerate}
\item A finite \emph{alphabet} of symbols. We assume that there is a
  special symbol `blank' and at least two other symbols, say $0$ and
  $1$. We assume that the letters $L$ and $R$ are \emph{not} among the
  symbols.
\item A \emph{tape}, which is an infinite succession of \emph{cells}.
  Each cell holds a symbol from the alphabet. A cell is said to be
  \emph{empty} if it holds the blank symbol.
\item A \emph{head} which moves along the tape, reads and writes
  the contents of the cells.
\item A finite set of \emph{states} and a distinguished \emph{starting
    state}.
\item A \emph{program}, which is a finite list of \emph{instructions},
  whicha are quadruples of the form
  %
  \begin{equation*}
    q \, s \, u \, r
  \end{equation*}
  %
  where $q$ and $q'$ are states, $s$ is a symbol, and $u$ is either
  the letter $L$, the letter $R$, or a symbol from the alphabet. The
  informal meaning of an instruction is ``if in state $q$ and the
  symbol under the head is $s$, then perform $u$ and go to state
  $r$''. Here ``perform $u$'' means ``move the head to the left'' if
  $u = L$, ``move the head to the right if $u = R$'', and ``write
  symbol $u$ to the cell under the head'' if $u$ is a symbol.
\end{enumerate}

 carries out basic steps, one at a time, as described by a
finite set of instructions, called the \emph{program}. The machine is
equipped with a tape consisting of cells that hold


 and a read/write head which can
move along the tape, in both directions, and


Informal description.

Formal description

Enumeration of Turing machines.

Partial computable functions.

Halting oracle.

s-m-n

u-t-m

\subsection{Robustness of computability}
\label{sec:robustness}

Equivalence with RAM machines and other forms.


\subsection{Type 1 and type 2 machines}
\label{sec:type-1-2}


\section{The graph model}
\label{sec:graph-model}


\section{Partial combinatory algebras}
\label{sec:pcas}

\subsection{$\lambda$-calculus}
\label{sec:lambda-calculus}




\section{Real-world programming languages}
\label{sec:programming-languages}


\section{Comparison of models of computation}
\label{sec:models-comparison}




%%% Local Variables: 
%%% mode: latex
%%% TeX-master: "notes"
%%% End: 
