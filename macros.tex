%%%%%%%%%%%%%%%%%%%%%%%%%%%%%%%%%%%%%%%%%%%%%%%%%%
%
% Macros for indexing

\newcommand{\footstyle}[1]{{#1}n}
\newcommand{\defstyle}[1]{\textsl{#1}}

\newcommand{\indexdef}[1]{\index{#1|defstyle}}%
\newcommand{\indexfoot}[1]{\index{#1|footstyle}}%
\newcommand{\indexsee}[2]{\index{#1|see{#2}}}%

%%% Theorems
\newtheorem{theorem}{Theorem}[chapter]
\newtheorem{proposition}[theorem]{Proposition}
\newtheorem{lemma}[theorem]{Lemma}
\newtheorem{corollary}[theorem]{Corollary}
\newtheorem{definition}[theorem]{Definition}
\newtheorem{exercise}[theorem]{Exercise}

\newenvironment{proof}{\medskip\noindent\emph{Proof.}}{\hfill$\Box$\medskip}

%%% Blackboard bold letters
\newcommand{\NN}{\mathbb{N}}
\newcommand{\NNx}{{\NN^{+}}}
\newcommand{\ZZ}{\mathbb{Z}}
\newcommand{\QQ}{\mathbb{Q}}
\newcommand{\RR}{\mathbb{R}}
\newcommand{\CC}{\mathbb{C}}

%%% quantifiers
\newcommand{\all}[3]{\forall\, #1 \,{\in}\, #2\,.\left(#3\right)}
\newcommand{\some}[3]{\exists\, #1 \,{\in}\, #2\,.\left(#3\right)}
\newcommand{\exactlyone}[3]{\exists!\, #1 \,{\in}\, #2\,.\left(#3\right)}
\newcommand{\lam}[3]{\lambda #1 \,{\in}\, #2\,.\left(#3\right)}
\newcommand{\uall}[2]{\forall\, #1\,.\left(#2\right)}
\newcommand{\usome}[2]{\exists\, #1\,.\left(#2\right)}
\newcommand{\uexactlyone}[3]{\exists!\, #1\,.\left(#2\right)}
\newcommand{\ulam}[2]{\lambda #1 .\left(#2\right)}
\newcommand{\xall}[3]{\forall\, #1 \,{\in}\, #2\,.\,#3}
\newcommand{\xsome}[3]{\exists\, #1 \,{\in}\, #2\,.\,#3}
\newcommand{\xexactlyone}[3]{\exists!\, #1 \,{\in}\, #2\,.\,#3}
\newcommand{\xuall}[2]{\forall\, #1\,.\,#2}
\newcommand{\xusome}[2]{\exists\, #1\,.\,#2}
\newcommand{\xuexactlyone}[2]{\exists!\, #1,.\,#2}
\newcommand{\xlam}[3]{\lambda #1 \,{\in}\, #2\,.\,#3}
\newcommand{\xulam}[2]{\lambda #1 .\,#2}
\newcommand{\tlam}[3]{\lambda #1 \,{:}\, #2\,.\,\left(#3\right)}
\newcommand{\xtlam}[3]{\lambda #1 \,{:}\, #2\,.\,#3}

%% Grammar
\newcommand{\bnfis}{\mathbin{{:}{:}{=}}}
\newcommand{\bnfor}{\mid}

%%% Substitution
\newcommand{\subst}[2]{#1[#2]}

%%% Sets
\newcommand{\set}[1]{\{#1\}}
\newcommand{\such}{\mid}
\newcommand{\pow}[1]{\mathcal{P}(#1)}

\newcommand{\tbigcup}{\bigcup\nolimits}
\newcommand{\tbigcap}{\bigcap\nolimits}

\newcommand{\two}{\mathsf{2}}
\newcommand{\Baire}{\mathbb{B}}
\newcommand{\cBaire}{\Baire_{\#}}
\newcommand{\Scott}{\mathbb{P}}
\newcommand{\cScott}{\Scott_{\#}}

%%% Functions
\newcommand{\parto}{\mathbin{\rightharpoonup}}
\newcommand{\dom}[1]{\mathsf{dom}(#1)}
\newcommand{\invim}[1]{{#1}^{-1}}

\newcommand{\defined}{\,{\downarrow}}
\newcommand{\divergent}{\,{\uparrow}}
\newcommand{\place}{{-}}

%%% Pairing
\newcommand{\pair}[1]{\langle #1 \rangle}
\newcommand{\xfst}{\mathtt{fst}}
\newcommand{\fst}[1]{\xfst,#1}
\newcommand{\xsnd}{\mathtt{snd}}
\newcommand{\snd}[1]{\xsnd,#1}

%%% Coding
\newcommand{\code}[1]{\ulcorner #1 \urcorner}

%%% Standard enumerations
\newcommand{\xpr}{\text{\boldmath{$\varphi$}}}
\newcommand{\pr}[2]{\xpr_{#1}(#2)}
\newcommand{\prm}[3]{\xpr^{(#1)}_{#2}(#3)}

\newcommand{\xfpr}{\text{\boldmath{$\eta$}}}
\newcommand{\fpr}[2]{\xfpr_{#1}(#2)}
\newcommand{\fprm}[3]{\xfpr^{(#1)}_{#2}(#3)}

\newcommand{\cons}[2]{#1 {:} #2}
\newcommand{\append}[2]{#1 \mathbin{{+}\!\!{+}} #2}
\newcommand{\seq}[1]{[#1]}
\newcommand{\seg}[2]{\overline{#1}(#2)}

%%% Lambda calculus
\newcommand{\unit}{\mathtt{unit}}
\newcommand{\ttunit}{{*}}
\newcommand{\ttfst}[1]{\mathtt{fst}\,#1}
\newcommand{\ttsnd}[1]{\mathtt{snd}\,#1}
\newcommand{\FV}[1]{\mathsf{FV}(#1)}

%% Logic
\newcommand{\ctx}{\mid}

% Axiom
\newcommand{\axiom}[1]{\dfrac{}{#1}}

% Axiom with a side condition
\newcommand{\axiomd}[2]{\dfrac{}{#1} \; #2}

% Inference rule
\newcommand{\infer}[2]{\begin{gathered}\dfrac{#1}{#2}\end{gathered}}
\newcommand{\inferr}[3]{\begin{gathered}\dfrac{#1\quad #2}{#3}\end{gathered}}
\newcommand{\inferrr}[4]{\begin{gathered}\dfrac{#1\quad #2 \quad #3}{#4}\end{gathered}}

\newcommand{\sep}{\qquad}
\newcommand{\fromassumption}[2]{
  \begin{gathered}[b]
    {\displaystyle #1} \\
    \vdots \\
    {#2}
  \end{gathered}}

% Inference rule with a side condition
\newcommand{\inferd}[3]{\begin{gathered}\dfrac{#1}{#2} \; #3\end{gathered}}

%%% Domain theory
\newcommand{\upper}[1]{{\uparrow}#1}
\newcommand{\wayb}{\ll}
\newcommand{\UU}{\mathbb{U}}

%%%% PCAs
\newcommand{\pcalam}[2]{\lambda^{*} #1 .\,#2}
\newcommand{\kleq}{\simeq}

\newcommand{\pcacomb}[1]{\mathtt{#1}}

%\newcommand{\pcato}{\stackrel{\scriptscriptstyle\mathsf{PCA}}{\longrightarrow}}
\newcommand{\pcato}{\xrightarrow{\scriptscriptstyle\mathsf{\ PCA\ }}{}}

\newcommand{\combK}{\pcacomb{K}}
\newcommand{\combS}{\pcacomb{S}}
\newcommand{\combI}{\pcacomb{I}}

\newcommand{\combY}{\pcacomb{Y}}
\newcommand{\combZ}{\pcacomb{Z}}
\newcommand{\combW}{\pcacomb{W}}

\newcommand{\combPair}{\pcacomb{pair}\,}
\newcommand{\combFst}{\pcacomb{fst}\,}
\newcommand{\combSnd}{\pcacomb{snd}\,}

\newcommand{\combIf}{\pcacomb{if}}
\newcommand{\combTrue}{\pcacomb{true}}
\newcommand{\combFalse}{\pcacomb{false}}



%%% Local Variables: 
%%% mode: latex
%%% TeX-master: "notes"
%%% End: 
