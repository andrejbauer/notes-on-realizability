\chapter{Background material}
\label{chap:background-material}

In this chapter we overview a selection of concepts which we need
later on. We also fix notation and a number of definitions. At the
momement the sections are not listed in any particular order.

\paragraph{Free and bound variables.}

Occurrences of variables in an expression may be \emph{free} or
\emph{bound}. Variables are bound when they are used to indicate the
range over which an operator acts. For example, in expressions
%
\begin{equation*}
  \xall{x}{\RR}{x^2 + y \geq 0},
  \qquad\qquad
  \sum_{k = 0}^n \frac{1}{k^2},
  \qquad\qquad
  \int_a^b f(t) \, dt,
\end{equation*}
%
the variables $x$, $k$, and $t$ are bound by the operators $\forall$,
$\sum$, and $\int$, respectively. The remaining variables are free. It
is really the \emph{occurrence} of a variable that is bound or free,
not the variable itself. In
%
\begin{equation*}
  P(x) \lor \xusome{x}{\lnot Q(x)}
\end{equation*}
%
the left-most occurence of $x$ is free whereas the other two are bound
by $\exists$.

\paragraph{Functions.}

The set of all functions from $A$ to $B$ is denoted by $B^A$ or $A \to B$.

\paragraph{Partial functions.}

A \emph{partial} function\footnote{In the literature on Type Two
  Effectivity the common notation is $f \mathbin{{:}{\subseteq}} A \to
  B$.} $f: A \parto B$ is a function that is defined on a subset
$\dom{f} \subseteq A$, called the \emph{domain} of~$f$. Sometimes
there is confusion between the domain~$\dom{f}$ and the set~$A$, which
is also called the domain. In such cases we call $\dom{f}$ the
\emph{support} of~$f$. If $f: A \parto B$ is a partial function and $x
\in A$, we write $f x \defined$ to indicate that $f x$ is defined. For
an expression~$e$, we also write $e \defined$ to indicate that~$e$ and
all of its subexpressions are defined. The symbol~$\defined$ is
sometimes inserted into larger expressions, for example, $f x \defined
= y$ means that $f x$ is defined and is equal to~$y$. If $e_1$ and
$e_2$ are two expressions whose values are possibly undefined, we
write $e_1 \kleq e_2$ to indicate that either $e_1$ and $e_2$ are both
undefined, or they are both defined and equal.




%%% Local Variables: 
%%% mode: latex
%%% TeX-master: "notes"
%%% End: 
